\documentclass[11pt,a4paper,oneside]{memoir}

% Packages
\usepackage[a4paper,margin=2.5cm, top=1.5in, bottom=1.5in]{geometry}
\usepackage{graphicx}
\usepackage{enumerate}
\usepackage{amsmath}
\usepackage{amssymb}
\usepackage{amsfonts}
\usepackage{amsthm}
\usepackage{tikz}
\usepackage{float}
\usepackage[T1]{fontenc}
\usepackage[
    colorlinks=true,
    %linkcolor=blue, urlcolor=blue, citecolor=blue % pdf
    linkcolor=black, urlcolor=black, citecolor=black % print
]{hyperref}

\usetikzlibrary{positioning}
\usetikzlibrary{arrows.meta}

% Look for images under the .images/ dir
\graphicspath{ {./images/} }
% Specifiy stype for bibliography
\bibliographystyle{abbrv}

% Theorem and Proposition styling
\theoremstyle{plain}
% Reset theorem numbering for each chapter
\newtheorem{thm}{Theorem}[chapter]
% Reset definition numbering for each chapter
\newtheorem{prop}[thm]{Proposition}
% Reset definition numbering for each chapter
\newtheorem{lem}[thm]{Lemma}

% Definition and Example styling
\theoremstyle{definition}
% Definition numbers are dependent on theorem numbers
\newtheorem{defn}[thm]{Definition}
% Example numbers are dependent on theorem numbers
\newtheorem{exmp}[thm]{Example}

\newcommand{\mmod}[1]{\ (\mathrm{mod}\ #1)}

\begin{document}

% -------------- TITLE PAGE --------------
\begin{titlingpage}
    \centering
 
        \Huge
        \parbox{10cm}{\begin{center}{\scshape \textbf{Chaos in Topological Dynamical Systems}}\end{center}}
             
        \vspace{1.25cm}

        \large
        by

        \vspace{1cm}
        
        \LARGE
        {\scshape \textbf{Fraser Robert Love}}
             
        \vspace{1.5cm}
             
        \large
        School of Mathematics and Statistics\\
        University of St Andrews\\

        \vfill

        {\large \today\par}
 \end{titlingpage}
% ------------ TITLE PAGE END ------------

% Set correct spacing between lines
\OnehalfSpacing

\noindent\textit{I certify that this project report has been written by me, is a record of work carried out by me, and is essentially different from work undertaken for any other purpose or assessment.}

\begin{center}
\includegraphics[width=1.3cm]{signature}
\end{center}

\vspace{1cm}

\begin{abstract}
    \noindent A topological dynamical system is a specific type of discrete dynamical system where the set it is acting on is a compact metric space and which often give rise to complex and chaotic behaviour. But what does it mean for a topological dynamical system to be chaotic and how does chaos arise? This project explores various definitions of chaos applied to topological dynamical systems and the axioms they require. Specifically we will cover the various interpretations of chaos from Devaney, Lyapunov and Li and Yorke. This text follows a strictly analytic and topological approach within the definative bounds of a topological dynamical system. We shall encounter a wide variety of chaotic behaviors, such as topological transitivity, sensitive dependence on initial coditions, the density of periodic points, scrambled sets and topological entropy. Finally the text shall conclude by characterising various chaotic systems before the author gives their own definition of chaos.
\end{abstract}

\newpage
\tableofcontents

\chapter{An Introduction to Topological Dynamics}
This aim of this text is to introduce the reader to topological dynamical systems and explore the various interpretations of chaos and their consequences. We shall be tackling chaos through the lense of topology and topological dynamical systems. A discrete dynamical system is defined by a metric space with a corresponding continuous function, mapping the metric space to itself. The function itself is termed a map or mapping whereby points in the underlying metric space are mapped to other points in the set by the application of this function. Topological dynamical systems themselves are a subset of discrete dynamical systems, with the extra requirement that the underlying metric space be compact (i.e.\ closed and bounded). This extra condition for compactness is useful for investigating the limiting behavior of the set of iterates of the map as it is repeatedly iterated to infinity; a relevant feature in the study of chaos. The term chaos itself, specifically deterministic chaos, has various definitions in mathematics and was first coined by Li and Yorke in their ubiquitious paper `Period Three Imples Chaos' \cite{li-yorke}. These numerous definitions provide differing constituents to chaos namely: topological transitivity, existence of a dense orbit, the density of periodic points, existence of an uncountable scrambled set and sensitive dependence on initial conditions. Hence topological dynamical systems can be chaotic according to one interpretation but not another. We shall aim to compare these definitions and understand their consequences, providing examples to topological dynamical systems that exhibit each type of chaos. This text assumes the reader to be a capable student of pure mathematics with a basic understanding of topology and analysis. The focus of this text is mainly topological; for the sake of brevity content from related areas of ergodic theory, group theory, etc.\ are excluded.

This chapter will briefly review some relevant results from topology before introducing ideas central to topological dynamics and the study of chaos in topological dynamical systems. We shall also introduce some popular topological dynamical systems that exhibit chaos, which we will be examining throughout this text. Subsequently, in Chapter 2 we will introduce notions of comparing and equivalating topological dynamical systems using the framework of topological conjugacy and symbolic dynamics. Here we will use these powerful results to prove topological results for the well k   nown tent map and logistic map using topologically conjugate maps. Furthermore, we shall begin our study of chaos via Sharkovsky's theorem. In Chapter 3 we will study the three various interpretations of chaos from Devaney, Lyapunov and Li and Yorke, analysing examples of topological dynamical systems which statisfy each definition. We shall look at the various properties topological dynamical systems require to be considered chaotic and compare the various definitions. Finally in Chapter 4 we will explore the consequences of chaos and characterise the interesting behavior of topological dynamical systems.

\section{Topological Dynamical Systems and Discrete Dynamics} \label{sec:dynsys}
We shall start this chapter by giving the definition of a topological dynamical system, including some important defintions from topology and introducing some preliminary definitions from discrete dynamics. We shall refer back to these definitions constantly for the remainder of this text. Furthermore, prominent examples of topological dynamical systems will be presented. The reader should remember these examples as they will be integral to understanding several propositions in later chapters. Note that the definitions and results in this section apply generally to continuous maps, however have been formulated in terms of topological dynamical systems for clarity and precision. Beginning, we shall introduce some preliminary definitions from topology.

\begin{defn}[Dense Set] \label{defn:dense}
    Let $(Z, d)$ be a metric space and $X, Y \subseteq Z$ where $X \subseteq Y$. The set $X$ is \emph{dense} in $Y$ if $\overline{X} = Y$, i.e.\ if for every $x \in Y$ there exits an open neighbourhood $U$ of $x$ with $y \in U$ such that $y \in X$.
\end{defn}

\begin{defn}[Compact  Space] \label{defn:compact}
    Let $(X, d_x)$ be metric spaces. The metric space $X$ is compact if every open cover $\mathcal{U}$ of $X$ has a finite subcover $\left\lbrace U_{i(1)}, U_{i(2)}, \cdots, U_{i(n)} \right\rbrace \subseteq \mathcal{U}$, i.e.\ if $\left\lbrace U_i \right\rbrace_{i\in I}$ is a collection of open subsets of $X$, where $X \subseteq \bigcup_{i \in I}U_i$ then there exists a finite subcollection $\left\lbrace U_{i(1)}, U_{i(2)}, \cdots, U_{i(n)} \right\rbrace \subseteq \mathcal{U}$ such that $X \subseteq \bigcup_{j = 1}^{n}U_{i(j)}$.
\end{defn}

Note that on $\mathbb{R}$ with the standard metric, all closed and bounded intervals are compact. Now since we now have a notion of what it means for a metric space to be compact we can now define the main object we shall be studying in this paper, the topological dynamical system.

\begin{defn}[Topological Dynamical System] \label{defn:topological-dynamical-system}
    Let $X$ be a non-empty compact metric space. A \emph{topological dynamical system} denoted $(X, f)$ is given by a continuous map $f: X \to X$. The system starts at an initial point $x \in X$ and evolves through successive iterations of the map $f$. After $k \in \mathbb{N}$ iterations of $f$, the system can be described by $f^n := f \circ f \circ \cdots \circ f$, where $f$ is mapped to the point $f^n(x)$. By convention we take $f^0$ to be the identity map.
\end{defn}

Having defined a topological dynamical system, we can now characterise the discrete dynamics of the underlying map through the following definitions.

\begin{defn}[Orbit] \label{defn:orbit}
    Let $(X, f)$ be a topological dynamical system. The \emph{orbit} or \emph{forward orbit} of a point $x \in X$ under $f$ is the set $\mathcal{O}_f(x) = \mathcal{O}^+_f(x) = \lbrace f^n(x) : n \geq 0 \rbrace = \lbrace x, f(x), f^2(x), \cdots \rbrace$ of iterates of $x$ under the map $f$. If $f$ is a homeomorphism (i.e $f^{-1}$ exists and is continuous) then the \emph{backward orbit} of $x$ under $f$ is similarly defined as $\mathcal{O}^-_f(x) = \lbrace f^n(x) : n \leq 0 \rbrace = \lbrace x, f^{-1}(x), f^{-2}(x), \cdots \rbrace$.
\end{defn}

Note that in this text, unless otherwise stated, the term, orbit, will simply refer to the forward orbit, as we will mostly be dealing with forward dynamics.

\begin{defn}[Periodic Point, Cycle] \label{defn:periodic-point}
    Let $(X, f)$ be a topological dynamical system. A point $x \in X$ is \emph{fixed} if $f(x) = x$ and \emph{periodic} if $f^n(x) = x$ for some $n \in \mathbb{N}$. The \emph{period} of a point $x$ is the least positive integer $k$ such that $f^k(x) = x$. If $x$ has a period of $k$ we say that $x$ is a \emph{period-$k$} point. The set of all period-$k$ points of $f$ is denoted by $\text{Per}_n(f)$. Moreover, $f^n(x) = x \iff n = lk$, for some $l \in \mathbb{N}$. The orbit $\mathcal{O}_f(x) = \lbrace x, f(x), \cdots, f^{k-1}(x) \rbrace$ of a periodic point is a finite set of unique points, called a \emph{perodic orbit} of period $k$ or simply a \emph{k-cycle}.
\end{defn}

In most topological dynamical systems only a small subset of points are periodic. Most often a larger set of points either enter a periodic orbit after a certain number of iterations of $f$ or converge asymptotically to a periodic orbit, leading us directly into the following defintions.

\begin{defn}[Eventually Periodic, Asymptotically Periodic] \label{defn:eventually-asymptotically-periodic}
    Let $(X, f)$ be a topological dynamical system. A point $x \in X$ is \emph{eventually periodic} of period $k$ if the point $x$ is not periodic and there exists a $n > 0$ such that $f^{k+i}(x) = f^i(x)$, for $i \geq n$. The point $x \in X$ is \emph{asymptotically periodic} to a periodic point $p \in X$ if $\lim_{n \to \infty} d(f^n(x), f^n(p)) = 0$.
\end{defn}

\begin{prop} \label{prop:eventually-periodic-implies-periodic}
    Let $(X, f)$ be a topological dynamical system. If $f$ is an invertible map, then every eventually periodic point is periodic.
    \begin{proof}
        Suppose $x \in X$ is eventually periodic of period $k$ in $f$. Then $f^{k + i}(x) = f^i(x)$ for some $n > 0$. By applying $i$ iterations of $f^{-1}$ we obtain, $f^{-i} \circ f^{k + i}(x) = f^{-i} \circ f^{i}(x) \implies f^k(x) = x$. Hence $x$ is periodic with period $k$.
    \end{proof}
\end{prop}

When studying chaos in topological dynamical systems it can be useful to understand how the system behaves for an increasing number of iterations. The $\omega$-limit set, defined below as the set of limit points of a particular orbit, allows us to understand the systems behaviour asymptotically.

\begin{defn}[Omega-limit Set] \label{defn:omega-limit-set}
    Let $(X, f)$ be a topolgical dynamical system. The $\omega$\emph{-limit} set of $x \in X$, denoted $\omega(x, f)$ is the set of all limit points of the orbit $\mathcal{O}_f(x)$ given by \[\omega(x, f) := \bigcap_{n=0}^\infty\overline{\left\lbrace f^k(x) : k \geq n \right\rbrace}\] and the $\omega$\emph{-limit} set of the entire map $f$ is defined as \[\omega(f) := \bigcup_{x \in X} \omega(x, f)\]
\end{defn}

By the definition, we can immediately see that the $\omega$-limit set of a period-$k$ point or eventually periodic point of period-$k$ is simply the $k$-cycle. Furthermore, if a point is asymptotically periodic then the $\omega$-limit set is clearly a cycle and hence finite. Now we are done with defining properties of discrete dynamics lets analyse the discrete dynamics of some popular topological dynamical systems.

\begin{exmp}[Tent Map] \label{exmp:tent-map}
    Let $T_s: [0, 1] \to [0,1]$ be the \emph{tent map}, where $T_s(x) = sx$ for $x \in \left[0, \frac{1}{2}\right]$ and $T_s(x) = s(1-x)$ for $x \in \left[\frac{1}{2}, 1\right]$. Let $s \in (1, 2]$. The map $T_2$ is shown in Figure \ref{fig:tent-logistic}. Since $[0, 1]$ is a closed interval it is compact. To find the fixed points let $T_s(x) = x$. If $x \in \left[0, \frac{1}{2}\right]$ then $sx = x \implies x = 0$ is a fixed point for all $s \in (1, 2]$. If $x \in \left[\frac{1}{2}, 1\right]$ then $s(1-x) = x \implies x = \frac{s}{1 + s}$ is a fixed point for all $s \in (1, 2]$. Clearly if $x \in \left[0, \frac{1}{4}\right] \cup \left[\frac{3}{4}, 1\right]$ then $T_s(x) \in \left[0, \frac{1}{2}\right]$, however if $x \in \left[\frac{1}{4}, \frac{1}{2}\right] \cup \left[\frac{1}{2}, \frac{3}{4}\right]$ then $T_s \in \left[\frac{1}{2}, 1\right]$. This behaviour makes calculating periodic points difficult. In Chapter \ref{chap:conjugacy-symbol-dynamics} we shall develop the technique of using symbolic dynamics to better analyse this discrete dynamics of this system.
\end{exmp}

\begin{exmp}[Logistic Map] \label{exmp:logitic-map}
    Let $F_{\mu}: [0, 1] \to [0, 1]$ be the \emph{logistic map}, where $F_{\mu}(x)=\mu x(1-x)$ and $\mu > 0$. The map $F_4$ is shown in Figure \ref{fig:tent-logistic}. Since $[0, 1]$ is a closed interval it is compact. To find the fixed points let $F_{\mu}(x) = x \implies x = 0$ and $x = \frac{\mu - 1}{\mu}$ are fixed points for all $\mu > 0$. It will be shown in Section \ref{sec:logistic_maps} that for $\mu \in (0, 3)$, any $x \in (0, 1)$ will be attracted towards the fixed point of $x = \frac{\mu - 1}{\mu}$.
\end{exmp}

\begin{figure}[h]
    \centering
    \includegraphics[width=4.5cm]{tent_1.3}
    \includegraphics[width=4.5cm]{logistic_2.85}
    \caption{The first iterates of the tent map $T_2$ and logistic map $F_4$ respectively.}
    \label{fig:tent-logistic}
\end{figure}

Depending on the parameters $s$ and $\mu$ the dynamical behaviour of the tent map and logistic map can range from predicitble periodicity to chaotic. In Chapter \ref{chap:conjugacy-symbol-dynamics} we shall subsequently prove that $T_2$ and $F_4$ are similar topologically speaking and share various topological properties. Moreover both of these topological dynamical systems are described by simple mathematical equations, however we shall prove later that they are chaotic and exhibit highly complex and irregular dynamics.

\section{Stability of Topological Dynamical Systems}
The stability of periodic points is integral to the emergence of chaos in a topological dynamical system. Periodic points that are stable attract nearby points and the system settles into a steady state. Unstable periodic points however cause nearby points to diverge and small pertubations can cascade into large differences. Here, we will define an notion of stability for periodic points in topological dynamical systems defined over closed intervals. This section is a preliminary for our the next section on studying the complex dynamics of the logistic family, specifically for $\mu > 3$ as the system becomes chaotic. The following results are adapted from Devaney \cite[Section 1.4]{devaney}. First lets define the notion of hyperbolicity and what it means for a periodic point to be attracting.

\begin{defn} \label{defn:hyperbolic}
    Let $(I, f)$ be a topological dynamical system where $f$ is a smooth map and $I$ is a closed interval. A period-$k$ point $x \in X$ is said to be \emph{hyperbolic} if $|(f^k)'(x)| \neq 1$.
\end{defn}

\begin{prop} \label{prop:attracting-point}
    Let $(I, f)$ be a topological dynamical system where $f$ is a smooth map and $I$ is a closed interval. If $p \in I$ is a hyperbolic period-$k$ point with $|(f^k)'(p)| < 1$, then there exists an open interval $U \subseteq I$ with $p \in U$ such that if $x \in U$, then \[ \lim_{n \to \infty} f^n(x) = p \]

    \begin{proof}
        Let $p$ be a period-$k$ point. By the Mean Value Theorem, for all $x \in I$, there exists a $c \in I$ such that \[f^k(x) - f^k(p) = f^k(x) - p = (f^k)'(c)(x-p)\] Since $f'$ is continuous at $c$, there exists $\varepsilon > 0$ such that $|(f^k)'(c)| < A < 1$ for $c \in U = [p - \varepsilon, p + \varepsilon]$ and some $A \in I$. Hence for some $x \in U$ \[|f^k(x) - p| = |f^k(x) - f^k(p)| \leq A|x-p| < |x-p| \leq \varepsilon\] Thus $f^k(x) \in U = [p - \varepsilon, p + \varepsilon]$. Moreover, since $|f^k(x) - p| < |x - p|$, $f^k(x)$ is closer to $p$ than $x$ is. Applying this same argument inductively we deduce that if $n = mk$ for some $m \in \mathbb{N}$ \[|f^n(x) - p| = |f^{mk}(x) - p| \leq A^m|x - p|\] Now since $A < 1$, $\lim_{m \to \infty}A^m = 0$. Hence $f^n(x) \to p$ as $n \to \infty$. 
    \end{proof}
\end{prop}

Finally with a sense of what it means for periodic points to attract nearby points we can make the following definition.

\begin{defn} \label{def:attractor}
    Let $(I, f)$ be a topological dynamical system where $f$ is a smooth map and $I$ is a closed interval. Suppose $p \in I$ is a hyperbolic period-$k$ point with orbit $\mathcal{O}(p) = \lbrace p, f(p), \cdots, f^k(p) \rbrace$. Then the point $p$ is called an \emph{attracting periodic point} or an \emph{attractor} if \[|(f^k)'(p)| = \left\lvert \prod_{n = 1}^k f'(p_k) \right\rvert < 1\]
\end{defn}

Now we have a notion of what it means for periodic points to be attracting. However we require for the hyperbolic period-$k$ point $p$ to have the property $|(f^k)'(p)| < 1$. Now we can ask what happens conversely when $|(f^k)'(p)| > 1$.

\begin{prop} \label{prop:repelling-point}
    Let $(I, f)$ be a topological dynamical system where $f$ is a smooth map and $I$ is a closed interval. If $p \in I$ is a hyperbolic period-$k$ point with $|(f^k)'(p)| > 1$. Then there is an open interval $U \subseteq I$ with $p \in U$ such that if $x \in U$, $x \neq p$, then there exists $n > 0$ such that $f^n(x) \notin U$.

    \begin{proof}
        Let $p$ be a period-$k$ point. By the Mean Value Theorem, for all $x \in \mathbb{R}$, there exists a $c \in \mathbb{R}$ such that \[f^k(x) - f^k(p) = f^k(x) - p = (f^k)'(c)(x - p)\] Since $f'$ is continuous at $c$, there exists $\varepsilon > 0$ such that $|(f^k)'(c)| > A > 1$ for $c \in U = [p - \varepsilon, p + \varepsilon]$ and some $A \in \mathbb{R}$. Hence for some $x \in U$ \[|f^k(x) - p| = |f^k(x) - f^k(p)| \geq A|x - p| \geq |x - p| \geq \varepsilon\] Since $|f^k(x) - p| > |x - p|$, $f^k(x)$ is further away from $p$ than $x$ is. If $f^k(x) \notin U = [p - \varepsilon, p + \varepsilon]$ we are done. However if not then we can apply the same argument inductively with $n = mk$, $m \in \mathbb{N}$ to obtain \[ |f^n(x) - p| = |f^{mk}(x) - p| \geq A^m|x - p|\] Now since $A > 1$, $\lim_{m \to \infty}A^m = \infty$ and $f^n(x) \to \infty$ as $n \to \infty$. Hence, eventually $f^n(x) \notin U$ for some $n \in \mathbb{N}$.
    \end{proof}
\end{prop}

Now that we have a sense of what it means for periodic points to repel nearby points we can make the following definition.

\begin{defn} \label{def:repellor}
    Let $(I, f)$ be a topological dynamical system where $f$ is a smooth map and $I$ is a closed interval. Suppose $p \in I$ is a hyperbolic period-$k$ point with orbit $\mathcal{O}(p) = \lbrace p, f(p), \cdots, f^k(p) \rbrace$. Then the point $p$ is called a \emph{repelling periodic point} or a \emph{repellor} if \[|(f^k)'(p)| = \left\lvert \prod_{n = 1}^k f'(p_k) \right\rvert > 1\]
\end{defn}

With a solid understanding of what it means for a periodic point to be a repellor or an attractor we can start to look for such points in topological dynamical systems defined over a closed interval. A family of such systems we have already came across is the family of logistic maps.

\section{The Family of Logistic Maps} \label{sec:logistic_maps}
Chaos can arise in the simplest of topological dynamical systems. The family of topological systems $([0, 1], F_{\mu})$ defined by the logistic maps $F_\mu(x) = \mu x(1-x)$ are non-linear topological systems which exibit chaotic phenomena. Using the definitions and theorems we have developed above we can characterise the behaviour of this family of topological systems and identify how this chaos arises. This section follows analysis performed by Devaney \cite[Section 1.5]{devaney} into the behaviour of these systems.

\begin{prop} \label{prop:logistic-properties}
    The topological dynamical system $([0,1], F_{\mu})$ described by $F_\mu(x) = \mu x (1-x)$ has the following properties.
    \begin{itemize}
        \item[(i)]$F_\mu$ has fixed points at $x = 0$ and $x = x_{\mu} = \frac{\mu - 1}{\mu}$
        \item[(ii)] If $1 < \mu < 3$ then $F_\mu$ has an attracting fixed point at $x = x_\mu = \frac{\mu - 1}{\mu}$ and a repelling fixed point at $x = 0$.
        \item[(iii)] If $1 < \mu < 3$ then $\lim_{n \to \infty} F_\mu^n(x) = x_\mu$.
        \item[(iv)] If $3 < \mu < 1 + \sqrt{6}$ then the fixed point $x = x_\mu = \frac{\mu - 1}{\mu}$ is now a repellor.
        \item[(v)] If $3 < \mu < 1 + \sqrt{6}$ then a new period-2 attractor emerges at \[x_{\mu_2} = \frac{\sqrt{\mu^2 - 2\mu - 3} + \mu + 1}{2\mu}\] 
    \end{itemize}
    \begin{proof}[Proof (i)]
        See Example \ref{exmp:logistic-fixed-points}
    \end{proof}
    \begin{proof}[Proof (ii)]
        By part (i) $F_\mu(x)$ has fixed points at $x = 0$ and $x_\mu = (\mu - 1) / \mu$. Here $F_\mu'(x) = \mu - 2\mu x$ so $F_\mu'(0) = \mu$ and $F_\mu'(x_\mu) = 2 - \mu$. Hence if $1 < \mu < 3$ then $|F'_\mu(0)| > 1$ and $|F'_\mu(x_\mu)| < 1$. Therefore by Definition \ref{def:attractor} and Definition \ref{def:repellor}, $x = 0$ is a repellor and $x = x_\mu$ is an attractor.
    \end{proof}
    \begin{proof}[Proof (iii)]
        From Part (i) $x_\mu$ is an attracting fixed point. By Proposition \ref{prop:attractor} we obtain $\lim_{n \to \infty}F_\mu^n(x) = x_\mu.$
    \end{proof}
    \begin{proof}[Proof (iv)]
        Since $\mu > 3$, $|F'_\mu(x_\mu)| = |2-\mu| > 1$. Hence by Definition \ref{def:repellor} the fixed point $x_\mu$ is a repelling fixed point.
    \end{proof}
    \begin{proof}[Proof (v)]
        Solving $F^2_\mu(x) = \mu(\mu x(1-x))(1-\mu x(1-x)) = x$ we find that $x = 0$, $x = x_\mu = \frac{\mu - 1}{\mu}$ and $x = x^\pm_{\mu_2} = \frac{\pm\sqrt{\mu^2 - 2\mu - 3} + \mu + 1}{2\mu}$ where $x^\pm_{\mu_2}$ dentotes the solutions with the positive an negative sign respectively. Hence, $ \left\lvert (F_\mu^2)'(x_{\mu_2})\right\rvert = \left\lvert F'(x^+_{\mu_2}) F'(x^-_{\mu_2}) \right\rvert = \left\lvert \left( - \sqrt{\mu^2 - 2\mu - 3} - 1 \right) \left( \sqrt{\mu^2 - 2\mu - 3} - 1\right) \right\rvert = \left\lvert -\mu^2 + 2\mu + 4 \right\rvert $
        Therefore if $3 < \mu < 1 + \sqrt{6}$ then $\left\lvert -\mu^2 + 2\mu + 4 \right\rvert < 1$ and so $x_{\mu_2}$ is a period-2 attractor.
    \end{proof}
\end{prop}

By the above result it is clear that for $1 < \mu < 3$, $F_\mu(x)$ only has two fixed points at $x = 0$ and $x_\mu$ and that all points $x \in (0, 1)$ are forward asymptotic to $x_\mu$. Hence the $\omega$-limit set of $F_\mu(x)$ is simply $\omega(f) = \left\lbrace 0, 1 \right\rbrace \cup x_\mu$.

\begin{figure}[h]
    \centering
    \includegraphics[width=5.5cm]{cobweb_0.1_2.9.pdf}
    \includegraphics[width=5.5cm]{cobweb_0.1_3.1.pdf}
    \caption{Successive iterates of the logitic family $F{\mu}(x) = \mu x(1-x)$ for $\mu = 2.9$ and $\mu = 3.1$ respectively, displaying the long term behaviour of the orbit of $x_0 = 0.1$. When $\mu = 2.9$ an attracting fixed point is clearly visible and when $\mu = 3.1 > \mu_1$ an attracting period 2 point can be seen.}
    \label{fig:cobweb_2.9_3.1}
\end{figure}

For $3 < \mu < 1 + \sqrt{6}$ the attracting fixed point at $x = x_\mu$ decays into a repellor and a period-2 attractor emerges around the point $\mu_1 = 3$. This can been seen clearly in Figure \ref{fig:cobweb_2.9_3.1} which shows the long term stability of the orbit. This behaviour we have just outlined is called a \emph{bifurcation}, specifically a \emph{period-doubling bifurcation} and the point $\mu_1 = 3$ is called a \emph{bifurcation point}. The next bifurcation point happens at $\mu_2 = 1 + \sqrt{6}$ whereby the stable period-2 attractor decays into a repellor and a period-4 attractor emerges. It turns out that there are many bifurcations as $\mu$ increases towards value of four. The value of $\mu$ at which the $k$-th bifurcation occurs is denoted $\mu_k$. The bifurcations occur ever closer together in what is called a \emph{period-doubling cascade}, until the whole system becomes in some sense chaotic. Figure \ref{fig:bifurcation_2.8} shows what is called a bifurcation diagram, which plots the stable orbits of $x$ as $r$ increases. From the bifurcation diagram we can clearly see the initial bifurcation at $\mu_1 = 3$ whereby the initial attracting fixed point $x = x_\mu$ decays into a repellor and the period-2 attractor $x = x_{\mu_2}$ emerges. Again the second bifurcation at $\mu_2 = 1 + \sqrt{6}$ is shown whereby the period-2 attractor $x = x_{\mu_2}$ decays into a repellor and a period-4 attractor emerges. These bifurcations cascade as $\mu$ increases until at a certain point called the \emph{accumulation point} ($\mu_\infty = \mu \approx 3.56995$) whereby periodicity transforms suddenly into chaos as slight pertubations in the starting value $x$ cascade into large changes over a iterations of $F_\mu$. A chaotic orbit for when $\mu > \mu_\infty$ can be seen in Figure \ref{fig:cobweb_3.5_3.9}. An interesting property of the period-doubling cascade is the ratio between bifurcation interval to the next bifurcation interval between each period doubling calculated using \[\delta_k  = \frac{\mu_k - \mu_{k-1}}{\mu_{k+1}-\mu_k}\]It turns out that as $k \to \infty$ this ratio converges to a constant $\delta \approx 4.66921$ called the \emph{Feigenbaum constant}. A remarkable discovery is that this constant holds for the bifurcations of every one-dimensional map with a only one quadratic maximum. Looking back at the logistic map a rather strange result from the bifurcation diagram is the presense of small isolated intervals of stability present amongst the chaos. These intervals are referred to as \emph{islands of stability}. The point $\mu = 1 + 2\sqrt{2}$ is the start of one such interval in which a period-3 attractor emerges from the chaos. However, once again a period-doubling cascade happens and the system decends back into chaos. For $\mu > 4$ no attractors exist and the whole system becomes unstable.
\begin{figure}[h]
    \centering
    \includegraphics[width=14cm]{bifurcation_2.8.png}
    \caption{Bifurcation diagram of the logitic family $F{\mu}(x) = \mu x(1-x)$ for $2.8 \leq r \leq 4.0$.}
    \label{fig:bifurcation_2.8}
\end{figure}

\begin{figure}[h]
    \centering
    \includegraphics[width=5.5cm]{cobweb_0.1_3.5.pdf}
    \includegraphics[width=5.5cm]{cobweb_0.1_3.9.pdf}
    \caption{Successive iterates of the logitic family $F_{\mu}(x) = \mu x(1-x)$ for $\mu = 3.5$ and $\mu = 3.9$ when $x_0 = 0.1$ respectively. When $\mu = 3.5 > \mu_2$ an attracting period-4 point is visible. For $\mu = 3.9 > \mu_\infty$ $F_\mu$ has infinite periodicity and has become visibly chaotic.}
    \label{fig:cobweb_3.5_3.9}
\end{figure}

\chapter{Topological and Symbolic Relationships} \label{chap:conjugacy-symbol-dynamics}

\section{Topological Conjugacy}
Topological conjugacy defines when two maps that exhibit the same topolgical behaviour and can be considered equivalent. If two maps are topologically conjugate, properties we have proved for one map can be applied to the other.

\begin{defn} \label{defn:topological-conjugacy}
    Let $(X, f)$ and $(Y, g)$ be discrete dynamical systems. The system $(Y, g)$ is \emph{topologically semi-conjugate} to $(X, f)$ if there exists a continuous, surjective map $\varphi: X \to Y$ termed a \emph{topological conjugacy} where $\varphi \circ f = g \circ \varphi$. Furthermore if the map $\varphi$ is a homeomorphism i.e. $\varphi$ is a bijection with $\varphi$ and $\varphi^{-1}$ both continuous then $(Y, g)$ is \emph{topologically conjugate} to $(X, f)$. In this case $\varphi$ is called a \emph{topological conjugacy}.
\end{defn}

\begin{center}
\begin{tikzpicture}
    \node(lt) {$X$};
    \node(rt) [right=of lt] {$X$};
    \node(lb) [below=of lt] {$Y$};
    \node(rb) [below=of rt] {$Y$};

    \draw[->] (lt.east) -- node[above] {$f$} (rt.west);
    \draw[->] (lb.east) -- node[above] {$g$} (rb.west);
    \draw[->] (lt.south) -- node[left] {$\varphi$} (lb.north);
    \draw[->] (rt.south) -- node[right] {$\varphi$} (rb.north);
\end{tikzpicture}
\end{center}

Topological conjugacy can be used for infering the properties of one topological system from another. For instance if the continuous maps $f$ and $g$ are topologically conjugate through some map $\varphi$ then if $x$ is a fixed point of $f$, $\varphi(x)$ is a fixed point of $g$. This can be seen mathematically as $\varphi(x) = \varphi \circ f(x) = g \circ \varphi(x)$. This leads us directly into the following lemma.

\begin{prop} \label{prop:conjugacy-preserves-periodic-points}
    Let $(X, f)$ and $(Y, g)$ be discrete dynamical systems and let $\varphi: X \to Y$ be a topological conjugacy. If $x$ is a period-$k$ point of $f$ then $\varphi(x)$ is a period-$k$ point of $g$.
    \begin{proof}
        Suppose $f^k(x) = x$, then by induction $\varphi(x) = \varphi \circ f^k(x) = g^k \circ \varphi (x)$. Since $\varphi$ is invertible $f^k(x) = \varphi \circ f^k \circ \varphi^{-1}(x)$. Now suppose $g^k(\varphi(x)) = \varphi(x)$, then $x = \varphi \circ \varphi^{-1}(x) = g^k \circ \varphi \circ \varphi(x)^{-1} = \varphi \circ f^k \circ \varphi^{-1}(x) = f^k(x)$.
    \end{proof}
\end{prop}

Therefore, to understand the periodic points of one dynamical system we can analyse the the periodic points of another system which is topologically conjugate to it. This becomes useful when dealing with dynamical sytems which have particularly complex behaviour such as the logistic maps, leading us into our next proposition.

\begin{prop} \label{prop:logisticquadratic}
    The topological dynamical system $([0, 1], F_{\mu})$ described by $F_\mu(x) = \mu x(1-x)$ and the discrete dynamical system $(\mathbb{R}, Q_c)$ described by $Q_c(x) = x^2 + c$ are topologically conjugate when $c = \frac{\mu}{2} - (\frac{\mu}{2})^2$.

    \begin{proof}
        Let $F_{\mu} : [0, 1] \to [0, 1]$, $F_\mu(x) = \mu x(1-x)$ be the logistic map and $Q_c: \mathbb{R} \to \mathbb{R}$, $Q_c(x) = x^2 + c$ be the quadratic map with $\mu > 0$. Suppose the conjugacy is of the form $\varphi = \alpha x + \beta$. Now $\varphi \circ F_\mu = Q_c \circ \varphi \iff \alpha(\mu x(1-x)) + \beta = (\alpha x + \beta)^2 + c \iff -\mu \alpha x^2 + \alpha \mu x + \beta = \alpha ^ 2 x^2 + 2\alpha\beta x+ \beta ^ 2 + c$. Collecting terms $-\mu \alpha = \alpha ^ 2 \implies -\mu = \alpha, \ \alpha\mu = 2\alpha\beta \implies \mu = 2\beta$ and $\beta = \beta^2 + c$. Hence $\varphi = -\mu x + \frac{\mu}{2}$ is the topological conjugacy when $c = \frac{\mu}{2} - (\frac{\mu}{2})^2$.
    \end{proof}
\end{prop}

Now we can use the fact that these maps are topologically conjugate to go back and prove the following interesting result about the logistic map.

\begin{prop}
    The topological dynamical system $([0, 1], F_{\mu})$ has a period-$3$ point when $\mu = 1 + 2\sqrt{2}$.
    \begin{proof}
    Let $Q(x) = x^2 + c$. Then $Q^3(x) = Q^2(x^2 + c) = Q((x^2 + c)^2 + c) = Q(x^4 + 2cx^2 + c^2 + c) = (x^4 + 2cx^2 +c^2 + c)^2 + c = x^8 + 4cx^6 + (6c^2 + 2c)x^4 + 4c(c^2 + c)x^2 + c^4 + 2c^3 + c^2 + c$. To find period-$3$ points we set $Q^3(x) = x$. Now solving this by dividing by a factor of $x^2 -x + c$ we obtain the following solution \[x^6 + x^5 + (3c + 1)x^4 + (2c + 1)^3 + (3c^2 + 3c + 1)x^2 + (c^2 + 2c + 1)x + (c^3 + 2c^2 + c) = 0\] It can be shown that when $c = -7/4$ we can complete the square to obtain the simpler $(x^3 -\frac{x^2}{2} - \frac{9x}{4} - \frac{1}{8})^2 = 0$ which has precisely three real roots. Using Proposition \ref{prop:logisticquadratic} the value of $c = -7/4$ corresponds to $\mu = 1 + 2\sqrt{2}$. Since $\varphi$ is a topological conjugacy between $F_\mu$ and $Q_c$ for this value of $c$, $F_\mu$ has a period-3 point for $\mu = 1 + 2\sqrt{2}$.
    \end{proof}
\end{prop}

Hence we have used topological conjugacy along with properties of the quadratic maps to prove results about the logistic maps. When $\mu = 4$ the logisitic map is topologically conjugate to another common map, the tent map.

\begin{prop} \label{prop:tent-logistic}
    The topological dynamical systems $([0, 1], F_4)$ and $([0, 1], T)$ where $T$ is the tent map are topologically conjugate.
    \begin{proof}
    The tent map $T: [0, 1] \to [0,1]$ is defined piecewise as $T(x) = 2x$ for $0 \leq x \leq \frac{1}{2}$ and $T(x) = 2(1-x)$ for $\frac{1}{2} \leq x \leq 1$. Suppose $\varphi: [0, 1] \to [0,1]$ is defined by $\varphi(x) = \sin^2(\frac{\pi x}{2})$ which is homeomorphic on $[0, 1]$ as is a continuous, bijective and $\varphi^{-1} = \frac{2}{\pi} \sin^{-1}(\sqrt{x})$ exists and is continuous on $[0, 1]$. Clearly $F_4 \circ \varphi = 4 \sin^2\left(\frac{\pi x}{2}\right) \cdot \left(1 - \sin^2\left(\frac{\pi x}{2}\right)\right) = \sin^2\pi x$ for $x \in [0, 1]$, $\varphi \circ T = \sin^2\pi x$ for $x \in [0, \frac{1}{2}]$ and $\varphi \circ T = \sin^2 (\pi (1-x)) = \sin^2 \pi x$. Hence we have shown that $\varphi \circ T = F_4 \circ \varphi$ so $F_\mu$ and $T$ are topologically conjugate when $\mu = 4$.
    \end{proof}
\end{prop}

Topological conjugacy preserves a wide range of topological properties of maps. A specific property that topological conjugacy preserves, which will prove useful is the following.

\begin{prop}
    Let $(X, f)$ and $(Y, g)$ be topological dynamical systems and $\varphi: X \to Y$ be a topological conjugacy. If $f$ has a dense orbit in $X$ then $g$ has a dense orbit in $Y$.
    \begin{proof}
        Before we begin note that, from topology, if $f: X \to Y$ is a continuous surjective function and $\overline{E} = X$ then $\overline{f(E)} = Y$. Now suppose $x \in X$ has a dense orbit in $f$, so $\overline{\mathcal{O}_f(x)} = X$. Since $\varphi$ is a homeomorphism $\varphi$ is continuous and surjective, hence $\overline{\mathcal{O}_{\varphi \circ f}(x)} = \overline{\mathcal{O}_{g \circ \varphi}(x)} = \overline{\mathcal{O}_{g}(\varphi(x))} = Y$. Hence $g$ has a dense orbit in $Y$.
    \end{proof}
\end{prop}

Hence we can now prove that $F_4$ has periodic points that are dense in $[0, 1]$ since $F_4$ is conjugate to the tent map $T$. This will prove useful later when characterising chaos.

\begin{prop} \label{prop:logisitc-periodic-dense}
    The topological dynamical system $([0, 1], F_4)$ has periodic points that are dense in $[0, 1]$.
    \begin{proof}
    Let $T(x)$ denote the tent map. Let $\left[\frac{k}{2^n}, \frac{k+1}{2^n}\right] \subseteq [0, 1]$ be an interval for some $0 \leq k \leq 2^{n-1}$ where $n \in \mathbb{N}$. Moreover $T^n$ has $2^n$ fixed points and so the tent map has $2^n$ period-$n$ points. Hence there is a periodic point in each interval $\left[\frac{k}{2^n}, \frac{k+1}{2^n}\right]$ and so the periodic points of $T$ are dense in $[0, 1]$. Therefore by the conjugacy mentioned in Proposition \ref{prop:tent-logistic} the periodic points of the logisitic map $F_4$ are dense in $[0, 1]$.
    \end{proof}
\end{prop}

In the next section on symbolic dynamics we shall see how we can further characterise the complex dynamics of the tent map and doubling map to infer specific properties about them.

\section{Symbolic Dynamics}
Symbolic dynamics studies how the shift map effects infinite sequences of symbols that describe the complex dynamical structure of specific discrete dynamical systems. More specifically it is the assignment of a sequence of discrete symbols to the orbits of topological dynamical systems. First lets start with some preliminary definitions given by Devaney \cite[Section 1.6]{devaney}.

\begin{defn}
    Let $(I, f)$ be a topological dynamical system where $I$ is a closed interval. Let $x \in I$ and suppose $I = I_0 \cup I_1 \cup \cdots \cup I_n$ where $I_i$ are closed intervals. The \emph{itinerary} of $x$ is the sequence $S(x) = s_1s_2\cdots$ where $s_i$ denotes the interval $I_i$ such that $f^i(x) \in I_i$.
\end{defn}

\begin{defn}
    The set $\Sigma_2 = \left\lbrace (s_1s_2\cdots): s_i \in \left\lbrace 0, 1 \right\rbrace \right\rbrace$ is called the \emph{sequence space} on the symbols 0 and 1. The set $\Sigma_2$ consists of infinite sequences (itineries) of these symbols. Let $(s)_{i=1}^{\infty} = (s_1s_2\cdots)$ and $(t)_{i=1}^{\infty} = (t_1t_2\cdots) \in \Sigma_2$. Define the metric for this space to be $d(s, t) = \Sigma_{i=1}^{\infty}|s_i - t_i|2^{-i}$.
\end{defn}

\begin{defn}
    Let $(s)_{i=1}^{\infty} = (s_1s_2s_3\cdots) \in \Sigma_2$. The \emph{shift map} $\sigma: \Sigma_2 \to \Sigma_2$ is given by $\sigma \left((s)_{i=1}^{\infty}\right) = \left((s)_{i=2}^{\infty}\right) = (s_2s_3s_4\cdots)$.
\end{defn}

\begin{prop}
    The shift map $\sigma: \Sigma_2 \to \Sigma_2$ is continuous.
    \begin{proof}
        Let $\varepsilon > 0$ and suppose $\underline{s} = (s)_{i=1}^{\infty}, \ \underline{t} = (t)_{i=1}^{\infty} \in \Sigma_2$. Choose $\delta = \varepsilon$ and suppose $d(\underline{s}, \underline{t}) = \Sigma_{i=1}^{\infty}|s_i - t_i|2^{-i} < \delta$. Then $d\left( \sigma\left(\underline{s}\right) - \sigma\left(\underline{t}\right) \right) = d\left(\left((s)_{i=2}^{\infty}\right) - \left((t)_{i=2}^{\infty}\right)\right) = \Sigma_{i=2}^{\infty}|s_i - t_i|2^{-i} \leq \Sigma_{i=1}^{\infty}|s_i - t_i|2^{-i} < \delta = \varepsilon$.
    \end{proof}
\end{prop}

\begin{defn}
    Let $T: [0, 1] \to [0, 1]$ be the \emph{tent map}, where $T(x) = 2x$ for $0 \leq x \leq \frac{1}{2}$ and $T(x) = 2(1-x)$ for $\frac{1}{2} \leq x \leq 1$.  Assign each $x \in [0, 1]$ to an itinerary $\Sigma_2 = \left\lbrace(s_0s_1s_2\cdots) : s_i \in \left\lbrace0, 1\right\rbrace \right\rbrace$ where the $i$th term of the sequence $s_i = l$ if $x \in \left[0, \frac{1}{2} \right]$ or $s_i = r$ if $x \in \left[\frac{1}{2}, 1 \right]$.
\end{defn}

\begin{defn}
    Let $D: [0,1] \to [0,1]$ be the \emph{doubling map}, where $D(x) = 2x \mmod 1$ or $D(x) = 2x$ for $0 \leq x \leq \frac{1}{2}$ and $D(x) = 2x - 1$ for $\frac{1}{2} \leq x \leq 1$. Assign each $x \in [0, 1]$ to an itinerary $\Sigma_2 = \left\lbrace(s_0s_1s_2\cdots) : s_i \in \left\lbrace0, 1\right\rbrace \right\rbrace$ where the $i$th term of the sequence $s_i = l$ if $x \in \left[0, \frac{1}{2} \right]$ or $s_i = r$ if $x \in \left[\frac{1}{2}, 1 \right]$.
\end{defn}

\begin{figure}[h]
    \centering
    \includegraphics[width=4.5cm]{tent_symbolic}
    \includegraphics[width=4.5cm]{doubling_symbolic}
    \caption{The tent map $T(x)$ and doubling map $D(x)$ respectively with corresponding intervals described above.}
    \label{fig:tent-doubling}
\end{figure}

Symbolic dynamics has an elegant application to various maps such as the tent map and doubling map shown in Figure \ref{fig:tent-doubling}. Now we have outlined their itineries we may naturally question the uniqueness of their itineries and if each itinerary necessarily defines a point. We shall first investigate the doubling map. The nature of the doubling map makes binary expansions a suitable choice for expressing points in this map. For each $x \in [0, 1]$ we can write $x=\sum_{i=1}^{\infty}b_i2^{-i}$ where $b_i \in \left\lbrace 0, 1 \right\rbrace$. Our first propostion describes an important relationship between the doubling map and the shift map which we shall expolore through the use of binary expansions.

\begin{prop} \label{prop:doubling-shift}
    The doubling map $D: [0, 1] \to [0, 1]$ and the shift map $\sigma: \Sigma_2 \to \Sigma_2$ are semi-conjugate via the semi-conjugacy $\varphi: \Sigma_2 \to [0, 1]$.
    \begin{proof}
        Let $\underline{b} = (b_i)_{i=1}^{\infty} \in \Sigma_2$ be a sequence of binary digits and let $\underline{c} = \sigma\left((b_i)_{i=1}^{\infty}\right) = (c_i)_{i=1}^{\infty} \in \Sigma_2$ where $c_i = b_{i + 1}$ be the sequence $\underline{b}$ shifted once. Define the map $\varphi: \Sigma_2 \to [0, 1]$ where $\varphi\left((b_i)_{i=1}^{\infty}\right) = \sum_{i=1}^{\infty} b_i2^{-i} \in [0, 1]$. Let $\sigma: \Sigma_2 \to \Sigma_2$ be the shift map. The function $\varphi$ is surjective, as every point $x \in [0, 1]$ has at least one binary expansion denoted $x=\sum_{i=1}^{\infty}b_i2^{-i}$ where $b_i \in \left\lbrace 0, 1 \right\rbrace$.
        \begin{center}
            \begin{tikzpicture}
                \node(lt) {$\Sigma_2$};
                \node(rt) [right=of lt] {$\Sigma_2$};
                \node(lb) [below=of lt] {$[0, 1]$};
                \node(rb) [below=of rt] {$[0, 1]$};
            
                \draw[->] (lt.east) -- node[above] {$\sigma$} (rt.west);
                \draw[->] (lb.east) -- node[above] {$D$} (rb.west);
                \draw[->] (lt.south) -- node[left] {$\varphi$} (lb.north);
                \draw[->] (rt.south) -- node[right] {$\varphi$} (rb.north);
            \end{tikzpicture}
        \end{center}
        Now we need to prove the above diagram is commutative. Using an arbitrary $\underline{b} \in \Sigma_2$, $\varphi \circ \sigma\left((b_i)_{i=1}^{\infty}\right) = \varphi\left((c_i)_{i=1}^{\infty}\right) = \sum_{i=1}^{\infty} c_i2^{-i} = \sum_{i=1}^{\infty} b_{i+1}2^{-i} \in [0, 1]$. Similarly $D \circ \varphi\left((b_i)_{i=1}^{\infty}\right) = D\left(\sum_{i=1}^{\infty} b_i2^{-i}\right) = \sum_{i=1}^{\infty} 2b_i2^{-i} \mmod 1 = b_1 + \sum_{i=2}^{\infty} b_i2^{-i+1} \mmod 1 = \sum_{i=2}^{\infty} b_i2^{-i+1} = \sum_{j=1}^{\infty} b_{j+1}2^{-j} \in [0, 1]$. Hence we have shown $\varphi \circ \sigma = D \circ \varphi$. and so the doubling map $D$ and the shift map $\sigma$ are semi-conjugate via $\varphi$. We can also see $\varphi$ is not injective as $\frac{1}{2} + \sum_{i=2}^{\infty}0 \cdot 2^{-i} = \sum_{i=2}^{\infty}2^{-i} = \frac{1}{2}$ and so $\varphi$ is not a homeomorphism and hence is simply a semi-conjugacy. 
    \end{proof}
\end{prop}

We can use this semi-conjugacy gained via symbolic dynamics to now prove results about the doubling map. 

\begin{prop}
    The periodic points of the doubling map $D: [0, 1] \to [0, 1]$ are dense.
    \begin{proof}
        Let $\underline{s} = (s)_{i=1}^{\infty}$ be an arbitrary point in $\Sigma_2$. Define $t_n = (s_0\cdots s_ns_0\cdots s_n\cdots)$ to be an infinite repeating sequence where $t_{n_i} = s_i$ for $1 \leq i \leq n$. Then $d(s, t) = \sum_{i = 0}^n|s_i - s_i|2^{-i} + \sum_{i=n+1}^{\infty}|s_i - t_i|2^{-i} \leq \sum_{i = n+1}^{\infty}2^{-i} = 2^{-n}$. Hence as $n \to \infty$ we have $t_n \to \underline{s}$. Since $\underline{s}$ was arbitrary, the periodic points of $\Sigma_2$ are dense. Since the doubling map is semi-conjugate by Proposition \ref{prop:doubling-shift} the periodic points of the doubling map $D$ are dense.
    \end{proof}
\end{prop}

\begin{prop}
    There exists a dense orbit in the doubling map $D: [0, 1] \to [0, 1]$.
    \begin{proof}
        Consider the sequence $\underline{s} = (0\ 1\ |\ 00\ 01\ 10\ 11\ |\ 000\ 001\ \cdots\ |\ \cdots)$ contructed by writing down all possible combinations of blocks of length one to infinity. Let $\underline{t} = (t)_{i=0}^{\infty} \in \Sigma_2$ be arbitrary. Let $\varepsilon > 0$. By construction we can perform some $k$ number of iterations of $\sigma$ such that if $n > N + k = \frac{1}{\varepsilon} + k$ iterations of $\sigma$ such that $d(\underline{s}, \underline{t}) = \sum_{i = k}^{n}|s_i - t_i|2^{-i} + \sum_{i = n+1}^{\infty}|s_i - t_i|2^{-i} \leq \sum_{i = n+1}^{\infty}|s_i - t_i|2^{-i} = 2^{-n} < 2^{-N} < \frac{1}{N} = \varepsilon$. Hence the orbit $\underline{s}$ is dense in $\Sigma_2$. Since the doubling map is semi-conjugate by Proposition \ref{prop:doubling-shift} the doubling map has a dense orbit.
    \end{proof}
\end{prop}

\section{Sharkovsky's Theorem and Type}\label{sec:sharkovsky}
Sharkovsky's theorem is a hugely important theorem which describes when chaos occurs in topological dynamical systems on the real numbers \cite{sharkovsky}. It states that in these topological dynamical systems, periodic points of all periods can occur, and that the presense of a periodic point of given period implies the existence of other periods given by a total ordering called \emph{Sharkovsky's Order}. First we will introduce a theorem that proves the existence of fixed points on closed intervals of the real numbers. This theorem is a simplified version of \emph{Brouwer's fixed-point theorem} \cite{brouwer}. The full theorem is used to prove the existence of fixed points when a continuous map is applied to compact, convex sets.

\begin{thm} \label{thm:interval-fixed-points}
    If $(I, f)$ is a topological dynamical system where $I$ is a closed interval and $I \subseteq f(I)$, then $f$ has a fixed point in $I$.
    \begin{proof}
        Let $I = [a, b]$ where $a, b \in \mathbb{R}$. Since $I \subseteq f(I)$ there exits $c, d \in I$ such that $f(c) = a$ and $f(d) = b$. Suppose $g(x) = f(x) - x$, then $g$ is continuous. Also $g(c) = a - c \geq 0$ and $g(d) = b - d \leq 0$. By the Intermediate Value Theorem there exists some $x \in I$ with $a \leq x \leq b$ such that $g(x) = 0$. Hence $f(x) = x$.
    \end{proof}
\end{thm}

First we will prove a special case of Sharkovsky's theorem, specifically detailing how the existence of a period three orbit proves the existence of all other periods of orbits. This theorem was discovered by Li and Yorke in their paper \emph{'Period Three Implies Chaos'} \cite{li-yorke}. This was the first paper to use the term \emph{chaos} in a mathematical context. The proof below is adapted from work by Devaney \cite[Chapter 1.10]{devaney}.

\begin{thm}\label{thm:period3chaos}
    Let $(I, f)$ be a topological dynamical system where $I$ is a closed interval. If $f$ has a point of period three, then $f$ has period points of all other periods.
    \begin{proof}
        Before we begin the proof lets make some simple observations:
        \begin{enumerate}[(i)]
            \item If $I, J$ are closed intervals with $I \subseteq J$ and $J \subseteq f(I)$ then by Theorem \ref{thm:interval-fixed-points} $f$ has a fixed point in $I$.
            \item If $A_0, A_1, \cdots$ are closed intervals with $A_{i+1} \subseteq f(A_i)$ for $0 \leq i \leq n - 1$ then there exists a subinterval $J_0 \subseteq A_0$ such that $f(J_0) \subseteq A_1$. Similarly, there exists a subinterval $J_1 \subseteq A_1$ such that $f(J_1) \subseteq A_2$. Hence there exists a subinterval $J_1 \subseteq J_0$ such that $f(J_1) \subseteq A_1$ and $f^2(J_1) = A_2$. Continuing we get a nested sequence of intervals which are mapped in order into each $A_i$. Hence there exists some $x \in A_0$ such that $f^i(x) \in A_i$ for all $i$.
        \end{enumerate}
        To begin the proof let $a, b, c \in \mathbb{R}$ with $f(a) = b$, $f(b) = c$ and $f(c) = a$. Assume wlog that $a < b < c$. Let $I_0 = [a,b]$ and $I_1 = [b,c]$. Hence $I_1 \subseteq f(I_0)$ and $I_0 \cup I_1 \subseteq f(I_1)$. By Theorem \ref{thm:interval-fixed-points} $f$ has a fixed point between $b$ and $c$. Similarly, $f^2$ has at least one fixed point between $a$ and $b$. Hence $f$ has a point of period 2. Now let $n \geq 2$. Define the nested sequence of intervals $A_0, A_1, \cdots, A_{n-2} \subseteq I_1$ inductively. Let $A_0 = I_1$. Since $I_1 \subseteq f(I_1)$, there exists a subinterval $A_1 \subseteq A_0$ and $f(A_1) = A_0 = I_1$. Furthermore there exists a subinterval $A_2 \subseteq A_1$ such that $f(A_2) = A_1$ and therefore $f^2(A_2) = A_0 = I_1$. Continuing we get a subinterval $A_{n-2} \subseteq A_{n-3}$ such that $f(A_{n-2}) = A_{n-3}$. By our second observation $x \in A_{n-2}$ then $f(x), f^2(x), \cdots, f^{n-2}(x) \subseteq A_0$ so $f^{n-2}(A_{n-2}) = A_0 = I_1$. As $I_0 \subseteq f(I_1)$ there exits a subinterval $A_{n-1} \subseteq A_{n-2}$ such that $f^{n-1}(A_{n-1}) = I_0$. Since $I_1 \subseteq f(I_0)$, $I_1 \subseteq f^n(A_{n-1})$ so that $f^n(A_{n-1})$ covers $A_{n-1}$. Hence by our first observation $f^n$ has a fixed point $p$ in $A_{n-1}$. The first $n-2$ iterates of $p$ are in $I_1$, the $n-1$th iterate is in $I_0$ and the $n$th is $p$. If $f^{n-1}(p)$ is in the interior of $I_0$ then $p$ has period $n$. However, if $f^{n-1}(p)$ is in the boundary of $I_0$ then $n = 2$ or $3$.
    \end{proof}

    \begin{center}
        \begin{tikzpicture}
            \node(lt) {$I_0$};
            \node(rt) [right=of lt] {$I_1$};
        
            \draw[->] ([yshift=3pt] lt.east) -- node[above] {} ([yshift=3pt] rt.west);
            \draw[->] ([yshift=-3pt] rt.west) -- node[above] {} ([yshift=-3pt] lt.east);
            \draw[->] (rt) to [out=410,in=310,loop,looseness=4.8] (rt);
        \end{tikzpicture}
    \end{center}

\end{thm}

This consequences of this theorem are remarkable. If you can find a point of period three for a map $f$ then there exists periodic points of all other periods, hence periodic points exist with infinite periods. Further note that the only assumption we have made for the mapping $f$ is that it is continuous and speaks to the generality of this result. This theorem however does not show the bigger picture of whats going on here, for this we need to introduce Sharkovsky's theorem, but first we need explain Sharkovsky's order.

\begin{defn}
    The total ordering on $\mathbb{N}$ defined below is named \emph{Sharkovsky's order}. \[ 3 \rhd 5 \rhd 7 \rhd 9 \rhd \cdots \rhd 2 \cdot 3 \rhd 2 \cdot 5 \rhd \cdots \rhd 2^2 \cdot 3 \rhd 2^2 \cdot 5 \rhd \cdots \rhd \cdots \rhd 2^3 \rhd 2^2 \rhd 2 \rhd 1 \]
\end{defn}
(i.e. first all the odd integers multiplied by $2^k$ for $k \in \mathbb{N}$. This exhausts all the natural numbers apart from powers of two, which are then ordered last in descending order.) This brings us to Sharkovsky's theorem.

\begin{thm}
    Let $(I, f)$ be a topological dynamical system where $I$ is a closed interval. If $f$ has a period point of period $k$ then for all integers $l \rhd k$, $f$ has periodic points of period $l$.
\end{thm}

The proof of this theorem will be ommited since it is outwith the scope of this text, however can be found here \cite{sharkovsky}. From the theorem above and noticing the Sharkovsky ordering it can be seen that if $f$ has a periodic point whose period is not a power of two, then $f$ has infinitely many many periodic points. The converse statement also holds. If $f$ has finitely many periodic points then they all must have periods that are powers of two. This sort of behaviour was present in the logistic family with the period doubling bifurcations in Section \ref{sec:logistic_maps}. Furthermore the logistic maps perodicity follows the Sharkovsky's ordering in reverse. Furthermore when all periods are present and we have reached the end of Sharkovsky's ordering the system becomes chaotic. It can be seen that this theorem is a generalised of Theorem \ref{thm:period3chaos} discovered by Li and Yorke \cite{li-yorke} as period three is the greatest period in Sharkovsky's ordering hence its presense implies the existence of all other periods. It can be useful to catagorise maps defined on a closed interval based on their periods. This leads us to the next definition which is based on work by Ruette, \cite[Section 3.3]{ruette}.

\begin{defn} \label{def:type}
    Let $(I, f)$ be a topological dynamical system where $I$ is a closed interval and let $n \in \mathbb{N} \cup \left\lbrace 2^{\infty} \right\rbrace$. Define the map $f$ to be of \emph{type} $n$ if the periods of periodic points of $f$ are the set $\left\lbrace m \in \mathbb{N} : m \unrhd n \right\rbrace$, where $\left\lbrace m \in \mathbb{N} : m \unrhd 2^\infty \right\rbrace$ means $\left\lbrace 2^k : k > 0 \right\rbrace$.
\end{defn}

\begin{exmp} \label{exmp:piecewise_sharkovsky}
    Let $f: [0, 4] \to [0, 4]$ be a piecewise linear map with $f(0) = 2$, $f(2) = 3$, $f(3) = 1$, $f(1) = 4$ and $f(4) = 0$. This graph of $f$ is shown in Figure \ref{fig:piecewise_linear}. Clearly $f$ is continuous. It can be seen that the point 0 is a period-$5$ point. Moreover $f^3[0, 1] = [1, 4]$, $f^3[1, 2] = [2, 4]$ and $f^3[3, 4] = [0, 3]$. So $f^3$ has no fixed points in those intervals. However $f^3[2, 3] = [0, 4]$ hence by by Theorem \ref{thm:interval-fixed-points} $f^3$ has at least one fixed point in $[2, 3]$. Moreover $f: [2, 3] \to [1, 3]$, $f: [1, 3] \to [2, 5]$ and $f: [1, 4] \to [0, 4]$ are all monotonically decreasing on $[2, 3]$. Therefore $f^3$ is monotonically decreasing on $[2, 3]$ and so the fixed point is unique. Therefore this can only be the fixed point for $f$, and as such no period 3 point exists. Hence by Sharkovsky's theorem the map $f$ contains periodic points with the period of every natural number apart from 3. Therefore using Definition \ref{def:type} it is clear that $f$ is of type $5$.

    \begin{figure}[h]
        \centering
        \includegraphics[width=5cm]{piecewise_0_4}
        \caption{Piecewise linear map $f$ with a period five point.}
        \label{fig:piecewise_linear}
    \end{figure}

\end{exmp}

Sharkovsky subsequently proved that continuous maps defined on intervals of the reals can constructed to be of any type \cite{sharkovsky} \cite{sharkovsky2}. In his proof he constructs orbits for odd periods, even periods, then periods that conside of powers of two separately. The construction to prove that continuous interval maps of any odd type can be created is similar to a more general version of Example \ref{exmp:piecewise_sharkovsky}. This theorem is termed Sharkovsky's Realisation Theorem and is stated below. An elegant proof of this theorem will is also outlined below. This proof was discovered by Alseda, Llibre, and Misiurewicz \cite[Section 2.2]{alm} and outlined by Burns and Hasselblatt \cite[Section 7]{burns-hasselblatt} using a family of continuous maps called the trunctated tent maps.

\begin{thm}
    Every tail of the Sharkovsky order is the set of periods for a topological dynamical system $(I, f)$ where $I$ is a closed interval.
    \begin{proof}
        Let $T_h: [0, 1] \to [0, 1]$ where $T_h(x) = \min(h, 1-2|x-1/2|)$ be the family of truncated tent maps with $h \in [0, 1]$ show in Figure \ref{fig:truncated_tent}. It is clear that $T_0$ has only one periodic point, a fixed point at $x = 0$, however $T_1$ has the period-3 orbit $\left\lbrace \frac{2}{7}, \frac{4}{7}, \frac{6}{7} \right\rbrace$ and hence has a periodic point for every period in the positive integers by Sharkovsky's Theorem. Hence,
        \begin{enumerate}[(i)]
            \item Any orbit $\mathcal{O}_{T_h} \subseteq [0, h)$ is also an orbit for $\mathcal O_{T_1}$ and any orbit $\mathcal{O}_{T_1} \subseteq [0, h]$ is also an orbit for $\mathcal{O}_{T_h}$.
        \end{enumerate}
        Now let $h(m) = \min \left\lbrace \max \mathcal{O}_{T_1} : \mathcal{O}_{T_1} \ \text{is a period-$m$ orbit} \right\rbrace$. From this we can obtain that,
        
        \begin{enumerate}[(i)]\setcounter{enumi}{1}
            \item $T_h$ has the period-$l$ orbit $\mathcal{O}_{T_h} \in [0, h)$ if and only if $h(l) < h$.
            \item The orbit $\mathcal{O}_{T_{h(m)}}$ is of period $m$ for $T_{h(m)}$, and all other orbits $\mathcal{O}_{T_{h(m)}}$ are contained within $[0, h(m))$
        \end{enumerate}
        Using Sharkovsky's Theorem it is clear that if $l \lhd m$ then $T_{h(m)}$ has a period-$l$ orbit that lies in $[0, h(m))$ and hence by (ii) $h(l) < h(m)$. Now by symmetry,
        \begin{enumerate}[(i)]\setcounter{enumi}{3}
            \item $h(l) < h(m)$ if and only if $l \lhd m$
        \end{enumerate}
        By (ii), (iii) and (iv) we can see that for any positive integer $m$ the periodic points of $T_{h(m)}$ is the tail of the Sharkovsky order from $m$ and for $l \lhd m$.

    \end{proof}

    \begin{figure}[h]
        \centering
        \includegraphics[width=5cm]{truncated_tent}
        \caption{The truncated tent map $T_h(x)$.}
        \label{fig:truncated_tent}
    \end{figure}
\end{thm}

\chapter{Defining Chaos}

\section{Devaney Chaos}

The term \emph{chaos} in Mathematics is vauge and has multiple definitions with no agreed definative definition. The most widely accepted definition and the one we will investigate first comes from Devaney \cite{devaney} called \emph{Devaney chaos} and takes a strictly topological approach. Before we introduce this key definition however we first need to set up some preliminary definitions. This first definition helps build an understanding of the dense, complex nature of chaos.

\begin{defn}
    Let $(X, f)$ be a topological dynamical system. The map $f$ is \emph{topologically transitive} if for every pair of non-empty open sets $U, V \subseteq X$ there exists $k > 0$ such that $f^k(U) \cap V \neq \emptyset$.
\end{defn}

Stated another way, in a topologically transitive map, points in an arbitraily small neighbourhood can be mapped to any other arbitrary neighbourhood under repeated a number of applications of the mapping. Hence the topological dynamical system cannot be partitioned into any number of disjoint open sets which are invarient under $f$ - i.e. if $U \in X$ then $f(U) \in U$.

\begin{exmp}
    Let $(S^1, f)$ be the topological dynamical system with mapping $f(\theta) = 2\theta$. Let $\theta_1, \theta_2 \in S^1$. Let $(\theta_1, \theta_2) = U$ define an arc between $\theta_1$ and $\theta_2$. Suppose now $d\left(\theta_1, \theta_2\right) > \frac{2\pi}{2^k}$ for some $k \in \mathbb{N}$. Then $d\left( f^k(\theta_1), f^k(\theta_2)\right) = d\left( 2^k\theta_1, 2^k\theta_2 \right) = 2^k d\left( \theta_1, \theta_2 \right) > 2^k \cdot \frac{2\pi}{2^k} = 2\pi$. Hence $f^k((\theta_1, \theta_2))$ covers $S^1$ so for any $V \subseteq S^1$ we obtain $f^k((\theta_1, \theta_2)) \cap V \neq \emptyset$. Hence $f$ is topologically transitive.
\end{exmp}

This next proposition links the existence of a dense orbit to a map $f$ being topologically transitive. This will become important later as there are slight variations between definitions of chaos as some definitions ask for the existence of a dense orbit and others ask for topological transitivity.

\begin{prop} \label{prop:dense-transitive}
    Let $(X, f)$ be a topological dynamical system. If there exists some $x \in X$ such that $\mathcal{O}(x)$ is dense then $f$ is topologically transitive.
    \begin{proof}
        Let $U, V \subseteq I$ with $U, V \neq \emptyset$. Let $x \in I$ such that $\overline{\mathcal{O}(x)} = I$. Then there exists $k, l \in \mathbb{N}$ such that $f^k(x) \in U$ and $f^l(x) \in V$. Assume $k > l$, then $f^k(x) \in f^{k - l}\left( \mathcal{O}_{f^l}(x) \right)$ and therefore $f^k(x) \in U \cap f^{k-l}(V)$. Hence $f$ is transitive.
    \end{proof}
\end{prop}

Here is an interesting example where the existence of a dense orbit implies topological transitivity.

\begin{exmp} \label{exmp:s1irrational}
    Let $(S^1, T_{\lambda})$ be the topological dynamical system with mapping $T_\lambda(\theta) = \theta + 2\pi \lambda$ where $\lambda \in \mathbb{R}$. Each orbit of $T_\lambda$ is dense in $S^1$ if $\lambda$ is irrational. Let $\theta \in S^1$ and suppose $\lambda$ is irrational. If $T_\lambda^n(\theta) = T_\lambda^m(\theta)$ then $(n - m)\theta \in \mathbb{Z}$ and so $n = m$. Since $S^1$ is compact every infinite subset of points on $S^1$ must have a limit. Hence if $\varepsilon > 0$ then there exists $n, m \in \mathbb{Z}$ such that $\left\lvert T_\lambda^n(\theta) - T_\lambda^m(\theta) \right\rvert < \varepsilon$. Let $k = m - n$, then $\left\lvert T_\lambda^k(\theta) - \theta \right\rvert < \varepsilon$. By definition $T_\lambda$ preserves distance in $S^1$ and since $T_\lambda^k$ maps the arc connecting $\theta$ and $T_\lambda^k(\theta)$ to the arc connecting $T_\lambda^k(\theta)$ and $T_\lambda^k(2 \theta)$ with length also less than $\varepsilon$. Therefore $\left\lbrace \theta, T_\lambda^k(\theta), T_\lambda^k(2 \theta), \cdots \right\rbrace$ partition $S^1$ into arcs of length less than $\varepsilon$. Hence this orbit is dense and so $T_\lambda$ is topologically transtitive by Proposition \ref{prop:dense-transitive}.
\end{exmp}

Generally any topological dynamical system $(X, f)$ the converse statement, that if a mapping $f$ is topologically transitive then there exists a dense orbit in $X$, is not true. In fact $(X, f)$ must be a topological dynamical system with no isolated points, for the converse statement to hold \cite[Section 2]{sergiy-lubomir}.

\begin{prop}
    Let $(X, f)$ be a topological dynamical system. The map $f$ is topologically transitive if and only if there exists a point $x \in X$ such that $\mathcal{O}(x)$ is dense and $X$ has no isolated points.
    \begin{proof}
        \textcolor{red}{(Use Rutte page 12) We proved the reverse direction in Proposition \ref{prop:dense-transitive}. Suppose now there exits some $x \in X$ where $\mathcal{O}(x)$ is dense.}
    \end{proof}
\end{prop}

To show the definitions are not equivalent for topological dynamical systems in general, consider the following example.

\begin{exmp}
    Let $(X, f)$ be the topological dynamical system where $X = \left\lbrace 0 \right\rbrace \cup \left\lbrace 2^{-n} : n \in \mathbb{N} \right\rbrace$ and the map $f$ is defined by $f(0) = 0$, $f(2^{-n}) = 2^{-n-1}$. The set $X$ contains infinitely many isolated points as we can choose $B_d(x, \varepsilon)$ around each $x = 2^{-n} \in X$ with $\varepsilon < \frac{1}{4} \min\left\lbrace 2^{-n} - 2^{-n-1}, 2^{-n + 1} - 2^{-n} \right\rbrace$ such that the open balls are disjoint and $B_d(x, \varepsilon) = \left\lbrace x \right\rbrace$. Now let $U = \left\lbrace \frac{1}{2} \right\rbrace$ and $V = \left\lbrace 1 \right\rbrace$. Then $f^k(U) = f^k(\left\lbrace \frac{1}{2} \right\rbrace) = \left\lbrace 2^{-k-1} \right\rbrace$. Hence $f^k(U) \cap V = \left\lbrace 2^{-k-1} \right\rbrace \cap \left\lbrace 1 \right\rbrace = \emptyset, \ \forall k \in \mathbb{N}$. Hence $f$ is not topologically transitive, however $\mathcal{O}(1) = \left\lbrace 1, 2^{-1}, 2^{-2}, \cdots \right\rbrace$ is dense as $\overline{\left\lbrace2^{-n}: n \in \mathbb{N}\right\rbrace} = X$.
\end{exmp}

In this text we will be using the more popular definition of Devaney chaos asking for $f$ to be topologically transitive. This next definition used in Devaney's definition of chaos emphasises the erratic, uncontrollable and unstable nature of chaos.

\begin{defn}\label{defn:sdic}
    Let $(X, f)$ be a topological dynamical system and $\varepsilon > 0$. A point $x \in X$ is \emph{$\varepsilon$-unstable} if, for every neighbourhood $U$ of $x$, there exits a point $y \in U$ and $k \geq 0$ such that $d\left(f^k(x), f^k(y)\right) \geq \varepsilon$. The map $f$ has $\emph{sensitive dependence on initial conditions}$ if for all points $x \in X$, $x$ is $\varepsilon$-unstable.
\end{defn}

In other words there exist points arbitrary close to $x$ that eventually get mapped at least $\varepsilon$ far apart under multiple applications of the map. Hence this definition states that small pertubations in the mappings input may eventually increase to become wildly different over time. However this definition is not enough on its own to characterise chaos. Here are two examples of topological dynamical systems with sensitive dependence on initial conditions. The first example is non-chaotic in the sense of Devaney, however the latter is chaotic as we shall see later.

\begin{exmp} \label{exmp:2x}
    \textcolor{red}{
    Let $([-1, 1], f)$ be the topolgical dynamical system where $f(x) = 4x^3$. Let $x, y \in [-1, 1], \varepsilon > 0$ and suppose $d(x, y) = \varepsilon$, then $d(f^k(x), f^k(y)) = |4^kx^{3k} - 4^ky^{3k}| = 4^k|x^{3k} - y^{3k}| \leq 4^k|x - y|  4^k \varepsilon$
    Hence we can always choose $k$ large enough so that this holds and so $f(x) = 2x$ has sensitive dependence on initial conditions.
    }
\end{exmp}

\begin{exmp} \label{exmp:rotations}
    Let $f: S^1 \to S^1$, with $f(\theta) = 2\theta$. Let $\theta_1, \theta_2 \in S^1$, $\varepsilon < 2^k \delta$ and suppose $|\theta_1 - \theta_2| = \delta$, then $\left\lvert f^k(\theta_1) - f^k(\theta_2)\right\rvert = \left\lvert 2^k\theta_1 - 2^k\theta_2 \right\rvert = 2^k \left\lvert \theta_1 - \theta_2 \right\rvert = 2^k \delta > \varepsilon$. Hence we can always choose a $k$ large enough so this holds and so $f$ has sensitive dependence on initial conditions.
\end{exmp}

This next example fufills our first definition of being topologically transitive, however is not sensitive to initial conditions.
\begin{exmp}
    The rotations $T: S^1 \to S^1$, $T_\lambda(\theta) = \theta + 2\pi \lambda$ where $\lambda$ is irrational are topologically transitive by Example \ref{exmp:s1irrational}. However, since the mappings are distance preserving, they are not sensitive to initial conditions.
\end{exmp}

The topological dynamical system in Example \ref{exmp:2x} is clearly a system we would never consider to be chaotic as its dynamics are easily explained. Hence our definition of chaos must include more than just a sensitive dependence on initial conditions and topological transitivity. Now using these definitions we have carefully established we can finally give an exact definition of Devaney chaos in a topological dynamical system.

\begin{defn}
    A topological dynamical system $(X, f)$ is \emph{chaotic in the sense of Devaney} if all of the following hold:
    \begin{itemize}
        \item[(i)]$f$ has sensitive dependence on initial conditions.
        \item[(ii)]$f$ is topologically transitive.
        \item[(iii)]periodic points of $f$ are dense in $I$.
    \end{itemize}
\end{defn}

The main feature of Devaney chaos is topological transitivity. After Devaney released this definition Silverman \cite{silverman} and Vellekoop and Berglund \cite{vellekoop-berglund} later proved that in a topological dynamical system $(I, f)$ where $I$ is a closed interval, topological transitivity implies that the other two conditions in Devaney chaos hold. Before we formally introduce this result however we need the following lemma.

\begin{lem} \label{lem:noperiodic}
    Let $f: I \to I$ be a continuous map and $I$ an interval. Suppose $J \subseteq I$ is an interval which contains no periodic points of $f$. If $z, f^m(z), f^n(z) \in J$ where $m, n \in \mathbb{N}, \ m < n$ then either $z < f^m(z) < f^n(z)$ or $z > f^m(z) > f^n(z)$.
    \begin{proof}
        Suppose there exists a $z \in J$ such that $z < f^m(z)$ and $f^m(z) > f^n(z)$. Define $g(x) = f^m(x)$, so $z < g(z)$. If $g^{k+1}(x) < g(z)$ for some $k \in \mathbb{N}, \ n \geq 1$ then $g^k(z) - z$ has a positive value in $z$ and a negative value in $g(z)$ and by the Intermediate Value Theorem contain a point $c \in (z, g(z)) \subseteq J$ with $g^k(c) - c = 0$ and hence a $km$-periodic point. Therefore $z < g^k(z)$ for all positive integers $k$. Now let $k = n - m > 0$. Then $z < f^{(n - m)m}(z)$. Assuming $f^{(n-m)}(f^n(z)) < f^m(z)$ then taking $g = f^{n-m}(x)$ similarly yields $f^{(n-m)m}(f^m(z)) < f^m(z)$. However this results in the function $f^{(n-m)m}(x) - x$ having a positive and negative value in $f^m(z)$.Hence, by the Intermediate Value Theorem a $(n-m)m$-periodic point exits in $J$. A contradiction. The other case for $z > f^m(z) > f^n(z)$ can be proved similarly.
    \end{proof}
\end{lem}

Before we prove that topological transitivity implies that the other two conditions for topological dynamical systems on closed intervals on the reals to exhibit Devaney chaos hold, we need one final result from Banks et.\ al \cite{bbcds}. This an important result, however a proof will not be included in this text for the sake of brevity.

\begin{lem} \label{lem:implies-sdic}
    Let $(X, f)$ be a topological dynamical system. If the map $f$ is topologically transitive and has dense periodic points then $f$ has sensitive dependence on initial conditions.
\end{lem}

Finally Vellekoop and Berglund proved that for topolgical dynamical system defineds over closed intervals on the real line topological transitivity implies the other two conditions needed for Devaney chaos.

\begin{lem}
    Let $(I, f)$ be a topological dynamical system where $I$ is a closed interval. If the map $f$ is topologically transitive then $(X, f)$ is chaotic in the sense of Devaney.
\end{lem}

This brings us to the main proposition. We will only prove the reverse direction for brevity as the forward direction is outwith the scope of this project.

\begin{prop}\label{prop:chaotic-transitive}
    A topological dynamical system $(I, f)$ where $I$ is a closed interval, is chaotic in the sense of Devaney if and only if $f$ is topologically transitive.
    \begin{proof}
        For the reverse direction suppose $f$ is topologically transitive. We can use Lemma \ref{lem:implies-sdic} to prove that if the periodic points are dense in $I$ then $f$ also has sensitive dependence on initial conditions and hence is chaotic in the sense of Devaney. Suppose that the periodic points are not dense in $I$, so there exists an interval $J \subseteq I$ where $J$ contains no periodic points. Let $x \in J$ where $x$ is not an endpoint and let $N \subsetneq J$ be a neighbourhood of $x$. Also let $E = J \ N$. Since $f$ is topologically transitive on $I$ there exists a positive integer $m$ with $f^m(N) \cap E \neq \emptyset$. Hence there exists a $y \in J$ such that $f^m(y) \in E \subsetneq J$ and since $J$ contains no periodic points $y \neq f^m(y)$. Moreover, since $f$ is continuous there exists an open neighbourhood $U$ of $y$ such that $f^m(U) \cap U \neq \emptyset$. Using topological transitivity again we can find a $n > m$ and a $z \in U$ with $f^n(z) \in U$. However then $0 < m < n$ with $z \in f^n(U)$ and $z \notin f^m(U) \implies z \leq f^n(z) \leq f^m(z)$. This is a contradiction by Lemma \ref{lem:noperiodic}. Hence the periodic points of $f$ are dense and by Lemma \ref{lem:implies-sdic} $f$ is sensitive to initial conditions. Therefore $f$ is chaotic in the sense of Devaney.
    \end{proof}
\end{prop}

As a result of this proposition, topological dynamical systems which have a topologically transitive map over a closed interval are chaotic in the sense of Devaney. Looking back at the definition of Devaney chaos it is clear that all of the conditions for chaos are topological and so are preserved under topological conjugate maps. This makes looking for Devaney chaotic systems much easier by the following proposition.

\begin{prop}
    Let $(X, f), (Y, g)$ be topologically conjugate, topological dynamical systems. If $f$ is chaotic in the sense of Devaney then $g$ is chaotic in the sense of Devaney.
\end{prop}

Lets introduce some examples of systems that exhibit Devaney chaos using topological conjugacy and properties of some examples of dynamical systems we have already observed.

\begin{exmp}
    \textcolor{red}{The logistic map $f: [0, 1] \to [0, 1]$ defined by $f: 4x (1-x)$ was shown to be topologically conjugate to the tent map $T: [0, 1] \to [0, 1]$ through $\varphi(x) = \sin^2(\frac{\pi x}{2})$ in Proposition \ref{prop:tent-logistic}. By Proposition \ref{prop:logisitc-periodic-dense} the periodic points of the logitic map are dense in $[0,1]$. }
\end{exmp}

\begin{exmp}
    Let $F_4: [0, 1] \to [0,1]$ be the logistic map, where $F_4(x) = 4x(1-x)$.
\end{exmp}

\section{Li-Yorke Chaos}

Now onto our second definition of Chaos. As mentioned in the Section \ref{sec:sharkovsky}, the paper \emph{'Period Three Implies Chaos'} by Li and Yorke \cite{li-yorke} first introduced the term \emph{chaos} in a mathematical context. However in this paper the did not formally define what chaos is. It turns out that the dynamical systems they termed \emph{chaotic} had two properties they were looking at, namely \emph{sensitivity to initial conditions} and \emph{infinitely many periodic orbits of different periods}. This leads us to a basic definition of \emph{Li-Yorke Chaos} which we will build upon.

\begin{defn}
    A topological dynamical system $(I, f)$ where $I$ is a closed interval is \emph{chaotic in the sense of Li-Yorke} if $f$ has a period \emph{three} orbit.
\end{defn}

We know from Sharkovsky's Theorem that if a function has a period-3 orbit then it has a periodic orbit of every period in the positive integers. It turns out that topological dynamical systems that exhibit this feature also follow other behaviors which we will define below and can lead us to a more robust definition of \emph{Li-Yorke Chaos}. First we need to set up some preliminary definitions. These are based on work by Ruette \cite[Section 5.1]{ruette}.

\begin{defn}
    Let $(X, f)$ be a topological dynamical system with $x, y \in X$ and $\delta > 0$. The pair $(x, y)$ is a \emph{Li-Yorke pair} if $\lim_{n \to \infty}\sup \left\lvert f^n(x) - f^n(y) \right\rvert \geq \delta$ and $\lim_{n\to\infty}\inf \left\lvert f^n(x) - f^n(y) \right\rvert = 0$.
\end{defn}
Hence if $x, y$ is a Li-Yorke pair then $x$ and $y$ can be mapped at least $\delta$ far apart under multiple iterations of the map. We have defined this behaviour before in Definition \ref{defn:sdic}, and it can be said that $x$ and $y$ have \emph{sensitive dependence on initial conditions}. Lets now define this behaviour generally over a whole set.

\begin{defn}
    A set $S \subseteq X$ is \emph{scrambled} if for all distinct $x, y \in S$, $(x, y)$ is a Li-Yorke pair.
\end{defn}

Now we have a notion of sensitive dependence on initial conditions that can be applied to a whole set. This leads us to a more concrete definition of Li-Yorke chaos which applies to all topological dynamical systems, not just those defined over closed intervals.

\begin{defn}
    A topological dynamical system $(X, f)$ is \emph{chaotic in the sense of Li-Yorke} if there exists and uncountable scrambled set $S \subseteq X$.
\end{defn}

From the definitions above it is clear that Li-Yorke chaos only relies on sensitive dependence on initial conditions and for infinitely many periodic orbits of different periods. Hence Li-Yorke chaos is more general than Devaney chaos, which further requires topological transitivity and for the periodic points of the map $f$ to be dense in $X$.

\section{Lyapunov Chaos}

\chapter{Consequences of Chaos}

\section{Topological Entropy}

\section{Fractals}

\section{Chaotic Attractors}


\bibliography{chaos}

\end{document}