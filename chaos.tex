\documentclass[11pt,a4paper,oneside]{memoir}

% Packages
\usepackage[a4paper,margin=2.5cm, top=1.5in, bottom=1.5in]{geometry}
\usepackage{graphicx}
\usepackage{enumerate}
\usepackage{amsmath}
\usepackage{amssymb}
\usepackage{amsfonts}
\usepackage{amsthm}
\usepackage{tikz}
\usepackage{float}
\usepackage[T1]{fontenc}
\usepackage[
    colorlinks=true,
    %linkcolor=blue, urlcolor=blue, citecolor=blue % pdf
    linkcolor=black, urlcolor=black, citecolor=black % print
]{hyperref}

\usetikzlibrary{positioning}
\usetikzlibrary{arrows.meta}

% Look for images under the .images/ dir
\graphicspath{ {./images/} }
% Specifiy stype for bibliography
\bibliographystyle{abbrv}

% Theorem and Proposition styling
\theoremstyle{plain}
% Reset theorem numbering for each chapter
\newtheorem{thm}{Theorem}[chapter]
% Reset definition numbering for each chapter
\newtheorem{prop}[thm]{Proposition}
% Reset definition numbering for each chapter
\newtheorem{lem}[thm]{Lemma}

% Definition and Example styling
\theoremstyle{definition}
% Definition numbers are dependent on theorem numbers
\newtheorem{defn}[thm]{Definition}
% Example numbers are dependent on theorem numbers
\newtheorem{exmp}[thm]{Example}

\newcommand{\mmod}[1]{\ (\mathrm{mod}\ #1)}

\begin{document}

% Set correct spacing between lines
\OnehalfSpacing

% -------------- TITLE PAGE --------------
\begin{titlingpage}
    \centering

        \vspace{1cm}
 
        \Huge
        \parbox{10cm}{\begin{center}{\scshape \textbf{Chaos in Topological Dynamical Systems}}\end{center}}


        \includegraphics[width=6cm]{uni_logo}

        
        \LARGE
        {\scshape \textbf{Fraser Robert Love}}
             
        \vspace{0.75cm}
             
        \large

        School of Mathematics and Statistics\\
        University of St Andrews\\

        \vspace{3cm}

        A dissertation submitted for the degree of\\
        \textit{BSc (Hons) Mathematics}\\

        \vfill

        {\large \today\par}
 \end{titlingpage}
% ------------ TITLE PAGE END ------------

\vspace*{2.5cm}

\noindent\textit{I certify that this project report has been written by me, is a record of work carried out by me, and is essentially different from work undertaken for any other purpose or assessment.}

\begin{center}
\includegraphics[width=1.3cm]{signature}
\end{center}

\vspace{1cm}

\begin{abstract}
    \noindent A topological dynamical system is comprised of a continuous mapping acting on a compact metric space. This project studies the complex, chaotic behaviour that can arise in these systems. Using the extra condition of compactness present in these systems, proves beneficial in analysis of chaos and the behaviour of these systems as the underlying mapping is iterated \emph{ad infinitum}. Various definitions of chaos will be examined, namely, Devaney chaos, Li-Yorke chaos and topological chaos. These definitions encompass aspects of indecomposability, repetitiveness and unpredictability; which when combined give a natural interpretation of chaos. This project shall study how these definitions specifically apply to topological dynamical systems on the interval, the unit circle, in sequence space and on compact countable sets. Numerous important topological properties of chaos will be introduced; such as topological transitivity, sensitive dependence, dense periodic points, scrambled sets, Li-Yorke pairs and positive topological entropy. Foundational tools from symbolic dynamics will be combined with topological conjugacy, to transfer these topological properties between systems. Finally, this text shall conclude by characterising various chaotic systems and comparing the definitions of chaos.
\end{abstract}

\vspace{2cm}

\renewcommand{\abstractname}{Acknowledgements}
\begin{abstract}
    \noindent I would like to express my gratitude to Prof.\ Mike Todd for his invaluable guidance and support throughout my time researching this honours project.
    
\end{abstract}

\newpage
\tableofcontents

\chapter{An Introduction to Topological Dynamics} \label{chap:introduction}
This aim of this text is to introduce the reader to topological dynamical systems and explore the various interpretations of chaos through the lense of topology and topological dynamical systems. A discrete dynamical system is defined by a metric space with a corresponding continuous function, mapping the metric space to itself. The function itself is termed a map or mapping whereby points in the underlying metric space are mapped to other points in the set by the application of this function. Topological dynamical systems themselves are a subset of discrete dynamical systems, with the extra requirement that the underlying metric space be compact (i.e.\ complete and totally bounded). This extra condition for compactness is useful for investigating the limiting behavior of the set of iterates of the map as it is repeatedly iterated to infinity; a relevant feature in the study of chaos. The term chaos itself, specifically deterministic chaos, has various definitions in mathematics and was first coined by Li and Yorke in their ubiquitious paper `Period Three Imples Chaos' \cite{li-yorke}. Since then, a number of authors have proposed their own definitions of chaos, hinging on the existence of various properties of the topological dynamical system. The properties we shall be studying in this text include: topological transitivity, existence of a dense orbit, the density of periodic points, existence of an uncountable scrambled set, sensitive dependence on initial conditions and topological entropy. Due to the differences between definitions of chaos, topological dynamical systems can be chaotic according to one interpretation but not another. We shall aim to compare these definitions and understand their consequences, providing examples of topological dynamical systems that exhibit each type of chaos. Specifically we shall be restrict our attention to four compact metric spaces: closed intervals, the unit circle, sequence space and compact countable sets; as these spaces have been heavily studied in literature. This text assumes the reader to be a capable student of pure mathematics with a basic understanding of topology and analysis. The focus of this text is mainly topological; for the sake of brevity content from related areas of ergodic theory, group theory, measure theory, etc.\ are excluded.

This chapter will briefly review some relevant results from topology before introducing ideas central to topological dynamics and the study of chaos in topological dynamical systems. We shall also introduce some popular topological dynamical systems that exhibit chaos, which we will be examining throughout this text. Subsequently, in Chapter \ref{chap:conjugacy-symbol-dynamics} we will introduce notions of comparing and equivalating topological dynamical systems using the framework of topological conjugacy and symbolic dynamics. Here we will use these powerful results to prove topological results for the well known tent map and logistic map using a topological conjugacy. Furthermore, we shall begin our study of chaos via Sharkovsky's theorem. Finally in Chapter \ref{chap:defining-chaos} we will study the three various interpretations of chaos from called Devaney chaos, Li and Yorke chaos, and topological chaos, analysing examples of topological dynamical systems which statisfy each definition. We shall look at the various properties topological dynamical systems require to be considered chaotic and compare the various definitions.

\section{Topological Dynamical Systems and Discrete Dynamics} \label{sec:topological-dynamical-systems}
We shall start this chapter by giving the definition of a topological dynamical system, including some important defintions from topology and introducing some preliminary definitions from discrete dynamics. We shall refer back to these definitions constantly for the remainder of this text. Furthermore, prominent examples of topological dynamical systems will be presented. The reader should remember these examples as they will be integral to understanding several propositions in later chapters. Note that the definitions and results in this section apply generally to continuous maps, however have been formulated in terms of topological dynamical systems for clarity and precision. Beginning, we shall introduce some preliminary definitions from topology.

\begin{defn}[Dense Set] \label{defn:dense}
    Let $(Z, d)$ be a metric space and $X, Y \subseteq Z$ where $X \subseteq Y$. The set $X$ is \emph{dense} in $Y$ if $\overline{X} = Y$, i.e.\ if for every $x \in Y$ there exists an open neighbourhood $U$ of $x$ with $y \in U$ such that $y \in X$.
\end{defn}

\begin{defn}[Finite Cover, Open Cover] \label{defn:cover}
    Let $(X, d)$ be a metric space. A \emph{finite cover} is a collection of sets $\mathcal{C} = \left\lbrace C_1, C_2, \dots, C_n \right\rbrace$ such that $X \subseteq \bigcup_{i = 1}^nC_i$. If the sets $C_1, C_2, \dots C_n$ are all open then $\mathcal{C}$ is a \emph{open cover}.
\end{defn}

\begin{defn}[Compact  Space] \label{defn:compact}
    A metric space $(X, d)$ is \emph{compact} if every open cover $\mathcal{U}$ of $X$ has a finite subcover $\left\lbrace U_{i(1)}, U_{i(2)}, \dots, U_{i(n)} \right\rbrace \subseteq \mathcal{U}$, i.e.\ if $\left\lbrace U_i \right\rbrace_{i\in I}$ is a collection of open subsets of $X$, where $X \subseteq \bigcup_{i \in I}U_i$ then there exists a finite subcollection $\left\lbrace U_{i(1)}, U_{i(2)}, \dots, U_{i(n)} \right\rbrace \subseteq \mathcal{U}$ such that $X \subseteq \bigcup_{j = 1}^{n}U_{i(j)}$. Furthermore, a metric space is compact if and only if it is complete and totally bounded.
\end{defn}

Note that on $\mathbb{R}$ with the standard metric, all closed intervals are compact. Since we now have a notion of what it means for a metric space to be compact, we can now define the main object we shall be studying in this paper, the topological dynamical system.

\begin{defn}[Topological Dynamical System] \label{defn:topological-dynamical-system}
    Let $X$ be a non-empty compact metric space. A \emph{topological dynamical system} denoted $(X, f)$ is given by a continuous map $f: X \to X$. The system starts at an initial point $x \in X$ and evolves through successive iterations of the map $f$. After $k \in \mathbb{N}$ iterations of $f$, the system can be described by $f^n := f \circ f \circ \dots \circ f$, where $x$ is mapped to the point $f^n(x)$. By convention we take $f^0$ to be the identity map.
\end{defn}

Having defined a topological dynamical system, we can now characterise the discrete dynamics of the underlying map through the following definitions.

\begin{defn}[Orbit] \label{defn:orbit}
    Let $(X, f)$ be a topological dynamical system. The \emph{orbit} or \emph{forward orbit} of a point $x \in X$ under $f$ is the set $\mathcal{O}_f(x) = \mathcal{O}^+_f(x) = \lbrace f^n(x) : n \geq 0 \rbrace = \lbrace x, f(x), f^2(x), \dots \rbrace$ of iterates of $x$ under the map $f$. If $f$ is a homeomorphism (i.e $f^{-1}$ exists and is continuous) then the \emph{backward orbit} of $x$ under $f$ is similarly defined as $\mathcal{O}^-_f(x) = \lbrace f^n(x) : n \leq 0 \rbrace = \lbrace x, f^{-1}(x), f^{-2}(x), \dots \rbrace$.
\end{defn}

Note that in this text, unless otherwise stated, the term, orbit, will simply refer to the forward orbit, as we will mostly be dealing with forward dynamics. An interesting characteristic which can occur in topological dynamical systems is the existence of a dense orbit. We shall explore various consequences that systems with this property hold in Chapter \ref{chap:defining-chaos}. For now, note that from Definition \ref{defn:dense} an orbit $\mathcal{O}_f(x)$ of $f$ is said to be dense in $X$ if for every $x \in X$ and $\varepsilon > 0$ there exists a $n \in \mathbb{N}$ such that $d(x_n, x) \leq \varepsilon$.

\begin{defn}[Periodic Point, Cycle] \label{defn:periodic-point}
    Let $(X, f)$ be a topological dynamical system. A point $x \in X$ is \emph{fixed} if $f(x) = x$ and \emph{periodic} if $f^n(x) = x$ for some $n \in \mathbb{N}$. The \emph{period} of a point $x$ is the least positive integer $k$ such that $f^k(x) = x$. If $x$ has a period of $k$ we say that $x$ is a \emph{period-$k$} point. The set of all period-$k$ points of $f$ is denoted by $\text{Per}_k(f)$ and the set of all periodic points of $f$ is denoted $\text{Per}(f)$. Moreover, $f^n(x) = x \iff n = lk$, for some $l \in \mathbb{N}$. The orbit $\mathcal{O}_f(x) = \lbrace x, f(x), \dots, f^{k-1}(x) \rbrace$ of a periodic point is a finite set of unique points, called a \emph{perodic orbit} of period $k$ or simply a \emph{k-cycle}.
\end{defn}

In most topological dynamical systems only a small subset of points are periodic. Most often a larger set of points either enter a periodic orbit after a certain number of iterations of $f$ or converge asymptotically to a periodic orbit, leading us directly into the following defintions.

\begin{defn}[Eventually Periodic, Asymptotically Periodic] \label{defn:eventually-asymptotically-periodic}
    Let $(X, f)$ be a topological dynamical system. A point $x \in X$ is \emph{eventually periodic} of period $k$ if the point $x$ is not periodic and there exists a $n > 0$ such that $f^{k+i}(x) = f^i(x)$, for $i \geq n$. The point $x \in X$ is \emph{asymptotically periodic} to a periodic point $p \in X$ if $\lim_{n \to \infty} d(f^n(x), f^n(p)) = 0$.
\end{defn}

\begin{prop} \label{prop:eventually-periodic-implies-periodic}
    Let $(X, f)$ be a topological dynamical system. If $f$ is an invertible map, then every eventually periodic point is periodic.
    \begin{proof}
        Suppose $x \in X$ is eventually periodic of period $k$ in $f$. Then $f^{k + i}(x) = f^i(x)$ for some $n > 0$. By applying $i$ iterations of $f^{-1}$ we obtain, $f^{-i} \circ f^{k + i}(x) = f^{-i} \circ f^{i}(x) \implies f^k(x) = x$. Hence $x$ is periodic with period $k$.
    \end{proof}
\end{prop}

When studying chaos in topological dynamical systems it can be useful to understand how the system behaves for an increasing number of iterations. The $\omega$-limit set, defined below as the set of limit points of a particular orbit, allows us to understand the systems behaviour asymptotically.

\begin{defn}[Omega-limit Set] \label{defn:omega-limit-set}
    Let $(X, f)$ be a topological dynamical system. The $\omega$\emph{-limit} set of $x \in X$, denoted $\omega(x, f)$ is the set of all limit points of the orbit $\mathcal{O}_f(x)$ given by \[\omega(x, f) := \bigcap_{n=0}^\infty\overline{\left\lbrace f^k(x) : k \geq n \right\rbrace}\] and the $\omega$\emph{-limit} set of the entire map $f$ is defined as \[\omega(f) := \bigcup_{x \in X} \omega(x, f)\]
\end{defn}

By the definition, we can immediately see that the $\omega$-limit set of a period-$k$ point or eventually periodic point of period-$k$ is simply the $k$-cycle. Furthermore, if a point is asymptotically periodic then the $\omega$-limit set is clearly a cycle and hence finite. Now we are done with defining properties of discrete dynamics lets analyse the discrete dynamics of some popular topological dynamical systems.

\begin{exmp}[Logistic Map] \label{exmp:logitic-map}
    Define $F_{\mu}: [0, 1] \to [0, 1]$ to be the \emph{logistic map}, where $F_{\mu}(x)=\mu x(1-x)$ and $\mu > 0$. Since $[0, 1]$ is a closed interval it is compact. Hence $([0, 1], F_{\mu})$ describes a topological dynamical system.

    \begin{figure}[h]
        \centering
        \includegraphics[width=4cm]{logistic_3}
        \caption{Logistic map $F_\mu$ with $\mu = 3$.}
        \label{fig:logistic_3}
    \end{figure}
\end{exmp}

\begin{exmp}[Tent Map] \label{exmp:tent-map}
    Define $T_s: [0, 1] \to [0,1]$ to be the \emph{tent map}, where $s \in (1, 2]$, $T_s(x) = sx$ for $x \in \left[0, \frac{1}{2}\right]$ and $T_s(x) = s(1-x)$ for $x \in \left[\frac{1}{2}, 1\right]$. Since $[0, 1]$ is a closed interval it is compact. Hence $([0, 1], T_s)$ describes a topological dynamical system. The dynamics of this system becomes difficult to understand as the number of iterations of $T_s$ increases due to the functions piecewise definition. In Chapter \ref{chap:conjugacy-symbol-dynamics} we shall develop the technique of using symbolic dynamics to better analyse the topological dynamics of this system.

    \begin{figure}[h]
        \centering
        \includegraphics[width=4cm]{tent_1.75}
        \caption{Tent map $T_s$ with $s = \frac{7}{4}$.}
        \label{fig:tent_1.75}
    \end{figure}
\end{exmp}

\begin{exmp}[Doubling Map on $\left\lbrack 0, 1 \right\rbrack$] \label{exmp:doubling-map}
    Define $D: [0,1] \to [0,1]$ to be the \emph{doubling map on} $[0, 1]$, where $D(x) = 2x$ for $x \in \left[0, \frac{1}{2}\right]$ and $D(x) = 2x - 1$ for $x \in \left[\frac{1}{2}, 1\right]$. Since $[0, 1]$ is a closed interval it is compact. However the map $D$ is not continuous as a discontunuity exists at $x = \frac{1}{2}$ and so this is not a dynamical system, however we shall use this maps properties to establish results about other topological dynamical systems. Just like the tent map, the dynamics of this system become difficult to understand as the number of iterations of $D$ increases.

    \begin{figure}[h]
        \centering
        \includegraphics[width=4cm]{doubling}
        \caption{Doubling map $D$.}
        \label{fig:doubling}
    \end{figure}
\end{exmp}

Note that throughout this text we shall define $S^1 = \left\lbrace z \in \mathbb{C}: |z| = 1 \right\rbrace = \left\lbrace e^{i\theta} : 0 \leq \theta \leq 2\pi \right\rbrace$ and is the unit circle in the complex plane. The metric space $(S^1, d)$ is defined using the arc length metric, where $d(x, y)$ is the shortest arc connecting $x$ to $y$. 

\begin{exmp}[Doubling Map on $S^1$] \label{exmp:doubling-map-s1}
    Define $\mathcal{D}: S^1 \to S^1$ to be the \emph{doubling map on} $S^1$, where $\mathcal{D}(z) = z^2$, or equivalently $\mathcal{D}(e^{i\theta}) = e^{2i\theta}$ for some $\theta \in \mathbb{R}$. Note that $S^1$ is compact as it is a closed subset of $\mathbb{R}$. Furthermore the map can be easily shown to be continuous. Hence $(S^1, \mathcal{D})$ describes a topological dynamical system.

    \begin{figure}[h]
        \centering
        \includegraphics[width=5cm]{doubling_circle}
        \caption{First ten iterations of the doubling map $\mathcal{D}$.}
        \label{fig:doubling-circle}
    \end{figure}
\end{exmp}

\begin{exmp}[Rigid Rotations] \label{exmp:rigid-rotations}
    Define $R_\alpha: S^1 \to S^1$ to be the \emph{rigid rotations of the unit circle}, where $\alpha \in [0, 2\pi)$ and the function $R_{\alpha}(z) = ze^{i\alpha}$, or equivalently $R_\alpha(e^{i\theta}) = e^{i(\theta + \alpha)}$. Note that $S^1$ is compact as it is a closed subset of $\mathbb{R}$. Hence $(S^1, R_{\alpha})$ describes a topological dynamical system.

    \begin{figure}[h]
        \centering
        \includegraphics[width=5cm]{rigid_circle}
        \caption{First ten iterations of the rigid rotations $R_\alpha$.}
        \label{fig:rigid-circle}
    \end{figure}
\end{exmp}

An interesting property of the rigid rotations is how the behaviour of the dynamical system changes depending on the rationality or irrationality of $\alpha$. This brings us to the following proposition.

\begin{prop} \label{prop:rigid-rotations-irrational}
    If $\alpha$ is an irrational number, then for all $z \in S^1$, the orbit $\mathcal{O}_{R_\alpha}(z)$ is infinite and dense on $S^1$.
    \begin{proof}
        Let $z \in S^1$ be arbitrary. If $R_\alpha^m(z) = R_\alpha^n(z)$ for some $m, n \in \mathbb{Z}$ then $ze^{(m-n)i\alpha} = z \implies (m - n)i\alpha = 0$. Since $\alpha \notin \mathbb{Q}$ and $(m - n)\alpha \neq 2\pi n$ for $n \in \mathbb{N}$ we have $m = n$. Hence all the points in the orbit are distinct and so $\mathcal{O}_{R_\alpha}(z)$ is infinite. Now let $w \in S^1$ be arbitrary and let $\varepsilon > 0$. Choose $N$ such that $\frac{2\pi}{N} < \varepsilon$. Now there exists $0 \leq l, k \leq N$ such that $d\left( R_\alpha^k, R_\alpha^l \right) \leq \frac{2\pi}{N}$. As $R_\alpha$ is an isometry i.e. $d(R_\alpha(x), R_\alpha(y)) = d(x, y)$ we obtain $d(R_\alpha^{(k - l)}(z), z) \leq \varepsilon$. Now since $R_\alpha$ is an isometry, the set $X = \lbrace R_\alpha^{n(k - l)}(z) : n \in \mathbb{N} \rbrace$ partitions $S^1$ into arcs of of length less than $\varepsilon$. Hence there must exist a $R_\alpha^{i(k - l)}(z) \in X$ such that $d(R_\alpha^{i(k - l)}(z), w) \leq \varepsilon$ and so $\mathcal{O}_{R_{\alpha}}$(z) is dense on $S^1$.
    \end{proof}
\end{prop}

Depending on the parameters $s$, $\mu$ and $\alpha$ the dynamical behaviour of these systems can range from predicitble periodicity to chaotic. In Chapter \ref{chap:conjugacy-symbol-dynamics} we shall subsequently prove that $T_2$ and $F_4$ are similar topologically speaking and share various topological properties. Moreover, both of these topological dynamical systems are described by simple mathematical equations, however we shall prove later that they are chaotic and exhibit highly complex and irregular dynamics.

\chapter{Topological and Symbolic Relationships} \label{chap:conjugacy-symbol-dynamics}
\input{chapters/topological_and_symbolic_relationships}

\chapter{Topological Characteristics and Definitions of Chaos} \label{chap:defining-chaos}
The term \emph{chaos} in Mathematics is vaguely defined and has no universally accepted definition. As mentioned in a previous chapter, the paper \emph{'Period Three Implies Chaos'} by Li and Yorke \cite{li-yorke} first introduced the term in a mathematical context, however without giving a precise, formal description of the phenomenon. In subsequent years, various mathematicians have attempted to define their own interpretations of chaos. All these interpretations rely on the topological dynamical system exhibiting various defined topological characteristics. The properties relied upon by the definitions of chaos we will look at include: topological transitivity or the existence of a dense orbit, sensitive dependence on initial conditions, that periodic points are dense in the underlying metric space, the existence of an uncountable scrambled set, and positive topological entropy. Firstly, in Section \ref{sec:topological-transitivity}, we shall investigate the properties of topological transitivity and the existence of a dense orbit. We shall give a definition of a topological transitivity and provide examples of systems which exhibit this phenomenon. Then we shall prove that for systems with no isolated points topological transitivity is an identical property to the existence of an orbit which is dense in the underlying metric space. Section \ref{sec:sdic} introduces sensitive dependence on initial conditions, meanwhile giving examples of systems which exhibit this phenomenon. In Section \ref{sec:devaney-chaos} we shall explore a widely accepted definition of chaos, termed \emph{Devaney chaos} \cite{devaney}. This definition relies upon the topological dynamical system possessing three properties, namely, topological transitivity, sensitive dependence on initial conditions and that periodic points are dense in the underlying metric space. However, we shall show that in general the property of having a dense set of periodic points is redundant. In Section \ref{sec:li-yorke-chaos} we shall explore \emph{Li-Yorke chaos} \cite{li-yorke}. This interpretation of chaos explores the types of topological dynamical systems Li and Yorke termed chaotic through the general definition of scrambled sets. Notably this type of chaos relies on the existence of an uncountable scrambled set. Finally, in Section \ref{sec:topological-chaos} we shall study topological chaos and positive topological entropy. Specifically we will find that positive topological entropy is the only requirement for a system to be topologically chaotic. Finally, Section \ref{sec:compairing-chaos} compares the various definitions of chaos. In this chapter we shall see that all the topological dynamical systems we have studied exhibit chaos according to at least one definition. Before we introduce these various types of chaos however, we first need to set up some preliminary definitions.

\section{Topological Transitivity and the Existence of a Dense Orbit} \label{sec:topological-transitivity}

There are many different types of chaos in topological dynamical systems. All of these different types of chaos require the topologically dynamical system to display different chaotic properties.
This first definition is a prime characteristic of chaotic systems. In fact we shall show later that in topological dynamical systems defined over a closed interval, it is the only characteristic needed to prove that a system is chaotic in the sense of Devaney.

\begin{defn}[Topological Transitivity] \label{defn:topological-transitivity}
    Let $(X, f)$ be a topological dynamical system. The map $f$ is \emph{topologically transitive} if for every pair of non-empty open sets $U, V \subseteq X$ there exists a $k > 0$ such that $f^k(U) \cap V \neq \emptyset$.
\end{defn}

Alternatively stated, in a topologically transitive system, points in an arbitrarily small neighbourhood can be mapped to any other arbitrarily small neighbourhood under repeated iterations of the map. Hence the topological dynamical system cannot be partitioned into two disjoint non-empty open sets which are invariant under the map -- i.e.\ if $U \in X$ then $f(U) \in U$. This next example, from \cite{kolyada-snoha}, provides a creative way of producing topologically transitive topological dynamical systems by using the orbit of any other topological dynamical system.

\begin{exmp}
    Let $(X, f)$ be a topological dynamical system and let $x \in X$ be a periodic point of $f$. Clearly $\mathcal{O}_f(x) = Y$ is finite and so $(Y, f|_y)$ is a topological dynamical system. $(Y, f|_y)$ is transitive as if $U \subseteq Y$ is open, then $f^k(U) = Y$ for some $k > 0$ and so $f^k(U) \cap V \neq 0$ for all $V \subseteq Y$ open.
\end{exmp}

The following proposition, proved by Silverman \cite{silverman}, states that if $(X, f)$ is a topological dynamical system and $X$ has no isolated points, the existence of a dense orbit of $f$ is equivalent to $f$ being topologically transitive. In his paper, Silverman explicitly states that transitivity implies the existence of a dense orbit if $X$ is separable and second-category. However, since our definition of a topological dynamical system takes $X$ to be a compact metric space these properties automatically hold. This proposition is hugely important as some authors give slight variations between definitions of chaos. Some definitions hinge on the existence of a dense orbit and others hinge on topological transitivity. Hence, by using this proposition, we can show that these definitions are equivalent in topological dynamical systems. The proof of this proposition follows Ruette \cite[\S 2.1]{ruette}.

\begin{prop} \label{prop:dense-transitive}
    Let $(X, f)$ be a topological dynamical system and suppose $X$ has no isolated points. The map $f$ is topologically transitive if and only if there exists some $x \in X$ such that $\mathcal{O}(x)$ is dense in $X$.
    \begin{proof}
        Assume that $f$ is transitive and let $U$ be a non-empty open set. By transitivity, for every non-empty open set $V$ there exists a $k > 0$ such that $f^k(U) \cap V \neq 0$. Hence, $\bigcup_{k \geq 0}f^{-n}(U)$ is dense in $X$. As $X$ is compact, there exists a countable basis of non-empty open sets $(U_n)_{n \geq 0}$. For all $l \geq 0$, $\bigcup_{k \geq 0}f^{-k}(U_n)$ is dense by transitivity. Now define $G = \bigcap_{n \geq 0}\bigcup_{k \geq 0}f^{-k}(U_n)$. This is a dense $G_\delta$ set. If $x \in G_\delta$ then $f^k(x)$ enters any set $U_n$ for some $k$. Hence $\mathcal{O}_f(x)$ is dense in $X$. For the reverse direction let $U, V \subseteq X$ be open with $U, V \neq \emptyset$. Let $x \in X$ such that $\overline{\mathcal{O}_f(x)} = X$. Then there exists $k \in \mathbb{N}$ such that $f^k(x) \in U$. Since $X$ has no isolated points $V\, \backslash \left\lbrace f^i(x) : 0 \leq i \leq k \right\rbrace$ is open and non-empty. Hence there exists an $l \in \mathbb{N}$ such that $f^l(x) \in V\, \backslash \left\lbrace f^i(x) : 0 \leq i \leq k \right\rbrace$. Since $l > k$ and $f^l(x) = f^{l - k} \circ f^{k}(x) \in f^{l - k}(U) \cap V$ we get $f^{l - k}(U) \cap V \neq 0$. Hence $f$ is transitive.
    \end{proof}
\end{prop}

For topological dynamical systems $(I, f)$ where $I$ is a closed interval, topological transitivity and the existence of a dense orbit are interchangeable. This is because closed intervals have no isolated points. Note, the condition for the topological dynamical system to have no isolated points is necessary in Proposition \ref{prop:dense-transitive}. The following is a counterexample to show the definitions are not equivalent for topological dynamical systems in general.

\begin{exmp} \label{exmp:dense-orbit-not-equal-transitive}
    Let $(X, f)$ be the topological dynamical system defined over the space $X = \left\lbrace 0 \right\rbrace \cup \left\lbrace 2^{-n} : n \in \mathbb{N} \right\rbrace$ with the standard metric, and the map $f$ where $f(0) = 0$ and $f(2^{-n}) = 2^{-n-1}$. It can be seen that $(X, d)$ is a compact space, where $d$ is the standard metric as $X$ is a closed subset of $\mathbb{R}$, containing its limit point $0$. The set $X$ contains infinitely many isolated points as we can choose $B_d(x, \varepsilon)$ around each $x = 2^{-n} \in X$ with $\varepsilon < \frac{1}{4} \min\left\lbrace 2^{-n} - 2^{-n-1}, 2^{-n + 1} - 2^{-n} \right\rbrace$ such that the open balls are disjoint and $B_d(x, \varepsilon) = \left\lbrace x \right\rbrace$. Now let $U = \left\lbrace \frac{1}{2} \right\rbrace$ and $V = \left\lbrace 1 \right\rbrace$. Then $f^k(U) = f^k(\left\lbrace \frac{1}{2} \right\rbrace) = \left\lbrace 2^{-k-1} \right\rbrace$. Hence $f^k(U) \cap V = \left\lbrace 2^{-k-1} \right\rbrace \cap \left\lbrace 1 \right\rbrace = \emptyset, \ \forall k \in \mathbb{N}$. Hence $f$ is not topologically transitive, however, $\mathcal{O}(1) = \left\lbrace 1, 2^{-1}, 2^{-2}, \dots \right\rbrace$ is dense as $\overline{\left\lbrace2^{-n}: n \in \mathbb{N}\right\rbrace} = X$.
\end{exmp}

Now let's introduce an example of a topological dynamical system with no isolated points. Using Proposition \ref{prop:dense-transitive} we shall see that this has both topological transitivity and the existence of a dense orbit.

\begin{exmp} \label{exmp:dense-orbit-and-transitive}
    In Example \ref{exmp:rigid-rotations} we introduced the rigid rotations, a topological dynamical system $(S^1, R_\alpha)$ where $R_{\alpha}(z) = ze^{i\alpha}$. Furthermore in Proposition \ref{prop:rigid-rotations-irrational} we proved that the irrational rotations gave rise to dense orbits and that these orbits where infinite. Hence $S^1$ does not contain an isolated point. Using Proposition \ref{prop:dense-transitive} we see that $(S^1, R_\alpha)$ is topologically transitive.
\end{exmp}

\begin{exmp} \label{exmp:logistic-tent-doubling-transitive}
    In Proposition \ref{prop:logisitc-tent-doubling-periodic-dense} we proved that the periodic points of the logistic map $([0, 1], F_4)$, the tent map $([0, 1], T_2)$ and the doubling map $([0, 1], D)$ are dense in $[0, 1]$. Clearly all these systems have no isolated points. Hence, using Proposition \ref{prop:dense-transitive} we can clearly see that all these maps are topologically transitive.
\end{exmp}

An important fact to note is that topological dynamical systems which are topologically transitive with an isolated point are in fact trivial, having only one periodic orbit. Hence in the rest of this text we shall restrict our study to topological dynamical systems without an isolated point and shall use the existence of a dense orbit and topological transitivity interchangeably.

\section{Sensitive Dependence on Initial Conditions} \label{sec:sdic}

To continue, let's introduce our second topological characteristic of chaos, also from Devaney \cite{devaney}.

\begin{defn}[Sensitive Dependence On Initial Conditions] \label{defn:sensitive-dependence}
    Let $(X, f)$ be a topological dynamical system and $\varepsilon > 0$. A point $x \in X$ is \emph{$\varepsilon$-unstable} if, for every neighbourhood $U$ of $x$, there exists a point $y \in U$ and $k \geq 0$ such that $d\left(f^k(x), f^k(y)\right) \geq \varepsilon$. The map $f$ has \emph{sensitive dependence on initial conditions} if for all points $x \in X$, $x$ is $\varepsilon$-unstable.
\end{defn}

In other words, there exist points arbitrary close to $x$ that eventually get mapped arbitrarily far apart under multiple iterations. Hence this definition states that small perturbations between iterates may eventually increase through repeated iterations of the map to become wildly different over time; behaviour which hopefully feels notionally chaotic to the reader. Note that the definition states that at least one point contained within each neighbourhood of $x$ gets mapped arbitrarily far apart, not all points. Here is an example of a topological dynamical system with sensitive dependence on initial conditions.

\begin{exmp} \label{exmp:doubling-map-s1-sensitive}
    Let $(S^1, D)$ be the doubling map. Take $\varepsilon = \frac{1}{5}$. Let $w \in S^1$ and let $\delta > 0$. Choose $k$ such that $2^{-(k+2)} \leq \delta$. Pick $z \in S^1$ such that $d(w, z) = 2^{-(k+2)} \leq \delta$. Hence the $(k + 2)$th digit in the binary expansions of $w$ and $z$ differ. Therefore $d\left(D^k(w), D^k(z)\right) =  d\left(2^kw, 2^kz\right) = 2^k d(w, z) = 2^{-k(k+2)} = \frac{1}{4} \geq \frac{1}{5} = \varepsilon$.
\end{exmp}

Note the example above displays a strong type of sensitive dependence on initial conditions termed expansiveness. In a topological dynamical system which exhibits expansiveness, all points arbitrarily close together eventually get mapped arbitrarily far apart; not just a proper subset of points as for the case of sensitive dependence on initial conditions. Another system which exhibits sensitive dependence on initial conditions is the shift map $(\Sigma_2, \sigma)$.

\begin{exmp} \label{exmp:shift-map-sensitive}
    Let $(\Sigma_2, \sigma)$ denote the shift map. Take $\varepsilon = 1$. Let $\underline{s} = (s)_{i=1}^{\infty} \in \Sigma_2$ and let $\delta > 0$. Choose $n$ such that $2^{-n} \leq \delta$. Pick $\underline{t} = (t)_{i=1}^{\infty} \in \Sigma_2$ such that $d(\underline{s}, \underline{t}) \leq 2^{-n} \leq \delta$. Hence $\underline{s}$ and $\underline{t}$ agree on the first $n+1$ symbols. Now there exists a $k > n + 1$ such that $s_k \neq t_k$. The first term of $\sigma^k(\underline{s})$ is $s_k$ and the first term of $\sigma^k(\underline{t})$ is $t_k$. Therefore $d(\sigma^k(\underline{s}), \sigma^k(\underline{t})) = \sum_{i = 0}^{\infty}|s_{i+k} - t_{i+k}|2^{-i} \geq |s_k - t_k|2^{0} = 1 = \varepsilon$.
\end{exmp}

The rigid rotations do not have sensitive dependence on initial conditions, which can be seen below.

\begin{exmp} \label{exmp:rigid-rotations-not-sensitive}
    Let $(S^1, R_\alpha)$ be the rigid rotations. Let $z_1,z_2 \in S^1$, $\varepsilon > 0$ and suppose $d(z_1, z_2) < \varepsilon$, then since $R_\alpha$ is an isometry $d(R_\alpha^k(z_1), R_\alpha^k(z_2)) = d(z_1, z_2) < \varepsilon$. Hence $(S^1, R_\alpha)$ does not have sensitive dependence on initial conditions.
\end{exmp}

We shall now use the definitions of topological transitivity or the existence of dense orbit and sensitive dependence on initial conditions to define Devaney chaos in topological dynamical systems and investigate examples of chaotic systems.

\section{Devaney Chaos} \label{sec:devaney-chaos}

Our first notion of chaos was developed by Devaney \cite{devaney} and is one of the most typically employed definitions of chaos. Devaney's interpretation of chaos includes unpredictability via sensitive dependence on initial conditions, repetitive behaviour through periodic points being dense, and should be indecomposable through topological transitivity.

\begin{defn} [Devaney Chaos] \label{defn:devaney-chaos}
    A topological dynamical system $(X, f)$ is \emph{chaotic in the sense of Devaney} if it is topologically transitive, has sensitive dependence on initial conditions, and if the periodic points of $f$ are dense in $X$.
\end{defn}

The main feature of Devaney chaos is topological transitivity. After Devaney released this definition Banks et al.\ \cite{bbcds} and Glasner et al.\ \cite{glasner-weiss} showed that sensitive dependence on initial conditions is redundant. Note that this result holds even for a general mapping $f: X \to X$.

\begin{prop} \label{prop:transitivity-dense-periodic-implies-sdic}
    Let $(X, f)$ be a topological dynamical system. If the map $f$ is topologically transitive and has dense periodic points then $f$ has sensitive dependence on initial conditions.
    \begin{proof}
        Let $(X, d)$ be a metric space. Observe that we can find a $\delta_0 > 0$ such that for all $x \in X$ there exists a periodic point $q \in X$ such that $dist(\mathcal{O}_f(q), x) \geq \delta_0/2$. Proving this, take $q_1, q_2$ to be arbitrary periodic points where $\mathcal{O}_f(q_1) \cup \mathcal{O}_f(q_2) = \emptyset$. Let $\delta_0 = dist(\mathcal{O}_f(q_1), \mathcal{O}_f(q_2))$ and suppose $q_1' \in \mathcal{O}_f(q_1)$ and  $q_2' \in \mathcal{O}_f(q_2)$ are points such that $d(q_1', q_2') = \delta_0$. For all $x \in X$ either $d(q_1', x) \leq d(q_2', x)$ or by symmetry $d(q_2', x) \leq (q_1', x)$. Using the triangle inequality we find $d(q_1', q_2') \leq d(q_1', x) + d(x, q_2')$ for all $x \in X$. Hence we either have $\delta_0 = d(q_1', q_2') \leq 2d(q_1', x)$ or $\delta_0 = d(q_1', q_2') \leq 2d(q_2', x)$. Therefore we either have $dist(\mathcal{O}_f(q_1), x) \geq \delta_0/2$ or $dist(\mathcal{O}_f(q_2), x) \geq \delta_0/2$. Using this observation we can now prove $f$ has sensitive dependence on initial conditions. First let $\delta = \delta_0/8$ and let $x \in X$ be arbitrary, $x \in N$ where $N$ is an open neighbourhood. Since the periodic points of $f$ are dense in $X$ there exists a period-$n$ point $p \in U = N \cap B(x, \delta)$ open. By the observation above, there exists a periodic point $q \in X$ with $dist(\mathcal{O}_f(p), x) \geq 4\delta$. Now define $V = \bigcap_{i=0}^n f^{-i}(B(f^i(q), \delta))$. Since $V$ is a finite intersection of open sets, it itself is open. Moreover $q \in V$, so $V$ is non-empty. Since $f$ is topologically transitive, there exists a $y \in U$ with natural number $k > 0$ such that $f^k(y) \in V$. Now suppose $j = \left\lfloor \frac{k}{n} \right\rfloor + 1$ such that $1 \leq nj - k \leq n$. Hence, $f^{nj}(y) = f^{nj - k}(f^k(y)) \in f^{nj - k}(V) \subseteq B(f^{nj - k}(q), \delta)$. We also have $f^{nj}(p) = p$, so by the triangle inequality $d(f^{nj}(p), f^{nj}(y)) = d(p, f^{nj}(y)) \geq d(x, f^{nj - k}(q)) - d(f^{nj - k}(q), f^{nj}(y)) - d(p, x)$. Now as $p \in B(x, \delta)$ and $f^{nj}(y) \in B(f^{nj - k}(q), \delta)$ we have $d(f^{nj}(p), f^{nj}(y)) > 4\delta - \delta - \delta = 2\delta$. Hence, by the triangle inequality either $d(f^{nj}(x), f^{nj}(y)) > \delta$ or $d(f^{nj}(x), f^{nj}(p)) > \delta$. By definition, $f$ has sensitive dependence on initial conditions.
    \end{proof}
\end{prop}

For a general topological dynamical system this is the only superfluous property. Silverman \cite{silverman} and Vellekoop and Berglund \cite{vellekoop-berglund} later proved that for a topological dynamical system $(I, f)$ where $I$ is a closed interval, topological transitivity or equivalently the existence of a dense orbit, implies $f$ has dense periodic points in $X$. We shall introduce and later prove this result, but first we require the following lemma from Block et al.\ \cite[\S 4.1]{block-coppel}.

\begin{lem} \label{lem:closed-interval-no-periodic-points}
    Let $f: I \to I$ be a continuous map and $I$ an interval. Suppose $J \subseteq I$ is an interval which contains no periodic points of $f$. If $z, f^m(z), f^n(z) \in J$ where $m, n \in \mathbb{N}, \ m < n$ then either $z < f^m(z) < f^n(z)$ or $z > f^m(z) > f^n(z)$.
    \begin{proof}
        Suppose there exists a $z \in J$ such that $z < f^m(z)$ and $f^m(z) > f^n(z)$. Define $g(x) = f^m(x)$, so $z < g(z)$. If $g^{k+1}(x) < g(z)$ for some $k \in \mathbb{N}, \ n \geq 1$ then $g^k(z) - z$ has a positive value in $z$ and a negative value in $g(z)$ and by the Intermediate Value Theorem contain a point $c \in (z, g(z)) \subseteq J$ with $g^k(c) - c = 0$ and hence a $km$-periodic point. Therefore $z < g^k(z)$ for all positive integers $k$. Now let $k = n - m > 0$. Then $z < f^{(n - m)m}(z)$. Assuming $f^{(n-m)}(f^n(z)) < f^m(z)$ then taking $g = f^{n-m}(x)$ similarly yields $f^{(n-m)m}(f^m(z)) < f^m(z)$. However, this results in the function $f^{(n-m)m}(x) - x$ having a positive and negative value in $f^m(z)$. Hence, by the Intermediate Value Theorem a $(n-m)m$-periodic point exists in $J$, a contradiction. The other case for $z > f^m(z) > f^n(z)$ can be proved similarly.
    \end{proof}
\end{lem}

\begin{prop} \label{prop:transitivity-interval-implies-dense-periodic}
    Let $(I, f)$ be a topological dynamical system. If the map $f$ is topologically transitive then $f$ has a dense set of periodic points.
    \begin{proof}
        We shall aim for a contradiction. Suppose that the periodic points are not dense in $I$, so there exists an interval $J \subseteq I$ where $J$ contains no periodic points. Let $x \in J$ where $x$ is not an endpoint and let $N \subsetneq J$ be a neighbourhood of $x$. Also let $E = J\, \backslash\, N$. Since $f$ is topologically transitive on $I$ there exists a positive integer $m$ with $f^m(N) \cap E \neq \emptyset$. Hence there exists a $y \in J$ such that $f^m(y) \in E \subsetneq J$ and since $J$ contains no periodic points $y \neq f^m(y)$. Moreover, since $f$ is continuous there exists an open neighbourhood $U$ of $y$ such that $f^m(U) \cap U \neq \emptyset$. Using topological transitivity again we can find a $n > m$ and a $z \in U$ with $f^n(z) \in U$. However, then $0 < m < n$ with $z \in f^n(U)$ and $z \notin f^m(U) \implies z \leq f^n(z) \leq f^m(z)$. This is a contradiction by Lemma \ref{lem:closed-interval-no-periodic-points}. Hence the periodic points of $f$ are dense.
    \end{proof}
\end{prop}

Note that this result cannot hold generally. In fact this result only holds in $\mathbb{R}$ because of the ordering used in Lemma \ref{lem:closed-interval-no-periodic-points}. Using Propositions \ref{prop:transitivity-dense-periodic-implies-sdic} and \ref{prop:transitivity-interval-implies-dense-periodic} we can clearly see that if $(I, f)$ is a topologically transitive system defined over a closed interval then it is chaotic in the sense of Devaney, giving us the following important result.

\begin{prop}\label{prop:chaotic-transitive}
    Let $(I, f)$ be a topological dynamical system where $I$ is a closed interval. If $f$ is topologically transitive then $(I, f)$ is chaotic in the sense of Devaney.
    \begin{proof}
        Suppose $f$ is topologically transitive. Propositions \ref{prop:transitivity-dense-periodic-implies-sdic} and \ref{prop:transitivity-interval-implies-dense-periodic} tell us that $f$ has sensitive dependence on initial conditions and the periodic points of $f$ are dense in $I$. Hence $(I, f)$ is chaotic in the sense of Devaney.
    \end{proof}
\end{prop}

As a result of this proposition, topological dynamical systems which have a topologically transitive map over a closed interval are automatically chaotic in the sense of Devaney. Here is a counterexample, proving that this result does not hold generally for topological dynamical systems.

\begin{exmp}
    Let $(S^1, R_\alpha)$ be the topological dynamical systems described by rigid rotations, except we now we shall take $\alpha$ to be solely irrational. In Example \ref{exmp:dense-orbit-and-transitive} we proved these topological dynamical systems to be topologically transitive, as irrational rotations gave rise to infinite, dense orbits, by Proposition \ref{prop:rigid-rotations-irrational}. However, we also proved in Example \ref{exmp:rigid-rotations-not-sensitive} that the system does not have sensitive dependence on initial conditions. Hence we have an example of a topological dynamical system which is topologically transitive but neither has dense periodic points nor sensitive dependence on initial conditions, and so is not Devaney chaotic.
\end{exmp}

The example above gives great insight into the troubles involved in trying to develop an all encompassing definition of chaos. The irrational rotations of $S^1$ have dense orbits and so are topologically transitive, but the map is not particularly interesting as nearby points are constantly mapped near together, never giving way to erratic or uncontrollable behaviour. Looking back at the definition of Devaney chaos it is clear that all of the conditions for chaos are topological and so are preserved under topological conjugate maps. This makes looking for Devaney chaotic systems much easier by the following proposition.

\begin{prop}
    Let $(X, f)$ and $(Y, g)$ be conjugate, topological dynamical systems. If $f$ is chaotic in the sense of Devaney then $g$ is chaotic in the sense of Devaney.
    \begin{proof}
        By Proposition \ref{prop:transitivity-dense-periodic-implies-sdic} sensitive dependence on initial conditions was found to be redundant for topological dynamical systems. In Proposition \ref{prop:conjugacy-preserves-dense-periodic-points} we proved that if $Per(f)$ are dense in $X$ then $Per(g)$ are dense in Y. Hence we just need to prove that topological conjugacy preserves topological transitivity. Let $\varphi: X \to Y$ be a topological conjugacy between $(X, f)$ and $(Y, g)$ and suppose $f$ is topologically transitive. Let $U, V \subseteq Y$ be non-empty open sets. Since $\varphi$ is surjective $\varphi^{-1}(U)$ and $\varphi^{-1}(V)$ are non-empty. As $f$ is topologically transitive, there exists a positive integer $k > 0$ such that $f^k(\varphi^{-1}(U)) \cap \varphi^{-1}(V) \neq \emptyset$. Let $x \in \varphi^{-1}(U)$ such that $f^k(x) \in \varphi^{-1}(V)$. Now set $y = \varphi(x) \in U$ and note that $\varphi \circ f^k(x) = g^k \circ \varphi(x) = g^k(y)$. Therefore $g^k(y) = \varphi \circ f^k(x) \in V$ and so $g^k(U) \cap V \neq \emptyset$.
    \end{proof}
\end{prop}

Now let's introduce some examples of topological dynamical systems that exhibit Devaney chaos. We will prove these examples are chaotic by using topological conjugacy and building on the properties we have already observed in various topological dynamical systems.

\begin{exmp}
    In Example \ref{exmp:logistic-tent-doubling-transitive} we showed that the periodic points of the logistic map $([0, 1], F_4)$, the tent map $([0, 1], T_2)$ and the doubling map $([0, 1], D)$ are all topologically transitive. By Proposition \ref{prop:chaotic-transitive} these systems are all chaotic in the sense of Devaney.
\end{exmp}

\section{Scrambled Sets and Li-Yorke Chaos} \label{sec:li-yorke-chaos}

Now onto our second definition of chaos. As mentioned in Section \ref{sec:sharkovskys-theorem-and-type}, the paper \emph{`Period Three Implies Chaos'} by Li and Yorke \cite{li-yorke} first introduced the term chaos in a mathematical context. In this paper they stated two properties of interval maps that lead to chaotic behaviour, namely sensitive dependence on initial conditions and the existence of an uncountable set with no periodic points. This was formally introduced in the following theorem.

\begin{thm} \label{thm:li-yorke-chaos-intervals}
    If $f: I \to I$ be a continuous interval map with a period three point, then there exists an uncountable set $S \subseteq I$ (containing no periodic points) such that, for all $x, y \in S$ where $x \neq y$
    \[\limsup_{n \to +\infty}\left\lvert f^n(x) - f^n(y) \right\rvert > 0, \ \ \ \ \liminf_{n \to +\infty}\left\lvert f^n(x) - f^n(y) \right\rvert = 0\] and for all periodic points $z \in S$ \begin{equation} \label{equ:no-assympotic-points}\limsup_{n \to +\infty}\left\lvert f^n(x) - f^n(z) \right\rvert > 0.\end{equation}
\end{thm}

Clearly by the requirement of (\ref{equ:no-assympotic-points}) in this theorem, the set $S$ contains no asymptotically stable points. Li and Yorke noted that interval maps which satisfied this equation displayed erratic and irregular behaviour. Hence within this theorem they defined a sense of chaos in interval maps. Note that this theorem does not hold for general metric spaces. For instance, take the rigid rotations $(S^1, R_{2\pi/3})$ as an example. Every point $z \in S^1$ is a period three point as $R^3(z) = ze^{3i \cdot 2\pi/3} = z$, however, there does not exist an uncountable set $S \subseteq S^1$ containing no periodic points, so Theorem \ref{thm:li-yorke-chaos-intervals} does not hold. Since this theorem was published, various authors have generalised the definition of Li-Yorke chaos to the realm of topological dynamical systems. We shall be taking the definition from Blanchard et al.\ \cite{bgsm} which uses the notion of a Li-Yorke pair, defined as follows.

\begin{defn}[Li-Yorke Pair] \label{defn:li-yorke-pair}
    Let $(X, f)$ be a topological dynamical system with $x, y \in X$ and $\delta > 0$. The pair $(x, y)$ is a \emph{Li-Yorke pair} if \[\limsup_{n \to +\infty} d\left( f^n(x), f^n(y) \right) \geq \delta \ \ \ \text{and} \ \ \ \liminf_{n\to+\infty} d\left( f^n(x), f^n(y) \right) = 0.\]
\end{defn}

Hence if $(x, y)$ is a Li-Yorke pair then $x$ and $y$ can be mapped at least $\delta$ far apart under multiple iterations of the map. This is the behaviour we defined, in Definition \ref{defn:sensitive-dependence}, as sensitive dependence on initial conditions. Furthermore if $(x, y)$ is a Li-Yorke pair then $x$ and $y$ can be mapped to the same point under multiple iterations of the map. In this regard we can think of iterations of these two points being scrambled amongst the whole set $X$. Let's now define this behaviour generally over a whole set, to introduce a definition of Li-Yorke chaos

\begin{defn} [Scrambled Set, Li-Yorke Chaos] \label{defn:scrambled-set}
    A set $S \subseteq X$ is \emph{scrambled} if for all distinct $x, y \in S$, $(x, y)$ is a Li-Yorke pair. A topological dynamical system $(X, f)$ is \emph{chaotic in the sense of Li-Yorke} if there exists an uncountable scrambled set $S \subseteq X$.
\end{defn}

Note that in this general version of Li-Yorke chaos the requirement for the set $X$ to have a period three point has been excluded. Furthermore the last requirement, in (\ref{equ:no-assympotic-points}), that no points converge asymptotically to periodic points has been removed. In fact, this extra requirement makes no difference for chaos in the sense of Li-Yorke as if $S$ is a scrambled set then every point except, at most, one point of $S$ satisfy this requirement. Recently it was shown by Lu et al.\ \cite{lu-zhu-wu} that topological conjugacy does not preserve Li-Yorke chaos. Using Theorem \ref{thm:li-yorke-chaos-intervals} we can prove that the following topological dynamical systems are Li-Yorke chaotic.

\begin{exmp}
    Take $([-1, 1], f)$ to be the topological dynamical system where $f(x) = 2 |x| - 1$. Clearly $x = \frac{1}{9}$ is a period three point as $f^3\left(\frac{1}{9}\right) = f^2\left(\frac{-7}{9}\right) = f\left(\frac{5}{9}\right) = \frac{1}{9}$. Hence by Theorem \ref{thm:li-yorke-chaos-intervals} $([-1, 1], f)$ is chaotic in the sense of Li-Yorke.
\end{exmp}

\begin{exmp}
    Let $([0, 1], F_\mu)$ where $1 + 2\sqrt{2} \leq \mu \leq 4$. Note that when $\mu = 1 + 2\sqrt{2}$ a period three point emerges, and hence by Theorem \ref{thm:li-yorke-chaos-intervals} is Li-Yorke chaotic.
\end{exmp}

Note that generally Li-Yorke chaos is not preserved through topological conjugacy \cite{wang}.

\section{Topological Entropy and Topological Chaos} \label{sec:topological-chaos}
Next we shall explore our final definition of chaos: topological chaos. This definition relies heavily on topological entropy, a conjugacy invariant property exhibited by some topological dynamical systems. Topological entropy is a property described by a non-negative real number expressing the complexity of a topological dynamical system by the asymptotic mean growth in the number of distinguishable collections of orbits at an arbitrarily fine yet finite resolution. The quantity was first outlined by Adler et al.\ \cite{adler} and uses the language of open covers. Later Bowen \cite{bowen} and Dinaburg \cite{dinaburg} reformulated this definition in terms of a metric and the separation of orbits. When the underling metric space is compact, i.e.\ in a topological dynamical system, these two definitions become equivalent. First let's introduce the former definition, which in an essence is more natural as it does not depend on the underlying metric space and so is more general in a topological sense.

\begin{defn}[Topological Entropy - Adler et al.]
    Let $(X, f)$ be a topological dynamical system, $\mathcal{C} = \left\lbrace C_1, C_2, \dots C_p \right\rbrace$, $\mathcal{D} = \left\lbrace D_1, D_2, \dots D_q \right\rbrace$ be finite covers and define the cover $\mathcal{C} \vee \mathcal{D} = \left\lbrace C_i \cap D_j : i \in [1, p], \ j \in [1, q] \right\rbrace$. The cover $\mathcal{C}$ is \emph{finer} than $\mathcal{D}$ if every element of $\mathcal{D}$ is also included in $\mathcal{C}$, and is expressed as $\mathcal{C} \prec \mathcal{D}$. Let $N(\mathcal{C})$ be the minimum cardinality of a subcover of $\mathcal{C}$, so $N(\mathcal{C}) = \min \left\lbrace n : \exists i(1), \dots, i(n) \in [1, p],\ X = C_{i(1)} \cup \dots \cup C_{i(n)}\right\rbrace$. Then, for all integers $n \geq 1$ we can define $N_n(\mathcal{C}, f) = N\left(\mathcal{C} \vee f^{-1}(\mathcal{C}) \vee \dots \vee f^{-(n-1)}(\mathcal{C})\right)$. The \emph{topological entropy} of the finite cover $\mathcal{C}$ is given by \[h(\mathcal{C}, f) = \lim_{n \to +\infty}\frac{\log{N_n(\mathcal{C}, f)}}{n} = \inf_{n \geq 1} \frac{\log{N_n(\mathcal{C}, f)}}{n}.\] The \emph{topological entropy} according to Adler et al.\,, denoted $h_A(f)$, of the topological dynamical system $(X, f)$ is given by \[h_{A}(f) = \sup\left\lbrace h(\mathcal{U}, f): \mathcal{U} \ \text{finite open cover of} \ X \right\rbrace.\]
\end{defn}

Now we shall introduce the Bowen-Dinaburg definition of topological entropy using the language of metric spaces. First we need to define the notion of $(n, \varepsilon)$-separated and $(n, \varepsilon)$-spanning sets, given by Bowen \cite{bowen}.

\begin{defn}[$(n, \varepsilon)$-Separated, $(n, \varepsilon)$-Spanning]
    Let $(X, f)$ be a topological dynamical system defined on the metric space $(X, d)$. A set $E \subseteq X$ is \emph{$(n, \varepsilon)$-separated} if for all distinct $x, y \in E$ there exists a $k$ with $0 \leq k < n$ such that $d(f^k(x), f^k(y)) \geq \varepsilon$. For $n \in \mathbb{N}$ where $n \geq 1$, define $d_n(x, y) = \max{\left\lbrace d(f^k(x), f^k(y)) : 0 \leq k < n \right\rbrace}$ and $B_n(x, \varepsilon) = \left\lbrace y \in X : d_n(x, y) < \varepsilon \right\rbrace$. A set $E \subseteq X$ is \emph{$(n, \varepsilon)$-spanning} if $X \subseteq \bigcup_{x \in E}B_n(x, \varepsilon)$. Let $r(n, \varepsilon)$ denote the minimum cardinality of an $(n, \varepsilon)$-spanning set and $s(n, \varepsilon)$ denote the maximum cardinality of an $(n, \varepsilon)$-separated set.
\end{defn}

Note that since compactness guarantees we can find a finite subcover for $X$ there always exists a $(n, \varepsilon)$-spanning set and a $(n, \varepsilon)$-spanning set with a finite cardinalities. In the above definition, $\varepsilon$ can be considered the resolution, i.e.\ the minimum distance at which the two points become distinguishable, with $r(n, \varepsilon)$ describing the minimum number of collections of indistinguishable orbits and $s(n, \varepsilon)$ describing the maximum number of collections of distinguishable orbits. This is because we have defined an $(n, \varepsilon)$-separated set to be a set such that all points in the set get mapped at least $\varepsilon$ away from each other in at least one of the first iterations of the map. Note that the following is an alternative form of the definition of a spanning set. If $F$ is an $(n, \varepsilon)$-spanning set then for every $x \in X$ there is a $y \in F$ for which $d(f^k(x), f^k(y)) \leq \varepsilon$ for all $0 \leq k < n$. The following lemma, also from Bowen \cite{bowen}, is particularly useful to setup Bowen and Dinaburg's definition of topological entropy. The proof of this following lemma is adapted from analysis by Ruette \cite[§4.1]{ruette}.

\begin{lem} \label{lem:finite-maximum-minimum-spanning-separated}
    If $(X, f)$ is a topological dynamical system with $\varepsilon > 0$ and $n \in \mathbb{N}$, then $r(n, \varepsilon) \leq s(n, \varepsilon) \leq r(n, \varepsilon / 2) < \infty$.
    \begin{proof}
        Suppose $E \subseteq X$ is an $(n, \varepsilon)$-separated set of maximum cardinality $s(n, \varepsilon)$. By the maximality of $E$, for every $y \in X\, \backslash \, E$, $E \cup \left\lbrace y \right\rbrace$ is not $(n, \varepsilon)$-separated, or alternatively $y \in \bigcup_{x \in E}B_n(x, \varepsilon)$. Clearly $E \subseteq \bigcup_{x \in E}B_n(x, \varepsilon)$ so $E$ is an $(n, \varepsilon)$-spanning set, and so $r(n, \varepsilon) \leq s(n, \varepsilon)$. Let $F$ be an $(n, \varepsilon / 2)$-spanning set of cardinality $r(n, \varepsilon / 2)$. For every $x \in X$, there exists $y(x) \in F$ such that $x \in B_n(y(x), \varepsilon / 2)$. If $x_1, x_2 \in E$ are distinct then we must have $y(x_1) \neq y(x_2)$ as then we would have $d(f^k(x_1), f^k(x_2)) < \varepsilon$ for $0 \leq k < n$. Hence $s(n, \varepsilon) \leq r(n, \varepsilon / 2)$.
    \end{proof}
\end{lem}

Finally we can now introduce Bowen and Dinaburg's definition of topological entropy.

\begin{defn}[Topological Entropy - Bowen and Dinaburg]
    Let $(X, f)$ be a topological dynamical system defined on the metric space $(X, d)$. Define \[\overline{h}(\varepsilon, f) = \limsup_{n \to \infty}\frac{\log{s(n, \varepsilon)}}{n} = \limsup_{n \to \infty} \frac{\log{r(n, \varepsilon)}}{n}.\] Note that these limits exists and is finite, as proved by Lemma \ref{lem:finite-maximum-minimum-spanning-separated}. The \emph{topological entropy} according to Bowen and Dinaburg, denoted $h_B(f)$, of the topological dynamical $(X, f)$ is given by \[h_B(f) = \sup_{\varepsilon > 0^+}\overline{h}(\varepsilon, f) = \lim_{\varepsilon \to 0^+}\overline{h}(\varepsilon, f).\]
\end{defn}

By taking the logarithm of $s(n, \varepsilon)$ or $r(n, \varepsilon)$ and dividing by $n$ we obtain the mean growth of distinguishable collections of orbits, which are at least $\varepsilon$ far apart in the first $n$ iterations of the map. Therefore, by then taking the limit superior we obtain the mean asymptotic growth in the number of these distinguishable collections of orbits. Letting $\varepsilon$ tend to zero we get the asymptotic mean growth in the number of collections of orbits at an arbitrarily fine resolution; this is the definition of topological entropy. As mentioned above, both definitions of topological entropy are equivalent in topological dynamical systems.

\begin{prop}
    If $(X, f)$ is a topological dynamical system, then $h_A(f) = h_B(f)$.
    \begin{proof}
        Since $X$ is a compact metric space $X$ we can find an open cover $\mathcal{U} = \left\lbrace U_1, U_2, \dots, U_n \right\rbrace$ of $X$ with $\text{diam}(U_i) \leq \varepsilon$ for all $i \in [1, n]$ and Lebesgue number $2\delta$. Hence we obtain $s(n, \varepsilon) \leq N(\mathcal{U}^n) \leq s(n, \delta)$ implying that $h_A(f) = h_B(f)$.
    \end{proof}
\end{prop}

Throughout the rest of this text we shall write $h_{top}(f)$ to denote either $h_A(f)$ or $h_B(f)$, depending on circumstances. The entropy of a topological dynamical system may be zero. If the underlying map is an isometry we get the following result.

\begin{prop} \label{prop:isometry-entropy}
    Let $(X, f)$ be a topological dynamical system. If $f: X \to X$ is an isometry, then $h_{top}(f) = 0$.
    \begin{proof}
        Let $(X, d)$ be the underlying metric space of $(X, f)$. Take $x, y \in X$ and let $\mathcal{U}$ be an open cover of $X$. As $f$ is an isometry $d(x, y) = d(f^n(x), f^n(y))$ where $n \in \mathbb{N}$. Hence $N_n(\mathcal{U}, f)$ is remains constant, no matter the choice of $n$. Therefore $h_{top}(\mathcal{U}, f) = \inf_{n \geq 1} \frac{1}{n}\log{N_n(\mathcal{C}, f)} = 0$.
    \end{proof}
\end{prop}

This result is clearly true as isometric mappings preserve distance, and so there should be a constant number of distinguishable orbits under repeated applications of the map. We shall now prove that topological entropy is a purely topological property, with its value being independent of the choice of metric.

\begin{lem} \label{lem:entropy-independent-of-metric}
    Let $(X, f)$ be a topological dynamical system defined over two equivalent metrics $d$ and $d'$. The topological entropy of $(X, f)$ with respect to $d$ and $d'$ is the same.
    \begin{proof}
        Consider the following map  $I: (X, d) \to (X, d')$ between metric spaces. As $d$ and $d'$ are equivalent, $I$ is a homeomorphism. Since $X$ is compact, $I$ is uniformly continuous. Hence if $\varepsilon > 0$ there exists a $\delta > 0$ such that $d(x, y) < \delta$ which implies that $d'(x, y) < \varepsilon$. Specifically if $d_n(x, y) \leq \delta$ then $d_n(x, y) \leq \varepsilon$ where $n \in \mathcal{N}$. Hence any $(n, \delta)$-spanning set is also a $(n, \varepsilon)$-spanning set. So $s_d(n, \delta) \geq s_{d'}(n, \varepsilon)$ and therefore we obtain  \[h_{top_{d'}} = \lim_{\varepsilon \to 0^+}{\overline{h}_d(\varepsilon, f)} \leq \lim_{\delta \to 0^+}{\overline{h}_{d'}(\delta, f)} = h_{top_d}.\] By repetition of the same argument, except with the map $I': (X, d) \to (X, d')$ we obtain $h_{top_{d'}}(f) \geq h_{top_d}(f)$. Therefore $h_{top_{d'}}(f) = h_{top_d}(f)$.
    \end{proof}
\end{lem}

This lemma proves particularly useful in proving that topological entropy is a conjugacy invariant property of topological dynamical systems.

\begin{prop} \label{prop:conjugacy-preserves-entropy}
    If $(X, f)$ and $(Y, g)$ topological dynamical systems which are topologically conjugate via the conjugacy $\varphi: X \to Y$, then $h_{top}(f) = h_{top}(g)$.
    \begin{proof}
    Let $d$ be a metric on $X$ and let $d'$ be the metric on $Y$ defined by $d'(y_1, y_2) = d(\varphi(y_1), \varphi(y_2))$ where $y_1, y_2 \in Y$. By Lemma \ref{lem:entropy-independent-of-metric} we know that $h_{top}(g)$ is independent on the definition of $d'$. Using the definition of $d_n$ we get that $d_n'(y_1, y_2) = \max{\left\lbrace d'(f^k(x), f^k(y)): 0 \leq k \leq n - 1 \right\rbrace} = \max{\left\lbrace d(\varphi(f^k(x)), \varphi(f^k(y))): 0 \leq k \leq n - 1 \right\rbrace} = \max{\left\lbrace d(f^k(\varphi(x)), f^k(\varphi(y))): 0 \leq k \leq n - 1 \right\rbrace}  = d_n(\varphi(y_1), \varphi(y_2))$. Hence as $\varphi$ is a bijection $(n, \varepsilon)$-separated sets and $(n, \varepsilon)$-spanning sets have the same cardinality for $X$ and $Y$. Therefore it follows that $h_{top}(g) = h_{top}(f)$.
    \end{proof}
\end{prop}

The converse statement is not true generally. To see this, let $(R_\alpha, S^1)$ be the irrational rigid rotations and $(R_\beta, S^1)$ be the rational rigid rotations. As the rigid rotations are an isometry, by Proposition \ref{prop:isometry-entropy} the topological entropy of both systems is zero, however, $(R_\alpha, S^1)$ and $(R_\beta, S^1)$ are not topologically conjugate. In the following example we shall now calculate the topological entropy of the shift map $(\Sigma_2, \sigma)$. Note that this example uses the following proposition, which we shall state without proof as this requires some insight into measure theory of which the reader is not assumed to be familiar with, for more detail see Walters \cite{walters}.

\begin{prop} \label{prop:entropy-generator}
    Let $(X, f)$ be a topological dynamical system and let $C$ be a topological generator. Then $h_{top}(f) = h_{top}(C, f)$.
\end{prop}

Note that proof of the above just requires us to prove $h_{top}(C, f) \geq h_{top}(f)$ as the reverse is true by definition as a topological generator is simply a finite open cover. This next example follows work done by Adler et al. \cite{adler}.

\begin{exmp} \label{exmp:shift-entropy}
    Let $(\Sigma_2, \sigma)$ denote the shift map and let $C = \left\lbrace [0], [1] \right\rbrace$ partition $\Sigma_2$. Then for $n \in \mathbb{N}$, $\bigvee_{j = 0}^{n - 1}\sigma^{-j}(C)$ is a partition of $\Sigma_2$ into $2^n$ sets. Since $C$ is a topological generator, by Proposition \ref{prop:entropy-generator}, $h_{top}(\sigma) = h_{top}(C, \sigma) = \lim_{n \to +\infty}\frac{1}{n} \log N_n(C, \sigma) = \lim_{n \to +\infty}\frac{1}{n} \log 2^n = \log 2$.
\end{exmp}

Before we introduce the following example for finding the topological entropy of the doubling map $(S^1, D)$ we need to establish the following lemma and propositions, which follow working by Butt \cite{butt}.

\begin{lem} \label{lem:metric-less-quarter}
    If $(S^1, D)$ is the doubling map with underlying metric space $(S^1, d)$ where $d$ is the arc length metric, then for $x, y \in S^1$ we have $d(x, y) \leq \frac{1}{4} \implies d(f(x), f(y)) = 2d(x, y)$.
    \begin{proof}
        Clearly $d(x, y) = |x - y|$ when $|x - y| \leq \frac{1}{2}$. Let $x, y$ be such that $d(x, y) \leq \frac{1}{4}$. Hence we have, $d(S^1(x), S^1(y)) = d(2x \mmod 1, 2y \mmod 1) = \min{(|2x - 2y \mmod 1|, 1 - |2x - 2y \mmod 1|)}$. As $|2x - 2y| \leq \frac{1}{2}$ whe have that $2x - 2y \mod 1 = 2x - 2y$. Therefore, $d(S^1(x), S^1(y)) = 2|x-y| = 2d(x, y)$.
    \end{proof}
\end{lem}

\begin{prop} \label{prop:spanning-set}
    The set $S_{n+k} = \left\lbrace \frac{i}{2^{n+k}} : 0 \leq i < 2^{n+k} - 1 \right\rbrace$ is an $(n, \varepsilon)$-spanning set for the doubling map $(S^1, D)$. 
    \begin{proof}
        Let $\varepsilon > 0$ and choose $k \geq 2$ such that $\frac{1}{2^{k+1}} \leq \varepsilon < \frac{1}{2^k}$. Note that for any $x \in S^1$ we have that $x \in \left[\frac{i}{2^{n+k}}, \frac{i + 1}{2^{n+k}}\right)$ where $0 \leq i < 2^{n+k} - 1$. Then choose $y \in S_{n+k}$ to be either endpoints of this dyadic interval so that $d(x, y) \leq \frac{1}{2^{n+k}} < \frac{1}{4}$. By Lemma \ref{lem:metric-less-quarter} we obtain, $d(D(x), D(y)) = 2d(x, y) \leq \frac{2}{2^{n+k}} < \frac{1}{4}$. Applying Lemma \ref{lem:metric-less-quarter} a total of $j$ times, where $0 \leq j < n$ we obtain $d(D^j(x), D^j(y)) = 2^jd(x, y) \leq \frac{2^j}{2^{n - k}} \leq \frac{2^n - 1}{2^{n - k}} < \frac{1}{2^{k + 1}} \leq \varepsilon$. Hence for any $x \in S^1$ we have $d_n(x, y) = \max{\left\lbrace d(D^j(x), D^j(y)) : 0 \leq j < n \right\rbrace} < \varepsilon$ for some $y \in S_{n + k}$, and so $S_{n+k}$ is a $(n, \varepsilon)$-spanning set for $(S^1, D)$.
    \end{proof}
\end{prop}

\begin{prop} \label{prop:separared-set}
    The set $S_{n-1+k} = \left\lbrace \frac{i}{2^{n-1+k}} : 0 \leq i < 2^{n - 1 + k} - 1 \right\rbrace$ is an $(n, \varepsilon)$-separared set for the doubling map $(S^1, D)$. 
    \begin{proof}
        Let $\varepsilon > 0$ and choose $k \geq 2$ such that $\frac{1}{2^{k+1}} \leq \varepsilon < \frac{1}{2^k}$. Let $x, y \in S_{n-1+k}$ be distinct. Note that we want to prove that $d_n(x, y) \geq \varepsilon$, that is, prove that there exists a $j$ where $0 \leq j < n$ such that $d(D^j(x), D^j(y)) \geq \varepsilon$ (this is the definition of being a $(n, \varepsilon)$-separated set). Now suppose there exists a $j$ such that $d(D^j(x), D^j(y)) \geq \frac{1}{4}$, then we are done as $\varepsilon < \frac{1}{4}$ by assumption. Hence suppose $d(D^j(x), D^j(y)) \leq \frac{1}{4}$ for all $0 \leq j < n$. Therefore we can apply Lemma \ref{lem:metric-less-quarter} a total of $n - 1$ times to show $d(D^{n-1}(x), D^{n-1}(y)) = 2^{n-1}d(x, y)$. Now note that for distinct $x, y \in S_{n-1+k}$ we get $d(x, y) \geq \frac{1}{2^{n-1+k}}$, so $2^{n-1}d(x, y) \geq \frac{2^{n-1}}{2^{n-1+k}} = \frac{1}{2^k} \geq \varepsilon$. Hence $S_{n-1+k}$ is a $(n, \varepsilon)$-separated set for $(S^1, D)$.
    \end{proof}
\end{prop}

\begin{exmp} \label{exmp:s1-doubling-entropy}
    Let $(S^1, D)$ be the doubling map. By Proposition \ref{prop:spanning-set} $S_{n+k}$ is $(n, \varepsilon)$-spanning set. Clearly this set has cardinality $2^{n+k}$, and so $r(n, \varepsilon) \leq 2^{n+k}$. Therefore we get that $\overline{h}(\varepsilon, D) = \limsup_{n \to \infty}\frac{\log{r(n, \varepsilon)}}{n} \leq \limsup_{n\to\infty}\frac{(n + k)\log 2}{n} = \log 2$. Using Proposition \ref{prop:separared-set}, $S_{n - 1 + k}$ is a $(n, \varepsilon)$-separared set with cardinality $2^{n - 1 + k}$, so $s(n, \varepsilon) \geq 2^{n - 1 + k}$. Therefore we also get that $\overline{h}(\varepsilon, D) = \limsup_{n \to \infty}\frac{\log s(n, \varepsilon)}{n} \geq \limsup_{n \to \infty}\frac{(n - 1 + k)\log 2}{n} = \log 2$. Hence using the squeeze theorem for limits we get that $h_{top}(D) = \lim_{\varepsilon \to 0^+}\overline{h}(\varepsilon, D) = \log 2$.
\end{exmp}

Generally it can be proved that for maps of the form $f(x) = \alpha x \mmod 1$, where $\alpha \in \mathbb{N}$, that $h(f) = \log \alpha$. Finally, with a detailed background in topological entropy we can now define topological chaos. Before the term chaos was coined by Li and Yorke in `Period Three Implies Chaos', Furstenburg in \cite{furstenberg}, stated that all topological dynamical systems with zero topological entropy are `deterministic'. In a later papers, Glasner and Weiss \cite{glasner-weiss} and separtely Blanchard \cite{blanchard}, defined that topological dynamical systems that have positive topological entropy exhibit topological chaos.

\begin{defn}
    A topological dynamical system $(X, f)$ exhibits topological chaos if it has positive topological entropy.
\end{defn}

Using Proposition \ref{prop:conjugacy-preserves-entropy} we can clearly deduce that topological chaos is preserved under topological conjugacy. Now let's introduce some examples of topological dynamical systems that exhibit topological chaos.

\begin{exmp}
    In Example \ref{exmp:shift-entropy} and Example \ref{exmp:s1-doubling-entropy} we showed that the shift map $(\Sigma_2, \sigma)$ and the doubling map $(S^1, D)$ have topological entropy $h_{top}(\sigma) = h_{top}(D) = \log 2 > 0$. Therefore $(\Sigma_2, \sigma)$ and $(S^1, D)$ exhibit topological chaos.
\end{exmp}

\section{Compairing Definitions of Chaos} \label{sec:compairing-chaos}

Blanchard et al. \cite{bgsm} proved that in a topological dynamical system, topological chaos implies the existence of Li-Yorke chaos. We shall not cover the proof of this proposition here as it contains ideas from ergodic theory and is outwith the bounds of this text.

\begin{prop}
    If a topological system $(X, f)$ is topologically chaotic then it is also Li-Yorke chaotic.
\end{prop}

In \cite{smital} Smital proved the inverse of this statement to be false implying there exists zero entropy topological dynamical systems which are Li-Yorke chaotic. Furthermore Li \cite{li} proved that for topological dynamical systems $(I, f)$ where $I$ is a closed interval positive topological entropy implies the system exhibits Devaney chaos, and vice versa.

\begin{prop}
    Let $(I, f)$ be a topological dynamical system and $I$ be a closed interval. The $(I, f)$ has positive topological entropy if and only if $(I, f)$ is chaotic in the sense of Devaney.
\end{prop}

Hence for topological dynamical systems $(I, f)$ where $I$ is a closed interval we get the following relation between the different types of chaos mentioned in this text. \[(I, f)\ \text{Devaney-chaotic} \iff (I, f)\ \text{topologically-chaotic} \implies (I, f)\ \text{Li-Yorke-chaotic}.\]

\chapter{Conclusion} \label{chap:conclusion}
In Chapter \ref{chap:introduction}, we defined the notion of a topological dynamical system $(X, f)$ as being a non-empty compact metric space $X$ with a corresponding continuous map $f: X \to X$. We then introduced some examples of topological dynamical systems, namely the logistic map $([0, 1], F_\mu)$, the tent map $([0, 1], T_s)$, the rigid rotations $(S^1, R_\alpha)$, the doubling maps defined over the unit interval $([0, 1], D)$ and unit circle the complex plane $(S^1, \mathcal{D})$. We also introduced important concepts from discrete dynamics, such as orbits, periodic points, cycles, eventually and asymptotically periodic points and $\omega$-limit sets.

Subsequently in Chapter \ref{chap:conjugacy-symbol-dynamics}, we introduced symbolic dynamics and topological conjugacy. We first defined topological conjugation, a term used to describe when two maps exhibit the same topological behaviour. We prove that various topological properties such as the density of orbits and periodic points, are conjugate invariant properties. Next we introduced symbolic dynamics, a field which studies how the shift map effects infinite sequences of symbols, called itineraries, that describe the complex dynamics of some topological dynamical systems. Specifically it was the assignment of orbits of a topological dynamical system to a sequence of discrete symbols in sequence space, denoted $\Sigma_2$. We defined the shift map $(\Sigma_2, \sigma)$ on sequence space and proved that it was in fact a topological dynamical system. Then we went on to prove the periodic points of the shift map are dense and that the shift map has a dense orbit. Furthermore, using topological conjugacy we proved that through various topological conjugacy's and semi-conjugacy's, shown in the diagram below (where an uni-directional arrow represents a semi-conjugacy and a bi-directional a full conjugacy), that the periodic points of logistic map $([0, 1], F_\mu)$, the tent map $([0, 1], T_2)$ and the doubling map $([0, 1], D)$ are dense in $[0, 1]$.
\begin{center}
    \begin{tikzpicture}
        \node(l1) {$\Sigma_2$};
        \node(r1) [right=of l1] {$\Sigma_2$};
        \node(l2) [below=of l1] {$[0, 1]$};
        \node(r2) [below=of r1] {$[0, 1]$};
        \node(l3) [below=of l2] {$[0, 1]$};
        \node(r3) [below=of r2] {$[0, 1]$};
        \node(l4) [below=of l3] {$[0, 1]$};
        \node(r4) [below=of r3] {$[0, 1]$};
    
        \draw[->] (l1.east) -- node[above] {$\sigma$} (r1.west);
        \draw[->] (l2.east) -- node[above] {$D$} (r2.west);
        \draw[->] (l1.south) -- node[left] {$\varphi_1$} (l2.north);
        \draw[->] (r1.south) -- node[right] {$\varphi_1$} (r2.north);

        \draw[->] (l3.east) -- node[above] {$T_2$} (r3.west);
        \draw[->] (l2.south) -- node[left] {$\varphi_2$} (l3.north);
        \draw[->] (r2.south) -- node[right] {$\varphi_2$} (r3.north);

        \draw[->] (l4.east) -- node[above] {$F_4$} (r4.west);
        \draw[<->] (l3.south) -- node[left] {$\varphi_3$} (l4.north);
        \draw[<->] (r3.south) -- node[right] {$\varphi_3$} (r4.north);
    \end{tikzpicture}
\end{center}
We next proved a simplified version of Sharkovsky's forcing theorem by Li and Yorke \cite{li-yorke}. This theorem stated that if a topological dynamical system $(I, f)$ where $I$ is a closed interval, has a period three point then it has periodic points of every other period. The full version of this theorem states that if a topological dynamical system has a period-$k$ point then there must exist a period-$l$ point for every $l \lhd k$ in the Sharkovsky ordering. Finally, in this chapter we proved Sharkovsky's realisation theorem, which states that every tail of Sharkovsky order is the set of periods for some topological dynamical system $(I, f)$. 

In Chapter \ref{chap:defining-chaos}, we introduced three important definitions of chaos and the topological properties they required. The first topological property we came across was that of topological transitivity, a property that ensures that points in arbitrary small neighbourhoods are mapped outside their initial neighbourhood under a repeated number of iterations of the map. Furthermore we also proved that if the topological dynamical system has no isolated points, such as in a closed interval, then transitivity and the existence of a dense orbit are equivalent. The next topological property we came across was that of sensitive dependence on initial conditions. This property ensures that points that are arbitrarily close together are eventually mapped arbitrary far apart under repeated applications of the map. We also touched on another property called expansiveness, where all points arbitrarily close together are mapped arbitrarily far apart. Our first notion of chaos came from Devaney \cite{devaney} and is widely accepted as a leading definition of chaos. Devaney's interpretation of chaos included unpredictability via sensitive dependence on initial conditions, repetitive behaviour through dense periodic points and indecomposability through topological transitivity. However, we proved that sensitive dependence on initial conditions is redundant. Furthermore if the topological dynamical system is defined over a closed interval the density of periodic points is also redundant. Hence on a topological dynamical system $(I, f)$ where $I$ is a closed interval, topological transitivity implies that the system is chaotic in the sense of Devaney. Furthermore, due to topological transitivity and the density of periodic points being purely topological properties, we proved that Devaney chaos is a conjugacy invariant property. Finally, we proved that $([0, 1], F_4)$, $([0, 1], T_2)$ and $([0, 1], D)$ all exhibit Devaney chaos as they are all topologically transitive. The second definition of chaos we uncovered was Li-Yorke chaos \cite{li-yorke}. This definition originally only applied to topological dynamical systems defined over closed intervals and required that the topological dynamical system contains an uncountable set with no asymptotically stable points. This notion was later generalised by Blanchard et al.\cite{blanchard} using Li-Yorke pairs and scrambled sets. We defined a Li-Yorke pair to be two points which can be mapped arbitrarily far apart and mapped together under multiple iterations of the map, ensuring the iterates of these points are essentially scrambled throughout the entire set $X$. The final definition we covered was topological chaos, the definition of which being a topological dynamical system with positive topological entropy. We gave two equivalent definitions for topological entropy itself. The first was by Adler et al.\cite{adler} and defined topological entropy using open covers. The second was by Bowen and Dinaburg \cite{bowen} \cite{dinaburg} and uses the language of metric spaces. From these definitions we concluded that topological entropy is a non-negative real number describing the complexity of a topological dynamics system by the asymptotic mean growth in the number of distinguishable collections of orbits at an arbitrarily fine resolution. We then proved that if $(X, f)$ is a topological dynamical system where $f$ is an isometry we have that the topological entropy of the system is zero, and hence is not topologically chaotic. Furthermore we proved that topological entropy and hence topological chaos it a conjugacy invariant property of topological dynamical systems. We then proved that the the doubling map $(S^1, \mathcal{D})$ and shift map $(\Sigma_2, \sigma)$ are topologically chaotic with a positive topological entropy equal to $\log 2$. Finally, to conclude the chapter we noted that topological chaos implies Li-Yorke chaos, and if $(I, f)$ is a topological dynamical system defined over a closed interval $I$ then Devaney chaos is equivalent to topological chaos.

This concludes our study of different types of chaos in topological dynamics systems. Note that the types of chaos we have studied is not exhaustive. Other definitions of chaos in literature include distributional chaos, $\omega$ chaos, $P$ chaos, Block chaos, Wiggins chaos, etc. As we have seen throughout this text, there is no one,  universally accepted definition of chaos. However, all definitions we have covered in this text believe that chaotic topological dynamical systems are irregular and complex on even the most arbitrarily small scales. From our analysis, topological transitivity appears to be the most important property of topological dynamical systems which are Devaney chaotic. Hence it is important that chaotic topological dynamical systems can map points in any arbitrarily small neighbourhood to any other arbitrary neighbourhood under repeated iterations. Furthermore, the existence of an uncountable scrambled set is the important property in Li-Yorke chaos. This means that chaotic topological dynamical systems must map an uncountable set of pairs of points arbitrarily far apart and arbitrarily close together under repeated iterations. Furthermore, topological chaos has the important property of positive topological entropy. Hence, the asymptotic mean growth in the number of collections of orbits at arbitrarily fine resolutions must be positive in a chaotic topological dynamical system. All of these properties we have outlined ensure that chaos has the property of arbitrarily fine complexity. The difference between Devaney chaos, Li-Yorke chaos and topological chaos is essentially how they characterise this arbitrarily fine complexity in their definition.


\bibliography{chaos}

\end{document}