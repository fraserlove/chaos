\documentclass[11pt]{article}

% Language setting
\usepackage[english]{babel}

% Set page size and margins
\usepackage[a4paper,top=2cm,bottom=2cm,left=3cm,right=3cm,marginparwidth=1.75cm]{geometry}

% Packages
\usepackage{amsmath}
\usepackage{amsfonts}
\usepackage{amsthm}
\usepackage{graphicx}
\graphicspath{ {./images/} }
\theoremstyle{definition}
\newtheorem{exmp}{Example}[section]

\title{Chaos in Discrete Dynamical Systems}
\author{Fraser Love}

\begin{document}
\maketitle

\noindent\textit{I certify that this project report has been written by me, is a record of work carried out by me, and is essentially different from work undertaken for any other purpose or assessment.}

\vspace{0.1cm}\hspace{13.1cm}\includegraphics[width=1.3cm]{signature}

\begin{abstract}
    You should include an abstract (i.e. a summary of the project in 100-200 words) and a list of contents in the preliminary pages. Some students even include a dedication.
\end{abstract}

\tableofcontents

\section{Introduction}
\subsection{Dynamical Systems}
A dynamical system is a set of states governed by a rule which specifies how the states evolve over time. Dynamical systems can either be discrete or continuous throughout time. A continuous dynamical system is described by a differential equation of any form. For example, $dx/dt = F(x)$, where $F: X \to X$ is some function. In this project however, we will be focusing on examining deterministic discrete dynamical systems only i.e. discrete systems where the present state is uniquely determined by previous states.

A discrete dynamical system (or topological dynamical system) is given by a map $f : X \to X$, where $X$ is a non-empty compact metric space. The system starts at a point $x$ and evolves through iterative applications of the map $f$ on points in $X$. After $n$ iterations of $f$ the system can be described by $f^n(x) = f \circ f \circ \cdots \circ f(x)$ where $n \in \mathbb{N}$. Any point $x_n$ can be described by the transformation of $f$ on its previous state as follows: $x_n = f(x_{n-1})$. 
\subsubsection{Initial Examples}
\begin{exmp}
The simplest discrete dynamical system could be the doubling map: \[f : \mathbb{R} \to \mathbb{R}, x \mapsto 2x\] Where each new state can be calculated from the previous state as follows: \[x_n = 2x_{n-1}\] Let $x_0$ be the initial value of the system. The value of the system after $n$ iterations of the map $f$ is given by: \[x_1 = f(x_0) = 2x_0\] \[x_2 = f(x_1) = f^2(x_0) = f^2(x_0) = 4x_0\] \[\vdots\] \[x_n = f(x_{n-1}) = f^2(x_{n-2}) = \cdots = f^n(x_0) = 2^nx_0\] This dynamical system grows exponentially and is unbounded.
\end{exmp}

\begin{exmp}
    A more interesting example is the logistic map $g: [0,1] \to [0,1]$ where $g(x)=rx(1-x)$. Lets initially set $r = 2$. Hence we have the discrete system: \[x_n = g(x_{n-1}) = 2x_{n-1}(1 - x_{n-1})\] For $x \ll 1$, $g(x) \simeq 2x$, however for $x \gg 0$, $g(x) \simeq 2(1-x) < 1$. It will be shown later that any initial $x_0 \neq 0, 1$ will be attracted towards the fixed point of $x = 1/2$. The logistic map is integral to discrete dynamical systems and we shall examine it alot more further into the project.
\end{exmp}

\subsubsection{Orbits, Trajectories and Periodic Points}
Let $f: X \to f(X)$ be a map and $x \in X$. The orbit of $x$ under $f$ is the set \[\mathcal{O}_f(x) = \lbrace f^n(x) : n \geq 0 \rbrace = \lbrace x, f(x), f^2(x), \cdots \rbrace\] of iterates of $x$ under $f$. The trajectory of $x$ is the infinite sequence $(f^n(x))_{n \geq 0}$. If $x$ is periodic there will be repetitions in this sequence.
\\ \\
A periodic point is a point $x \in X$ such that $f^n(x) = x$ for some $n \in \mathbb{N}$. Hence, the period of a point $x$ is the least positive integer $p$ such that $f^p(x) = x$. Moreover $f^n(x) = x \iff n = kp$ for some $k \in \mathbb{N}$ (i.e. $n$ is a multiple of $p$). This means that $\mathcal{O}_f(x) = \lbrace x, f(x), \cdots, f^{p-1}(x) \rbrace$ is a finite set of unique points.
\\ \\ A fixed point is simply a periodic point of period one, so $f(x) = x$. Hence $\mathcal{O}_f(x) = \lbrace x \rbrace$. The fixed points of a system can be simply calculated by setting $f(x) = 0$ and solving for $x$.

\begin{exmp}
Find the fixed points of the logistic map: \[g(x) = 2x(1-x) = 2x - 2x^2 = 0\] Hence, $x = 0$ and $x = 1/2$ are fixed points, which is true as $g(\frac{1}{2}) = 2 \cdot \frac{1}{2} \cdot (1 - \frac{1}{2}) = \frac{1}{2}$ and $g(0) = 2 \cdot 0 \cdot (1 - 0) = 0$
\end{exmp}

\subsubsection{Cobweb Plots}
A cobweb plot can be used to explore the convergence of fixed and periodic points. In such a plot the sucessive iterates of a map $f$ are plotted on the graph of $f$ and $y = x$, with horizontal and vertical lines used to track the evolution of the system.
\includegraphics[width=7cm]{cobweb_0.2_2.0}
\includegraphics[width=7cm]{cobweb_0.2_3.0}

\section{One-Dimensional Maps}
\subsection{Fixed and Periodic Points}
\subsection{Stability of Fixed Points}
\subsection{Logistic Maps}
\section{What is Chaos?}
\cite{sparrow2012lorenz}
Brief overview into definitions of chaos with examples.
\section{Periodic Points and Stability}
\section{Higher-Dimensional Dynamical Systems}
\section{Higher-Dimensional Stability}
\section{Chaos in Higher Dimensions}
\subsection{Chaotic Attractors}
\section{Fractals}
\subsection{Julia Sets}

\bibliographystyle{abbrv}
\bibliography{chaos}

\end{document}