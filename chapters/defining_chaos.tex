The term \emph{chaos} in Mathematics is vauge and has no universally accepted definition, hence various interpretations of chaos been invented throughout the years. As mentioned previously, the paper \emph{'Period Three Implies Chaos'} by Li and Yorke \cite{li-yorke} first introduced the term in a mathematical context. However this paper never formally defines a notion of chaos. In subsequent years, various mathematicians have attempted to define their own interpretations of chaos. All these interpretations rely on the topological dynamical system exhibiting various defined topological characteristics. The properties relied upon by the definitions of chaos we will look at include: topological transitivity or the existence of a dense orbit, sensitive dependence on initial conditions, that perodic points are dense in the underlying metric space, and the existence of an uncountable scrambled set. In Section \ref{sec:devaney-chaos} we shall explore a widely accepted definition of chaos, termed \emph{Devaney chaos}, developed by Devaney \cite{devaney}. This definition relies upon the topological dynamical system having three properties, namely topological transitivity, sensitive dependence on initial conditions and that perodic points are dense in the underlying metric space. However, we shall show that if we have a topological dynamical system with no isolated points then the latter two properties are redundant, and are implied by topological transitivity. In Section \ref{sec:li-yorke-chaos} we shall explore \emph{Li-Yorke chaos} \cite{li-yorke}. This interpretation of chaos explores the types of topological dynamical systems Li and Yorke termed chaotic through the powerful definition of scrambled sets. Notably this type of chaos relies on the existence of an uncountable scrambled set. Finally in Section \ref{sec:topological-chaos} we shall study topological chaos and topological entropy. Specifically we will find that positive topological entropy is the only requirement for a topological dynamical system to be topologically chaotic. Topological dynamical systems that exhibit all these types of chaos will be explored. We shall see that even the most trivial topological dynamical systems such as the tent map and logistic map exhibit chaos according to multiple interpretations. Before we introduce these various types of chaos however, we first need to set up some preliminary definitions.

\section{Topological Characteristics of Chaos} \label{sec:characteristics-of-chaos}

There are many different types of chaos in topological dynamical systems. All of these different types of chaos require the topologically dynamical system to display different topologically chaotic properties.
This first definition is a prime characteristic of chaotic systems. In fact we shall show later that in topological dynamical systems defined over a closed interval it is the only characteristic needed to prove that a system is chaotic in the sense of Devaney.

\begin{defn}[Topological Transitivity] \label{defn:topological-transitivity}
    Let $(X, f)$ be a topological dynamical system. The map $f$ is \emph{topologically transitive} if for every pair of non-empty open sets $U, V \subseteq X$ there exists $k > 0$ such that $f^k(U) \cap V \neq \emptyset$.
\end{defn}

Alternatively stated, in a topologically transitive mapping, points in an arbitraily small neighbourhood can be mapped to any other arbitrary neighbourhood under a repated number of iterations of the map. Hence the topological dynamical system cannot be partitioned into two disjoint non-empty open sets which are invarient under the map -- i.e.\ if $U \in X$ then $f(U) \in U$.

\begin{exmp}
    Let $(S^1, \mathcal{D})$ be the doubling map over $S^1$, where $\mathcal{D}(z) = z^2$. Let $z_1, z_2 \in S^1$. Let $(z_1, z_2) = U$ define an arc between $z_1$ and $z_2$. Suppose now $d\left(z_1, z_2\right) > \frac{2\pi}{2^k}$ for some $k \in \mathbb{N}$. Then $d\left(\mathcal{D}^k(z_1), \mathcal{D}^k(z_2)\right) = d\left( 2^k z_1, 2^k z_2 \right) = 2^k d\left( z_1, z_2 \right) > 2^k \cdot \frac{2\pi}{2^k} = 2\pi$. Hence $\mathcal{D}^k((z_1, z_2))$ covers $S^1$ so for any $V \subseteq S^1$ we obtain $\mathcal{D}^k(U) \cap V \neq \emptyset$. Hence $f$ is topologically transitive.
\end{exmp}

This next example, from \cite{kolyada-snoha}, provides a creative way of producing topologically transitive topological dynamical systems by using the orbit of any other topological dynamical system.

\begin{exmp}
    Let $(X, f)$ be a topological dynamical system and let $x \in X$ be a periodic point of $f$. Clearly $\mathcal{O}_f(x) = Y$ is finite and so $(Y, f|_y)$ is a topological dynamical system. $(Y, f|_y)$ is transitive as if $U \subseteq Y$ is open, then $f^k(U) = Y$ for some $k > 0$ and so $f^k(U) \cap V \neq 0$ for all $V \subseteq Y$ open.
\end{exmp}

This next proposition, proved by Silverman \cite{silverman}, states that if $(X, f)$ is a topological dynamical system and $X$ has no isolated points, the existence of a dense orbit of $f$ is equivalent to $f$ being topologically transitive. In his paper, Silverman explicitly states that transitivity implies the existence of a dense orbit if $X$ is separable and second-catagory. However, since our definition of a topological dynamical system takes $X$ to be a compact metric space these propeties automatically hold. This proposition is hugely important some authors give slight variations between definitions of chaos, some definitions ask for the existence of a dense orbit and others ask for topological transitivity. Hence using this proposition we can show that these definitions are equivalent in topological dynamical systems. The proof of this proposition follows Ruette \cite[\S 2.1]{ruette}.

\begin{prop} \label{prop:dense-transitive}
    Let $(X, f)$ be a topological dynamical system and suppose $X$ has no isolated points. The map $f$ is topologically transitive if and only if there exists some $x \in X$ such that $\mathcal{O}(x)$ is dense in $X$.
    \begin{proof}
        Assume that $f$ is transitive and let $U$ be a non-empty open set. By transitivity, for every non-empty open set $V$ there exists a $k > 0$ such that $f^k(U) \cap V \neq 0$. Hence, $\bigcup_{k \geq 0}f^{-n}(U)$ is dense in $X$. As $X$ is compact, there exists a countable basis of non-empty open sets $(U_n)_{n \geq 0}$. For all $l \geq 0$, $\bigcup_{k \geq 0}f^{-k}(U_n)$ is dense by transitivity. Now define $G = \bigcap_{n \geq 0}\bigcup_{k \geq 0}f^{-k}(U_n)$. This is a dense $G_\delta$ set. If $x \in G_\delta$ then $f^k(x)$ enters any set $U_n$ for some $k$. Hence $\mathcal{O}_f(x)$ is dense in $X$. For the reverse direction let $U, V \subseteq X$ be open with $U, V \neq \emptyset$. Let $x \in X$ such that $\overline{\mathcal{O}_f(x)} = X$. Then there exists $k \in \mathbb{N}$ such that $f^k(x) \in U$. Since $X$ has no isolated points $V\, \backslash \left\lbrace f^i(x) : 0 \leq i \leq k \right\rbrace$ is open and non-empty. Hence there exists an $l \in \mathbb{N}$ such that $f^l(x) \in V\, \backslash \left\lbrace f^i(x) : 0 \leq i \leq k \right\rbrace$. Since $l > k$ and $f^l(x) = f^{l - k} \circ f^{k}(x) \in f^{l - k}(U) \cap V$ we get $f^{l - k}(U) \cap V \neq 0$. Hence $f$ is transitive.
    \end{proof}
\end{prop}

Since closed intervals have no isolated points, for topological dynamical systems $(I, f)$ where $I$ is a closed interval, topological transitivity and the existence of a dense orbit are interchangable. Note that the condition for the topological dynamical system to have no isolated points is necessary. The following is a counterexample to show the definitions are not equivalent for topological dynamical systems in general.

\begin{exmp} \label{exmp:dense-orbit-not-equal-transitive}
    Let $(X, f)$ be the topological dynamical system defined over the space $X = \left\lbrace 0 \right\rbrace \cup \left\lbrace 2^{-n} : n \in \mathbb{N} \right\rbrace$ with the standard metric, and the map $f$ where $f(0) = 0$ and $f(2^{-n}) = 2^{-n-1}$. It can be seen that $(X, d)$ is a compact space, where $d$ is the standard metric as $X$ is a closed subset of $\mathbb{R}$, containing its limit point $0$. The set $X$ contains infinitely many isolated points as we can choose $B_d(x, \varepsilon)$ around each $x = 2^{-n} \in X$ with $\varepsilon < \frac{1}{4} \min\left\lbrace 2^{-n} - 2^{-n-1}, 2^{-n + 1} - 2^{-n} \right\rbrace$ such that the open balls are disjoint and $B_d(x, \varepsilon) = \left\lbrace x \right\rbrace$. Now let $U = \left\lbrace \frac{1}{2} \right\rbrace$ and $V = \left\lbrace 1 \right\rbrace$. Then $f^k(U) = f^k(\left\lbrace \frac{1}{2} \right\rbrace) = \left\lbrace 2^{-k-1} \right\rbrace$. Hence $f^k(U) \cap V = \left\lbrace 2^{-k-1} \right\rbrace \cap \left\lbrace 1 \right\rbrace = \emptyset, \ \forall k \in \mathbb{N}$. Hence $f$ is not topologically transitive, however $\mathcal{O}(1) = \left\lbrace 1, 2^{-1}, 2^{-2}, \dots \right\rbrace$ is dense as $\overline{\left\lbrace2^{-n}: n \in \mathbb{N}\right\rbrace} = X$.
\end{exmp}

Now lets introduce and example of a topological dynamical system with no isolated points. By Proposition \ref{prop:dense-transitive} we shall see that this max has both topological transitivty and the existence of a dense orbit.

\begin{exmp} \label{exmp:dense-orbit-and-transitive}
    In Example \ref{exmp:rigid-rotations} we introduced the rigid rotations, a topological dynamical system $(S^1, R_\alpha)$ where $R_{\alpha}(z) = ze^{i\alpha}$. Furthermore in Proposition \ref{prop:rigid-rotations-irrational} we proved that the irrational rotations gave rise to dense oribts and that these orbits where infinite. Hence $S^1$ does not contain an isolated point. Using Proposition \ref{prop:dense-transitive} we see that $(S^1, R_\alpha)$ is topologically transitive.
\end{exmp}

\begin{exmp} \label{exmp:logistic-tent-doubling-transitive}
    In Proposition \ref{prop:logisitc-tent-doubling-periodic-dense} we proved that the periodic points of the logistic map $([0, 1], F_4)$, the tent map $([0, 1], T_2)$ and the doubling map $([0, 1], D)$ are dense in $[0, 1]$. Clearly all these systems have no isolated points. Hence, using Proposition \ref{prop:dense-transitive} we can clearly see that all these maps are topologically transitive.
\end{exmp}

An important fact to note is that topological dynamical systems that are topologically transitive with an isolated point are in fact trivial, having only one periodic orbit. Hence in the rest of this text we shall restrict our study to the topological dynamical systems without an isolated point. Continuing, lets introduce our second topological characteristic of chaos, coming from Devaney \cite{devaney}.

\begin{defn}[Sensitive Dependence On Initial Conditions] \label{defn:sensitive-dependence}
    Let $(X, f)$ be a topological dynamical system and $\varepsilon > 0$. A point $x \in X$ is \emph{$\varepsilon$-unstable} if, for every neighbourhood $U$ of $x$, there exists a point $y \in U$ and $k \geq 0$ such that $d\left(f^k(x), f^k(y)\right) \geq \varepsilon$. The map $f$ has $\emph{sensitive dependence on initial conditions}$ if for all points $x \in X$, $x$ is $\varepsilon$-unstable.
\end{defn}

In other words, there exist points arbitrary close to $x$ that eventually get mapped at least $\varepsilon$ far apart under multiple applications of the map. Hence this definition states that small pertubations between iterates may eventually increase through repeated iterations of the map to become wildly different over time; behaviour which hopefully feels notionally chaotic to the reader. Note that the definition states that at least one point contained within each neighbourhood of $x$ gets mapped arbitraily far apart, not all points. Here is an example of a topological dynamical system with sensitive dependence on initial conditions.

\begin{exmp} \label{exmp:doubling-map-s1-sensitive}
    Let $(S^1, \mathcal{D})$ be the doubling map over the $S^1$, where $\mathcal{D}(z) = z^2$. Let $z_1, z_2 \in S^1$, $\varepsilon < 2^k \delta$ and suppose $d(z_1, z_2) = \delta$, then $d\left(\mathcal{D}^k(z_1), \mathcal{D}^k(z_2)\right) =  d\left(2^kz_1, 2^kz_2\right) = 2^k d(z_1, z_2) = 2^k \delta > \varepsilon$. Hence we can always choose a $k$ large enough so this holds, and so $f$ has sensitive dependence on initial conditions.
\end{exmp}

The same is true for the doubling map $([0, 1], D)$ over the unit interval. Note the example above displays a strong type of sensitive dependence on initial conditions termed expansiveness. In a topological dynamical system which exhibits expansiveness, all points arbitraily close together eventually get mapped arbitraily far apart; not just a proper subset of points as for sensitive dependence on initial conditions. We can see that the rigid rotations do not statisfy the definition of having sensitive dependence on initial conditions.

\begin{exmp} \label{exmp:rigid-rotations-not-sensitive}
    Let $(S^1, R_\alpha)$ be the rigid rotations. Let $z_1,z_2 \in S^1$, $\varepsilon > 0$ and suppose $d(z_1, z_2) < \varepsilon$, then since $R_\alpha$ is an isometry $d(R_\alpha^k(z_1), R_\alpha^k(z_2)) = d(z_1, z_2) < \varepsilon$. Hence $(S^1, R_\alpha)$ does not have sensitive dependence on initial conditions.
\end{exmp}

This concludes our study of topological characteristics of chaos. We shall now use the definitions of topological transitivity or the existence of dense orbit and sensitive dependence on initial conditions to define notions of chaos in topological dynamical systems and investigate examples of chaotic systems.

\section{Devaney Chaos} \label{sec:devaney-chaos}

Our first notion of chaos was developed by Devaney \cite{devaney} and is one of the most widely used definitions of chaos. Devaney's interpretation of chaos includes unpredictability via sensitive dependence on initial conditions, repetitive behaviour through periodic points being dense, and should be indecomposable through topological transitivity. Two characteristics which we developed in Section \ref{sec:characteristics-of-chaos}.

\begin{defn} [Devaney Chaos] \label{defn:devaney-chaos}
    A topological dynamical system $(X, f)$ is \emph{chaotic in the sense of Devaney} if it is topologically transitive, has sensitive dependence on initial conditions, and if the periodic points of $f$ are dense in $X$.
\end{defn}

The main feature of Devaney chaos is topological transitivity. After Devaney released this definition Banks et al.\ \cite{bbcds} and Glasner et al.\ \cite{glasner-weiss} showed that sensitive dependence on initial conditions is redundant. Note that this result holds even for a general mapping $f: X \to X$.

\begin{prop} \label{prop:transitivity-dense-periodic-implies-sdic}
    Let $(X, f)$ be a topological dynamical system. If the map $f$ is topologically transitive and has dense periodic points then $f$ has sensitive dependence on initial conditions.
    \begin{proof}
        Let $(X, d)$ be a metric space. Observe that we can find a $\delta_0 > 0$ such that for all $x \in X$ there exists a periodic point $q \in X$ such that $dist(\mathcal{O}_f(q), x) \geq \delta_0/2$. Proving this, take $q_1, q_2$ to be arbitrary periodic points where $\mathcal{O}_f(q_1) \cup \mathcal{O}_f(q_2) = \emptyset$. Let $\delta_0 = dist(\mathcal{O}_f(q_1), \mathcal{O}_f(q_2))$ and suppose $q_1' \in \mathcal{O}_f(q_1)$ and  $q_2' \in \mathcal{O}_f(q_2)$ are points such that $d(q_1', q_2') = \delta_0$. For all $x \in X$ either $d(q_1', x) \leq d(q_2', x)$ or by symmetry $d(q_2', x) \leq (q_1', x)$. Using the triangle inequality we find $d(q_1', q_2') \leq d(q_1', x) + d(x, q_2')$ for all $x \in X$. Hence we either have $\delta_0 = d(q_1', q_2') \leq 2d(q_1', x)$ or $\delta_0 = d(q_1', q_2') \leq 2d(q_2', x)$. Therefore we either have $dist(\mathcal{O}_f(q_1), x) \geq \delta_0/2$ or $dist(\mathcal{O}_f(q_2), x) \geq \delta_0/2$. Using this observation we can now prove $f$ has sensitive dependence on initial conditions. First let $\delta = \delta_0/8$ and let $x \in X$ be arbitrary, $x \in N$ where $N$ is an open neighbourhood. Since the periodic points of $f$ are dense in $X$ there exists a period-$n$ point $p \in U = N \cap B(x, \delta)$ open. By the observation above, there exists a periodic point $q \in X$ with $dist(\mathcal{O}_f(p), x) \geq 4\delta$. Now define $V = \bigcap_{i=0}^n f^{-i}(B(f^i(q), \delta))$. Since $V$ is a finite intersection of open sets, it itself is open. Moreover $q \in V$ so $V$ is non-empty. Since $f$ is topologically transitive, there exists a $y \in U$ with natural number $k > 0$ such that $f^k(y) \in V$. Now suppose $j = \left\lfloor \frac{k}{n} \right\rfloor + 1$ such that $1 \leq nj - k \leq n$. Hence, $f^{nj}(y) = f^{nj - k}(f^k(y)) \in f^{nj - k}(V) \subseteq B(f^{nj - k}(q), \delta)$. We also have $f^{nj}(p) = p$, so by the triangle inequality $d(f^{nj}(p), f^{nj}(y)) = d(p, f^{nj}(y)) \geq d(x, f^{nj - k}(q)) - d(f^{nj - k}(q), f^{nj}(y)) - d(p, x)$. Now as $p \in B(x, \delta)$ and $f^{nj}(y) \in B(f^{nj - k}(q), \delta)$ we have $d(f^{nj}(p), f^{nj}(y)) > 4\delta - \delta - \delta = 2\delta$. Hence, by the triangle inequality either $d(f^{nj}(x), f^{nj}(y)) > \delta$ or $d(f^{nj}(x), f^{nj}(p)) > \delta$. By definition, $f$ has sensitive dependence on initial conditions.
    \end{proof}
\end{prop}

For a general topological dynamical system this is the only superfluous property. Silverman \cite{silverman} and Vellekoop and Berglund \cite{vellekoop-berglund} later proved that for a topological dynamical system $(I, f)$ where $I$ is a closed interval, topological transitivity or equivalently the existence of a dense orbit, implies $f$ has dense periodic points in $X$. We shall introduce and later prove this result, but first we need the following lemma, which can be found here \cite[\S 4.1]{block-coppel}.

\begin{lem} \label{lem:closed-interval-no-periodic-points}
    Let $f: I \to I$ be a continuous map and $I$ an interval. Suppose $J \subseteq I$ is an interval which contains no periodic points of $f$. If $z, f^m(z), f^n(z) \in J$ where $m, n \in \mathbb{N}, \ m < n$ then either $z < f^m(z) < f^n(z)$ or $z > f^m(z) > f^n(z)$.
    \begin{proof}
        Suppose there exists a $z \in J$ such that $z < f^m(z)$ and $f^m(z) > f^n(z)$. Define $g(x) = f^m(x)$, so $z < g(z)$. If $g^{k+1}(x) < g(z)$ for some $k \in \mathbb{N}, \ n \geq 1$ then $g^k(z) - z$ has a positive value in $z$ and a negative value in $g(z)$ and by the Intermediate Value Theorem contain a point $c \in (z, g(z)) \subseteq J$ with $g^k(c) - c = 0$ and hence a $km$-periodic point. Therefore $z < g^k(z)$ for all positive integers $k$. Now let $k = n - m > 0$. Then $z < f^{(n - m)m}(z)$. Assuming $f^{(n-m)}(f^n(z)) < f^m(z)$ then taking $g = f^{n-m}(x)$ similarly yields $f^{(n-m)m}(f^m(z)) < f^m(z)$. However this results in the function $f^{(n-m)m}(x) - x$ having a positive and negative value in $f^m(z)$. Hence, by the Intermediate Value Theorem a $(n-m)m$-periodic point exists in $J$, a contradiction. The other case for $z > f^m(z) > f^n(z)$ can be proved similarly.
    \end{proof}
\end{lem}

\begin{prop} \label{prop:transitivity-interval-implies-dense-periodic}
    Let $(I, f)$ be a topological dynamical system. If the map $f$ is topologically transitive then $f$ has a dense set of periodic points.
    \begin{proof}
        We shall aim for a contradiction. Suppose that the periodic points are not dense in $I$, so there exists an interval $J \subseteq I$ where $J$ contains no periodic points. Let $x \in J$ where $x$ is not an endpoint and let $N \subsetneq J$ be a neighbourhood of $x$. Also let $E = J\, \backslash\, N$. Since $f$ is topologically transitive on $I$ there exists a positive integer $m$ with $f^m(N) \cap E \neq \emptyset$. Hence there exists a $y \in J$ such that $f^m(y) \in E \subsetneq J$ and since $J$ contains no periodic points $y \neq f^m(y)$. Moreover, since $f$ is continuous there exists an open neighbourhood $U$ of $y$ such that $f^m(U) \cap U \neq \emptyset$. Using topological transitivity again we can find a $n > m$ and a $z \in U$ with $f^n(z) \in U$. However then $0 < m < n$ with $z \in f^n(U)$ and $z \notin f^m(U) \implies z \leq f^n(z) \leq f^m(z)$. This is a contradiction by Lemma \ref{lem:closed-interval-no-periodic-points}. Hence the periodic points of $f$ are dense.
    \end{proof}
\end{prop}

Note that this result cannot hold generally because of the ordering over $\mathbb{R}$ used in Lemma \ref{lem:closed-interval-no-periodic-points}. Using Propositions \ref{prop:transitivity-dense-periodic-implies-sdic} and \ref{prop:transitivity-interval-implies-dense-periodic} we can clearly see that if $(I, f)$ is a topologically transitive topological dynamical system defined over an closed interval then it is chaotic in the sense of Devaney, giving us the following important result.

\begin{prop}\label{prop:chaotic-transitive}
    Let $(I, f)$ be a topological dynamical system where $I$ is a closed interval. If $f$ is topologically transitive then $(I, f)$ is chaotic in the sense of Devaney.
    \begin{proof}
        Suppose $f$ is topologically transitive. Propositions \ref{prop:transitivity-dense-periodic-implies-sdic} and \ref{prop:transitivity-interval-implies-dense-periodic} tell us that $f$ has sensitive dependence on initial conditions and the periodic points of $f$ are dense in $I$. Hence $(I, f)$ is chaotic in the sense of Devaney.
    \end{proof}
\end{prop}

As a result of this proposition, topological dynamical systems which have a topologically transitive map over a closed interval are automatically chaotic in the sense of Devaney. Here is a counterexample, proving that this result does not hold generally for topological dynamical systems.

\begin{exmp}
    Let $(S^1, R_\alpha)$ where $R_\alpha(z) = ze^{i\alpha}$ be the topological dynamical systems described by rigid rotations, except we now we shall take $\alpha$ to be solely irrational. In Example \ref{exmp:dense-orbit-and-transitive} we proved these topological dynamical systems to be topologically transitive, as irrational rotations gave rise to infinite, dense orbits. However we also proved in Example \ref{exmp:rigid-rotations-not-sensitive} that the system does not have sensitive dependence on initial conditions. Furthermore, irrational rotations give rise to infinite orbits and so no periodic points exist. Hence we have an example of a topological dynamical system which is topologically transitive but neither has dense periodic points nor sensitive dependence on initial conditions.
\end{exmp}

The example above gives great insight into the troubles involved in trying to develop an all encompassing definition of chaos. The irrational rotations of $S^1$ have dense orbits and so are topologically transitive, but the map is not particularly interesting as nearby points are constantly mapped near together, never giving way to irratic or uncontrollable behaviour. Looking back at the definition of Devaney chaos it is clear that all of the conditions for chaos are topological and so are preserved under topological conjugate maps. This makes looking for Devaney chaotic systems much easier by the following proposition.

\begin{prop}
    Let $(X, f)$ and $(Y, g)$ be topologically conjugate, topological dynamical systems. If $f$ is chaotic in the sense of Devaney then $g$ is chaotic in the sense of Devaney.
    \begin{proof}
        By Proposition \ref{prop:transitivity-dense-periodic-implies-sdic} sensitive dependence on initial conditions was found to be redundant for topological dynamical systems. In Proposition \ref{prop:conjugacy-preserves-dense-periodic-points} we proved that if $Per(f)$ are dense in $X$ then $Per(g)$ are dense in Y. Hence we just need to prove that topological conjugacy preserves topological transitivity. Let $\varphi: X \to Y$ be a topological conjugacy between $(X, f)$ and $(Y, g)$ and suppose $f$ is topologically transitive. Let $U, V \subseteq Y$ be non-empty open sets. Since $\varphi$ is surjective $\varphi^{-1}(U)$ and $\varphi^{-1}(V)$ are non-empty. As $f$ is topologically transitive, there exists a positive integer $k > 0$ such that $f^k(\varphi^{-1}(U)) \cap \varphi^{-1}(V) \neq \emptyset$. Let $x \in \varphi^{-1}(U)$ such that $f^k(x) \in \varphi^{-1}(V)$. Now set $y = \varphi(x) \in U$ and note that $\varphi \circ f^k(x) = g^k \circ \varphi(x) = g^k(y)$. Therefore $g^k(y) = \varphi \circ f^k(x) \in V$ and so $g^k(U) \cap V \neq \emptyset$.
    \end{proof}
\end{prop}

Now lets introduce some examples of topological dynamical systems that exhibit Devaney chaos, using topological conjugacy and building on the properties we have already observed in various topological dynamical systems.

\begin{exmp}
    In Example \ref{exmp:logistic-tent-doubling-transitive} we showed that the periodic points of the logistic map $([0, 1], F_4)$, the tent map $([0, 1], T_2)$ and the doubling map $([0, 1], D)$ are all topologically transitive. By Proposition \ref{prop:chaotic-transitive} these systems are all chaotic in the sense of Devaney.
\end{exmp}

\section{Li-Yorke Chaos} \label{sec:li-yorke-chaos}

Now onto our second definition of chaos. As mentioned in Section \ref{sec:sharkovskys-theorem-and-type}, the paper \emph{`Period Three Implies Chaos'} by Li and Yorke \cite{li-yorke} first introduced the term chaos in a mathematical context. In this paper they stated two properties of interval maps that lead to chaotic behaviour, namely sensitive dependence on initial conditions and the existence of an uncountable set with no periodic points. This was formally introduced in the following theorem.

\begin{thm} \label{thm:li-yorke-chaos-intervals}
    If $f: I \to I$ be a continuous interval map with a period three point, then there exists an uncountable set $S \subseteq I$ (containing no periodic points) such that, for all $x, y \in S$ where $x \neq y$
    \[\limsup_{n \to +\infty}\left\lvert f^n(x) - f^n(y) \right\rvert > 0, \ \ \ \ \liminf_{n \to +\infty}\left\lvert f^n(x) - f^n(y) \right\rvert = 0\] and for all periodic points $z \in S$ \begin{equation} \label{equ:no-assympotic-points}\limsup_{n \to +\infty}\left\lvert f^n(x) - f^n(z) \right\rvert > 0.\end{equation}
\end{thm}

Clearly by the requirement of (\ref{equ:no-assympotic-points}) in this theorem, the set $S$ contains no asymptotically stable points. Li and Yorke noted that interval maps which satisfied this equation displayed irratic and irregular behaviour. Hence within this theorem they have defined a sense of chaos in interval maps. Note that this theorem does not hold for general topological spaces. For instance, take the rigid rotations $(S^1, R_{2\pi/3})$ as an example. Every point $z \in S^1$ is a period three point as $R^3(z) = ze^{3i \cdot 2\pi/3} = z$, however there does not exist an uncountable set $S \subseteq S^1$ containing no periodic points, so Theorem \ref{thm:li-yorke-chaos-intervals} does not hold. Since this theorem was published, various authors have generalised the definition of Li-Yorke chaos to the realm of topological dynamical systems. We shall be taking the definition from Blanchard et al.\ \cite{bgsm} which uses the notion of a Li-Yorke pair, defined as follows.

\begin{defn}[Li-Yorke Pair] \label{defn:li-yorke-pair}
    Let $(X, f)$ be a topological dynamical system with $x, y \in X$ and $\delta > 0$. The pair $(x, y)$ is a \emph{Li-Yorke pair} if \[\limsup_{n \to +\infty} d\left( f^n(x), f^n(y) \right) \geq \delta \ \ \ \text{and} \ \ \ \liminf_{n\to+\infty} d\left( f^n(x), f^n(y) \right) = 0.\]
\end{defn}

Hence if $(x, y)$ is a Li-Yorke pair then $x$ and $y$ can be mapped at least $\delta$ far apart under multiple iterations of the map. This is the behaviour we defined, in Definition \ref{defn:sensitive-dependence}, as sensitive dependence on initial conditions. Furthermore if $(x, y)$ is a Li-Yorke pair then $x$ and $y$ can be mapped to the same point under multiple iterations of the map. In this regard we can think of iterations of these two points being scrambled amoungst the whole set $X$. Lets now define this behaviour generally over a whole set, to introduce a definition of Li-Yorke chaos

\begin{defn} [Scrambled Set, Li-Yorke Chaos] \label{defn:scrambled-set}
    A set $S \subseteq X$ is \emph{scrambled} if for all distinct $x, y \in S$, $(x, y)$ is a Li-Yorke pair. A topological dynamical system $(X, f)$ is \emph{chaotic in the sense of Li-Yorke} if there exists an uncountable scrambled set $S \subseteq X$.
\end{defn}

Note that in this general version of Li-Yorke chaos the requirement for the set $X$ to have a period three point has been excluded. Furthermore the last requirement, in (\ref{equ:no-assympotic-points}), that no points converge asymptotically to periodic points has been removed. Infact, this extra requirement makes no difference for chaos in the sense of Li-Yorke as if $S$ is a scrambled set then evey point except, at most, one point of $S$ satisfy this requirement. Recently it was shown by Lu et al.\ \cite{lu-zhu-wu} that topological conjugacy does not preserve Li-Yorke chaos. Using Theorem \ref{thm:li-yorke-chaos-intervals} we can prove that the following topological dynamical systems are Li-Yorke chaotic.

\begin{exmp}
    Take $([-1, 1], f)$ to be the topological dynamical system where $f(x) = 2 |x| - 1$. Clearly $x = \frac{1}{9}$ is a period three point as $f^3\left(\frac{1}{9}\right) = f^2\left(\frac{-7}{9}\right) = f\left(\frac{5}{9}\right)$. Hence by Theorem \ref{thm:li-yorke-chaos-intervals} $([-1, 1], f)$ is chaotic in the sense of Li-Yorke.
\end{exmp}

\begin{exmp}
    Let $([0, 1], F_\mu)$ where $1 + 2\sqrt{2} \leq \mu \leq 4$. Note that when $\mu = 1 + 2\sqrt{2}$ a period three point emerges, and hence by Theorem \ref{thm:li-yorke-chaos-intervals} is Li-Yorke chaotic.
\end{exmp}

\section{Topological Chaos and Entropy} \label{sec:topological-chaos}
Next we shall explore our final definition of chaos: topological chaos. This definition relies heavily on topological entropy, a conjugacy invarient property exhibited by some topological dynamical systems. Topological entropy is a non-negative real number describing the complexity of a topological dynamical system by the asymptotic mean growth in the number of distinguishable collections of orbits at an arbitraily fine yet finite resolution. The quantity was first outlined by Adler et al.\ \cite{adler} and uses the language of open covers. Later Bowen \cite{bowen} and Dinaburg \cite{dinaburg} reformulated this definition in terms of a metric and the separation of orbits. When the underling metric space is compact, i.e.\ in a topological dynamical system, these two definitions become equivalent. First lets introduce the former definition, which in an essense is more natural as it does not depend on the underlying metric space and so is more general in a topological sense.

\begin{defn}[Topological Entropy - Adler et al.]
    Let $(X, f)$ be a topological dynamical system, $\mathcal{C} = \left\lbrace C_1, C_2, \dots C_p \right\rbrace$, $\mathcal{D} = \left\lbrace D_1, D_2, \dots D_q \right\rbrace$ be finite covers and define the cover $\mathcal{C} \vee \mathcal{D} = \left\lbrace C_i \cap D_j : i \in [1, p], \ j \in [1, q] \right\rbrace$. The cover $\mathcal{C}$ is \emph{finer} than $\mathcal{D}$ if every element of $\mathcal{D}$ is also included in $\mathcal{C}$, and is expressed as $\mathcal{C} \prec \mathcal{D}$. Let $N(\mathcal{C})$ be the minimum cardinality of a subcover of $\mathcal{C}$, so $N(\mathcal{C}) = \min \left\lbrace n : \exists i(1), \dots, i(n) \in [1, p],\ X = C_{i(1)} \cup \dots \cup C_{i(n)}\right\rbrace$. Then, for all integers $n \geq 1$ we can define $N_n(\mathcal{C}, f) = N\left(\mathcal{C} \vee f^{-1}(\mathcal{C}) \vee \dots \vee f^{-(n-1)}(\mathcal{C})\right)$. The \emph{topological entropy} of the finite cover $\mathcal{C}$ is given by \[h(\mathcal{C}, f) = \lim_{n \to +\infty}\frac{\log{N_n(\mathcal{C}, f)}}{n} = \inf_{n \geq 1} \frac{\log{N_n(\mathcal{C}, f)}}{n}.\] The \emph{topological entropy} according to Adler et al.\,, denoted $h_A(f)$, of the topological dynamical system $(X, f)$ is given by \[h_{A}(f) = \sup\left\lbrace h(\mathcal{U}, f): \mathcal{U} \ \text{finite open cover of} \ X \right\rbrace.\]
\end{defn}

Now we shall introduce the Bowen-Dinaburg definition of topological entropy using the language of metric spaces. First we need to define the notion of $(n, \varepsilon)$-separated and $(n, \varepsilon)$-spanning sets, given by Bowen \cite{bowen}.

\begin{defn}[$(n, \varepsilon)$-Separated, $(n, \varepsilon)$-Spanning]
    Let $(X, f)$ be a topological dynamical system defined on the metric space $(X, d)$. A set $E \subseteq X$ is \emph{$(n, \varepsilon)$-separated} if for all distinct $x, y \in E$ there exists a $k$ with $0 \leq k < n$ such that $d(f^k(x), f^k(y)) \geq \varepsilon$. For $n \in \mathbb{N}$ where $n \geq 1$, define $d_n(x, y) = \max{\left\lbrace d(f^k(x), f^k(y)) : 0 \leq k < n \right\rbrace}$ and $B_n(x, \varepsilon) = \left\lbrace y \in X : d_n(x, y) < \varepsilon \right\rbrace$. A set $E \subseteq X$ is \emph{$(n, \varepsilon)$-spanning} if $X \subseteq \bigcup_{x \in E}B_n(x, \varepsilon)$. Let $r(n, \varepsilon)$ denote the minimum cardinality of an $(n, \varepsilon)$-spanning set and $s(n, \varepsilon)$ denote the maximum cardinality of an $(n, \varepsilon)$-separated set.
\end{defn}

Note that since compactness guarentees we can find a finite subcover for $X$ there always exists a $(n, \varepsilon)$-spanning set and a $(n, \varepsilon)$-spanning set with a finite cardinalities. In the above definition, $\varepsilon$ can be considered the resolution, i.e.\ the minimum distance at which the two points become distinguishable, with $r(n, \varepsilon)$ describing the minimum number of collections of indistinguishable orbits and $s(n, \varepsilon)$ describing the maximum number of collections of distinguishable orbits. This is because we have defined an $(n, \varepsilon)$-separated set to be a set such that all points in the set get mapped at least $\varepsilon$ away from each other in at least one of the first iterations of the map. Note that the following is an alternative form of the definition of a spanning set. If $F$ is an $(n, \varepsilon)$-spanning set then for every $x \in X$ there is a $y \in F$ for which $d(f^k(x), f^k(y)) \leq \varepsilon$ for all $0 \leq k < n$. The following lemma, also from Bowen \cite{bowen}, is particularly useful to setup Bowen and Dinaburg's definition of topological entropy. The proof of this folowing lemma is adapted from analysis by Ruette \cite[§4.1]{ruette}.

\begin{lem} \label{lem:finite-maximum-minimum-spanning-separated}
    If $(X, f)$ is a topological dynamical system with $\varepsilon > 0$ and $n \in \mathbb{N}$, then $r(n, \varepsilon) \leq s(n, \varepsilon) \leq r(n, \varepsilon / 2) < \infty$.
    \begin{proof}
        Suppose $E \subseteq X$ is an $(n, \varepsilon)$-separated set of maximum cardinality $s(n, \varepsilon)$. By the maximality of $E$, for every $y \in X\, \backslash \, E$, $E \cup \left\lbrace y \right\rbrace$ is not $(n, \varepsilon)$-separated, or alternatively $y \in \bigcup_{x \in E}B_n(x, \varepsilon)$. Clearly $E \subseteq \bigcup_{x \in E}B_n(x, \varepsilon)$ so $E$ is an $(n, \varepsilon)$-spanning set, and so $r(n, \varepsilon) \leq s(n, \varepsilon)$. Let $F$ be an $(n, \varepsilon / 2)$-spanning set of cardinality $r(n, \varepsilon / 2)$. For every $x \in X$, there exists $y(x) \in F$ such that $x \in B_n(y(x), \varepsilon / 2)$. If $x_1, x_2 \in E$ are distinct then we must have $y(x_1) \neq y(x_2)$ as then we would have $d(f^k(x_1), f^k(x_2)) < \varepsilon$ for $0 \leq k < n$. Hence $s(n, \varepsilon) \leq r(n, \varepsilon / 2)$.
    \end{proof}
\end{lem}

Finally we can now introduce Bowen and Dinaburg's definition of topological entropy.

\begin{defn}[Topological Entropy - Bowen and Dinaburg]
    Let $(X, f)$ be a topological dynamical system defined on the metric space $(X, d)$. Define \[\overline{h}(\varepsilon, f) = \limsup_{n \to \infty}\frac{\log{s(n, \varepsilon)}}{n} = \limsup_{n \to \infty} \frac{\log{r(n, \varepsilon)}}{n}.\] Note that these limits exists and is finite, as proved by Lemma \ref{lem:finite-maximum-minimum-spanning-separated}. The \emph{topological entropy} according to Bowen and Dinaburg, denoted $h_B(f)$, of the topological dynamical $(X, f)$ is given by \[h_B(f) = \sup_{\varepsilon > 0^+}\overline{h}(\varepsilon, f) = \lim_{\varepsilon \to 0^+}\overline{h}(\varepsilon, f).\]
\end{defn}

By taking the logarithm of $s(n, \varepsilon)$ or $r(n, \varepsilon)$ and dividing by $n$ we obtain the mean growth of distinguishable collections of orbits, which are at least $\varepsilon$ far apart in the first $n$ iterations of the map. Therefore by then taking the limit superior we obtain the mean asymptotic growth in the number of these distinguishable collections of orbits. Letting $\varepsilon$ tend to zero we get the asymptotic mean growth in the number of collections of orbits at an arbitraily fine resolution; this is the definition of topological entropy. As mentioned above, both definitions of topological entropy are equivalent in topological dynamical systems.

\begin{prop}
    If $(X, f)$ is a topological dynamical system, then $h_A(f) = h_B(f)$.
    \begin{proof}
        Since $X$ is a compact metric space $X$ we can find an open cover $\mathcal{U} = \left\lbrace U_1, U_2, \dots, U_n \right\rbrace$ of $X$ with $\text{diam}(U_i) \leq \varepsilon$ for all $i \in [1, n]$ and Lebesgue number $2\delta$. Hence we obtain $s(n, \varepsilon) \leq N(\mathcal{U}^n) \leq s(n, \delta)$ implying that $h_A(f) = h_B(f)$.
    \end{proof}
\end{prop}

Throughout the rest of this text we shall write $h_{top}(f)$ to denote either $h_A(f)$ or $h_B(f)$, depending on circumstances. The entropy of a topological dynamical system may be zero. If the underlying map is an isometry we get the following result.

\begin{prop} \label{prop:isometry-entropy}
    Let $(X, f)$ be a topological dynamical system. If $f: X \to X$ is an isometry, then $h_{top}(f) = 0$.
    \begin{proof}
        Let $(X, d)$ be the underlying metric space of $(X, f)$. Take $x, y \in X$ and let $\mathcal{U}$ be an open cover of $X$. As $f$ is an isometry $d(x, y) = d(f^n(x), f^n(y))$ where $n \in \mathbb{N}$. Hence $N_n(\mathcal{U}, f)$ is remains constant, no matter the choice of $n$. Therefore $h_{top}(\mathcal{U}, f) = \inf_{n \geq 1} \frac{1}{n}\log{N_n(\mathcal{C}, f)} = 0$.
    \end{proof}
\end{prop}

This result is clearly true as isometric mappings preserve distance, and so there should be a constant number of distinguishable orbits under repeated applications of the map. We shall now prove that topological entropy is a purely topological property, with its value being independent of the choice of metric.

\begin{lem} \label{lem:entropy-independent-of-metric}
    Let $(X, f)$ be a topological dynamical system defined over two equivalent metrics $d$ and $d'$. The topological entropy of $(X, f)$ with respect to $d$ and $d'$ is the same.
    \begin{proof}
        Consider the following map  $I: (X, d) \to (X, d')$ between metric spaces. As $d$ and $d'$ are equivalent, $I$ is a homeomorphism. Since $X$ is compact, $I$ is uniformly continuous. Hence if $\varepsilon > 0$ there exists a $\delta > 0$ such that $d(x, y) < \delta$ which implies that $d'(x, y) < \varepsilon$. Specifically if $d_n(x, y) \leq \delta$ then $d_n(x, y) \leq \varepsilon$ where $n \in \mathcal{N}$. Hence any $(n, \delta)$-spanning set is also a $(n, \varepsilon)$-spanning set. So $s_d(n, \delta) \geq s_{d'}(n, \varepsilon)$ and therefore we obtain  \[h_{top_{d'}} = \lim_{\varepsilon \to 0^+}{\overline{h}_d(\varepsilon, f)} \leq \lim_{\delta \to 0^+}{\overline{h}_{d'}(\delta, f)} = h_{top_d}.\] By repetition of the same arguement, except with the map $I': (X, d) \to (X, d')$ we obtain $h_{top_{d'}}(f) \geq h_{top_d}(f)$. Therefore $h_{top_{d'}}(f) = h_{top_d}(f)$.
    \end{proof}
\end{lem}

This lemma proves particularly useful in proving that topological entropy is a conjugacy invarient property of topological dynamical systems.

\begin{prop} \label{prop:conjugacy-preserves-entropy}
    If $(X, f)$ and $(Y, g)$ topological dynamical systems which are topologically conjugate via the conjugacy $\varphi: X \to Y$, then $h_{top}(f) = h_{top}(g)$.
    \begin{proof}
    Let $d$ be a metric on $X$ and let $d'$ be the metric on $Y$ defined by $d'(y_1, y_2) = d(\varphi(y_1), \varphi(y_2))$ where $y_1, y_2 \in Y$. By Lemma \ref{lem:entropy-independent-of-metric} we know that $h_{top}(g)$ is independent on the definition of $d'$. Using the definition of $d_n$ we get that $d_n'(y_1, y_2) = \max{\left\lbrace d'(f^k(x), f^k(y)): 0 \leq k \leq n - 1 \right\rbrace} = \max{\left\lbrace d(\varphi(f^k(x)), \varphi(f^k(y))): 0 \leq k \leq n - 1 \right\rbrace} = \max{\left\lbrace d(f^k(\varphi(x)), f^k(\varphi(y))): 0 \leq k \leq n - 1 \right\rbrace}  = d_n(\varphi(y_1), \varphi(y_2))$. Hence as $\varphi$ is a bijection $(n, \varepsilon)$-separated sets and $(n, \varepsilon)$-spanning sets have the same cardinality for $X$ and $Y$. Therefore it follows that $h(g) = h(f)$.
    \end{proof}
\end{prop}

The converse statement is not true generally. To see this, let $(R_\alpha, S^1)$ be the irrational rigid rotations and $(R_\beta, S^1)$ be the rational rigid rotations. As the rigid roatations are an isometry, by Proposition \ref{prop:isometry-entropy} the topological entropy of both systems is zero, however $(R_\alpha, S^1)$ and $(R_\beta, S^1)$ are not topologically conjugate. In the following example we shall now calculate the topological entropy of the shift map $(\Sigma_2, \sigma)$. Note that this example uses the following proposition, which we shall state without proof as this requires some insight into measure theory of which the reader is not assumed to be familar with, for more detail see Walters \cite{walters}.

\begin{prop} \label{prop:entropy-generator}
    Let $(X, f)$ be a topological dynamical system and let $C$ be a topological generator. Then $h_{top}(f) = h_{top}(C, f)$.
\end{prop}

Note that proof of the above just requires us to prove $h_{top}(C, f) \geq h_{top}(f)$ as the reverse is true by definition as a topological generator is simply a finite open cover. This next example follows work done by Adler et al. \cite{adler}.

\begin{exmp} \label{exmp:shift-entropy}
    Let $(\Sigma_2, \sigma)$ denote the shift map and let $C = \left\lbrace [0], [1] \right\rbrace$ partition $\Sigma_2$. Then for $n \in \mathbb{N}$, $\bigvee_{j = 0}^{n - 1}\sigma^{-j}(C)$ is a partition of $\Sigma_2$ into $2^n$ sets. Since $C$ is a topological generator, by Proposition \ref{prop:entropy-generator}, $h_{top}(\sigma) = h_{top}(C, \sigma) = \lim_{n \to +\infty}\frac{1}{n} \log N_n(C, \sigma) = \lim_{n \to +\infty}\frac{1}{n} \log 2^n = \log 2$.
\end{exmp}

Before we introduce the following example for finding the topological entropy of the doubling map $(S^1, \mathcal{D})$ we need to establish the following lemma and propositions, which follow working by Butt \cite{butt}.

\begin{lem} \label{lem:metric-less-quarter}
    If $(S^1, \mathcal{D})$ is the doubling map with underlying metric space $(S^1, d)$ where $d$ is the arc length metric, then for $x, y \in S^1$ we have $d(x, y) \leq \frac{1}{4} \implies d(f(x), f(y)) = 2d(x, y)$.
    \begin{proof}
        Clearly $d(x, y) = |x - y|$ when $|x - y| \leq \frac{1}{2}$. Let $x, y$ be such that $d(x, y) \leq \frac{1}{4}$. Hence we have, $d(S^1(x), S^1(y)) = d(2x \mmod 1, 2y \mmod 1) = \min{(|2x - 2y \mmod 1|, 1 - |2x - 2y \mmod 1|)}$. As $|2x - 2y| \leq \frac{1}{2}$ whe have that $2x - 2y \mod 1 = 2x - 2y$. Therefore, $d(S^1(x), S^1(y)) = 2|x-y| = 2d(x, y)$.
    \end{proof}
\end{lem}

\begin{prop} \label{prop:spanning-set}
    The set $S_{n+k} = \left\lbrace \frac{i}{2^{n+k}} : 0 \leq i < 2^{n+k} - 1 \right\rbrace$ is an $(n, \varepsilon)$-spanning set for the doubling map $(S^1, \mathcal{D})$. 
    \begin{proof}
        Let $\varepsilon > 0$ and choose $k \geq 2$ such that $\frac{1}{2^{k+1}} \leq \varepsilon < \frac{1}{2^k}$. Note that for any $x \in S^1$ we have that $x \in \left[\frac{i}{2^{n+k}}, \frac{i + 1}{2^{n+k}}\right)$ where $0 \leq i < 2^{n+k} - 1$. Then choose $y \in S_{n+k}$ to be either endpoints of this dyadic interval so that $d(x, y) \leq \frac{1}{2^{n+k}} < \frac{1}{4}$. By Lemma \ref{lem:metric-less-quarter} we obtain, $d(\mathcal{D}(x), \mathcal{D}(y)) = 2d(x, y) \leq \frac{2}{2^{n+k}} < \frac{1}{4}$. Applying Lemma \ref{lem:metric-less-quarter} a total of $j$ times, where $0 \leq j < n$ we obtain $d(\mathcal{D}^j(x), \mathcal{D}^j(y)) = 2^jd(x, y) \leq \frac{2^j}{2^{n - k}} \leq \frac{2^n - 1}{2^{n - k}} < \frac{1}{2^{k + 1}} \leq \varepsilon$. Hence for any $x \in S^1$ we have $d_n(x, y) = \max{\left\lbrace d(\mathcal{D}^j(x), \mathcal{D}^j(y)) : 0 \leq j < n \right\rbrace} < \varepsilon$ for some $y \in S_{n + k}$, and so $S_{n+k}$ is a $(n, \varepsilon)$-spanning set for $(S^1, \mathcal{D})$.
    \end{proof}
\end{prop}

\begin{prop} \label{prop:separared-set}
    The set $S_{n-1+k} = \left\lbrace \frac{i}{2^{n-1+k}} : 0 \leq i < 2^{n - 1 + k} - 1 \right\rbrace$ is an $(n, \varepsilon)$-separared set for the doubling map $(S^1, \mathcal{D})$. 
    \begin{proof}
        Let $\varepsilon > 0$ and choose $k \geq 2$ such that $\frac{1}{2^{k+1}} \leq \varepsilon < \frac{1}{2^k}$. Let $x, y \in S_{n-1+k}$ be distinct. Note that we want to prove that $d_n(x, y) \geq \varepsilon$, that is, prove that there exists a $j$ where $0 \leq j < n$ such that $d(\mathcal{D}^j(x), \mathcal{D}^j(y)) \geq \varepsilon$ (this is the definition of being a $(n, \varepsilon)$-separated set). Now suppose there exists a $j$ such that $d(\mathcal{D}^j(x), \mathcal{D}^j(y)) \geq \frac{1}{4}$, then we are done as $\varepsilon < \frac{1}{4}$ by assumption. Hence suppose $d(\mathcal{D}^j(x), \mathcal{D}^j(y)) \leq \frac{1}{4}$ for all $0 \leq j < n$. Therefore we can apply Lemma \ref{lem:metric-less-quarter} a total of $n - 1$ times to show $d(\mathcal{D}^{n-1}(x), \mathcal{D}^{n-1}(y)) = 2^{n-1}d(x, y)$. Now note that for distinct $x, y \in S_{n-1+k}$ we get $d(x, y) \geq \frac{1}{2^{n-1+k}}$, so $2^{n-1}d(x, y) \geq \frac{2^{n-1}}{2^{n-1+k}} = \frac{1}{2^k} \geq \varepsilon$. Hence $S_{n-1+k}$ is a $(n, \varepsilon)$-separated set for $(S^1, \mathcal{D})$.
    \end{proof}
\end{prop}

\begin{exmp} \label{exmp:s1-doubling-entropy}
    Let $(S^1, \mathcal{D})$ be the doubling map over $S^1$. By Proposition \ref{prop:spanning-set} $S_{n+k}$ is $(n, \varepsilon)$-spanning set. Clearly this set has cardinality $2^{n+k}$, and so $r(n, \varepsilon) \leq 2^{n+k}$. Therefore we get that $\overline{h}(\varepsilon, \mathcal{D}) = \limsup_{n \to \infty}\frac{\log{r(n, \varepsilon)}}{n} \leq \limsup_{n\to\infty}\frac{(n + k)\log 2}{n} = \log 2$. Using Proposition \ref{prop:separared-set}, $S_{n - 1 + k}$ is a $(n, \varepsilon)$-separared set with cardinality $2^{n - 1 + k}$, so $s(n, \varepsilon) \geq 2^{n - 1 + k}$. Therefore we also get that $\overline{h}(\varepsilon, \mathcal{D}) = \limsup_{n \to \infty}\frac{\log s(n, \varepsilon)}{n} \geq \limsup_{n \to \infty}\frac{(n - 1 + k)\log 2}{n} = \log 2$. Hence using the squeeze theorem for limits we get that $h_{top}(\mathcal{D}) = \lim_{\varepsilon \to 0^+}\overline{h}(\varepsilon, \mathcal{D}) = \log 2$.
\end{exmp}

Generally it can be proved that for maps of the form $f(x) = \alpha x \mmod 1$, where $\alpha \in \mathbb{N}$, that $h(f) = \log \alpha$. Finally with a detailed background in topological entropy we can now define topological chaos. Before the term chaos was coined by Li and Yorke in `Period Three Implies Chaos', Furstenburg in \cite{furstenberg}, stated that all topological dynamical systems with zero topological entropy are `deterministic'. In a later papers, Glasner and Weiss \cite{glasner-weiss} and separtely Blanchard \cite{blanchard}, defined that topological dynamical systems that have positive topological entropy exhibit topological chaos.

\begin{defn}
    A topological dynamical system $(X, f)$ exhibits topological chaos if it has positive topological entropy.
\end{defn}

Using Proposition \ref{prop:conjugacy-preserves-entropy} we can clearly deduce that topological chaos is preserved under topological conjugacy. Now lets introduce some examples of topological dynamical systems that exhibit topological chaos.

\begin{exmp}
    In Example \ref{exmp:shift-entropy} and Example \ref{exmp:s1-doubling-entropy} we showed that the shift map $(\Sigma_2, \sigma)$ and the doubling map $(S^1, \mathcal{D})$ have topological entropy $h_{top}(\sigma) = h_{top}(\mathcal{D}) = \log 2 > 0$. Therefore $(\Sigma_2, \sigma)$ and $(S^1, \mathcal{D})$ exhibit topological chaos.
\end{exmp}

Blanchard et al. \cite{bgsm} proved that in a topological dynamical system, topological chaos implies the existence of Li-Yorke chaos. We shall not cover the proof of this proposition here as it contains ideas from ergodic theory and is outwith the bounds of this text.

\begin{prop}
    If a topological system $(X, f)$ is topologically chaotic then it is also Li-Yorke chaotic.
\end{prop}

In \cite{smital} Smital proved the inverse of this statement to be false implying there exists zero entropy topological dynamical systems which are Li-Yorke chaotic. Furthermore Li \cite{li} proved that for topological dynamical systems $(I, f)$ where $I$ is a closed interval positive topological entropy implies the system exhibits Devaney chaos, and vice versa.

\begin{prop}
    Let $(I, f)$ be a topological dynamical system and $I$ be a closed interval. The $(I, f)$ has positive topological entropy if and only if $(I, f)$ is chaotic in the sense of Devaney.
\end{prop}

Hence for topological dynamical systems $(I, f)$ where $I$ is a closed interval we get the following relation between the different types of chaos mentioned in this text. \[\text{Devaney chaos} \iff \text{Topological chaos} \implies\text{Li-Yorke chaos}.\]