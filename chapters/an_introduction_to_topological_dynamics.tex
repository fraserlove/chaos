The aim of this text is to introduce the reader to topological dynamics and explore the various interpretations of chaos through the lens of topology and topological dynamical systems. In advance of introducing topological dynamical systems we shall define the superset containing them; the set of discrete dynamical systems. A discrete dynamical system is defined by a metric space with a corresponding continuous function, sending the metric space to itself. This function is termed a map or mapping whereby points in the underlying metric space are mapped to other points in the set by the application of this function. Topological dynamical systems themselves are a subset of discrete dynamical systems, with the extra requirement that the underlying metric space is compact (i.e.\ complete and totally bounded). This extra condition for compactness is useful for investigating the limiting behaviour of the set of iterates of the map as it is repeatedly iterated \emph{ad infinitum}; a relevant feature in the study of chaos.

The concept of chaos, specifically deterministic chaos, has no universally accepted definition within the literature of discrete dynamics. The term was first coined by Li and Yorke in their ubiquitous paper `Period Three Implies Chaos' \cite{li-yorke}. Since then, a number of authors have proposed their own definitions of chaos, hinging on the existence of various properties of the topological dynamical system. The properties we shall be studying in this text include: topological transitivity / the existence of a dense orbit, the density of periodic points, the existence of an uncountable scrambled set, sensitive dependence on initial conditions and positive topological entropy. Due to the differences between definitions of chaos, topological dynamical systems can be chaotic according to one interpretation but not another. We shall aim to compare these definitions and understand their consequences, providing examples of topological dynamical systems that exhibit each type of chaos. Specifically, we shall restrict our attention to four compact metric spaces: closed intervals, the unit circle, sequence space and compact countable sets; as these spaces provide intuitive analysis and feature heavily in dynamical systems literature. This text assumes the reader to be a capable student of pure mathematics with a basic understanding of topology and analysis. The focus of this text is mainly topological; for the sake of brevity content from related areas of ergodic theory, group theory, measure theory, etc.\ are excluded.

This chapter will briefly review some relevant results from topology before introducing ideas central to topological dynamics and the study of chaos in topological dynamical systems. We shall also introduce some popular topological dynamical systems, which we will be examining throughout this text. We will eventually prove that all of these systems exhibit at least one form of chaos. Subsequently, in Chapter \ref{chap:conjugacy-symbol-dynamics} we introduce notions of comparing topological dynamical systems using the framework of topological conjugacy and symbolic dynamics. This will later allows us to transfer topological properties between the tent map, logistic map, doubling map and shift map, which we shall prove are all topologically semi-conjugate. Concluding Chapter \ref{chap:conjugacy-symbol-dynamics}, we shall begin our study of chaos by looking at the presence of points of different periods in topological dynamical systems defined over the real numbers via Sharkovsky's realisation and forcing theorems. Finally, in Chapter \ref{chap:defining-chaos}, we shall explore and discuss the different characteristics of chaos and their importance in providing a natural definition of chaos. Then we will study three widely studied interpretations of chaos called Devaney chaos, Li and Yorke chaos, and topological chaos; meanwhile analysing examples of topological dynamical systems which satisfy each definition. Concluding the text, we shall compare the various definitions of chaos, including the specific case when the underlying metric space is simply the real numbers.

\section{Topology, Discrete Dynamics and Topological Dynamical Systems} \label{sec:topological-dynamical-systems}
This section will start by formally defining a topological dynamical system. We will then follow by introducing some important definitions from topology and discrete dynamics. We shall refer back to these definitions constantly for the remainder of this text. Furthermore, prominent examples of topological dynamical systems will be presented. The reader should remember these examples as they will be integral to understanding several ideas in later chapters. Note that the definitions and results in this section apply generally to continuous maps over an arbitrary set, however, have been formulated in terms of topological dynamical systems for clarity and precision. To begin with, we shall introduce some preliminary definitions from topology.

\begin{defn}[Closure] \label{defn:closure}
    Let $(X, d)$ be a metric space and $Y \subseteq X$. A point $x \in Y$ is in the \emph{closure} of $Y$, denoted $\overline{Y}$, if for every $\varepsilon > 0$, there exists some $y \in Y$ such that $d(x, y) \leq \varepsilon$.
\end{defn}


\begin{defn}[Dense Set] \label{defn:dense}
    Let $(X, d)$ be a metric space and $Y \subseteq X$. The set $Y$ is \emph{dense} in $X$ if for every $x \in X$ and $\varepsilon > 0$ there exists a $y \in Y$ such that $d(x, y) \leq \varepsilon$, i.e.\ for every $x \in X$ there exists an open neighbourhood $U$ where $x \in U$, and a $y \in U$ such that $y \in Y$.
\end{defn}

\begin{defn}[Finite Cover, Open Cover] \label{defn:cover}
    Let $(X, d)$ be a metric space. A \emph{finite cover} is a collection of sets $\mathcal{C} = \left\lbrace C_1, C_2, \dots, C_n \right\rbrace$ such that $X \subseteq \bigcup_{i = 1}^nC_i$. If the sets $C_1, C_2, \dots C_n$ are all open then $\mathcal{C}$ is a \emph{open cover}.
\end{defn}

\begin{defn}[Compact  Space] \label{defn:compact}
    A metric space $(X, d)$ is \emph{compact} if every open cover $\mathcal{U}$ of $X$ has a finite subcover $\left\lbrace U_{i(1)}, U_{i(2)}, \dots, U_{i(n)} \right\rbrace \subseteq \mathcal{U}$, i.e.\ if $\left\lbrace U_i \right\rbrace_{i\in I}$ is a collection of open subsets of $X$, where $X \subseteq \bigcup_{i \in I}U_i$ then there exists a finite subcollection $\left\lbrace U_{i(1)}, U_{i(2)}, \dots, U_{i(n)} \right\rbrace \subseteq \mathcal{U}$ such that $X \subseteq \bigcup_{j = 1}^{n}U_{i(j)}$. Furthermore, a metric space is compact if and only if it is complete and totally bounded.
\end{defn}

Note that on $\mathbb{R}$ with the standard metric, all closed intervals are compact; this will be used throughout the text. Since we now have a notion of what it means for a metric space to be compact, we can now define the main object we shall be studying in this paper, the topological dynamical system.

\begin{defn}[Topological Dynamical System] \label{defn:topological-dynamical-system}
    Let $X$ be a non-empty compact metric space. A \emph{topological dynamical system} denoted $(X, f)$ is given by a continuous map $f: X \to X$. The system starts at an initial point $x \in X$ and evolves through successive iterations of the map $f$. After $n \in \mathbb{N}$ iterations of $f$, the system can be described by $f^n := f \circ f \circ \dots \circ f$, where $x$ is mapped to the point $f^n(x)$. By convention we take $f^0$ to be the identity map.
\end{defn}

Having defined a topological dynamical system, we can now characterise the discrete dynamics of the underlying map $f$ through the following definitions.

\begin{defn}[Orbit] \label{defn:orbit}
    Let $(X, f)$ be a topological dynamical system. The \emph{orbit} or \emph{forward orbit} of a point $x \in X$ under $f$ is the set $\mathcal{O}_f(x) = \mathcal{O}^+_f(x) = \lbrace f^n(x) : n \geq 0 \rbrace = \lbrace x, f(x), f^2(x), \dots \rbrace$ of iterates of $x$ under the map $f$. If $f$ is a homeomorphism (i.e $f^{-1}$ exists and is continuous) then the \emph{backward orbit} of $x$ under $f$ is similarly defined as $\mathcal{O}^-_f(x) = \lbrace f^n(x) : n \leq 0 \rbrace = \lbrace x, f^{-1}(x), f^{-2}(x), \dots \rbrace$.
\end{defn}

The reader should note that in this text, unless stated otherwise, the term, orbit, will simply refer to the forward orbit, as we will be exclusively dealing with forward dynamics. An interesting characteristic which can occur in topological dynamical systems is the existence of an orbit which is dense in the underlying set $X$. We shall explore various consequences that systems with this property hold in Chapter \ref{chap:defining-chaos}. For now, note that from Definition \ref{defn:dense} an orbit $\mathcal{O}_f(x)$ of $f$ is said to be dense in $X$ if, for every $x \in X$ and $\varepsilon > 0$ there exists a $y \in X$ and $k \in \mathbb{N}$ such that $d(f^k(y), x) \leq \varepsilon$. That is, for each point $x \in X$ we can find a point $y \in X$ such that the distance between $x$ and the $k$-th iterate of $y$ is arbitrarily small.

\begin{defn}[Periodic Point, Cycle] \label{defn:periodic-point}
    Let $(X, f)$ be a topological dynamical system. A point $x \in X$ is \emph{fixed} if $f(x) = x$ and \emph{periodic} if $f^n(x) = x$ for some $n \in \mathbb{N}$. The \emph{period} of a point $x$ is the least positive integer $k$ such that $f^k(x) = x$. If $x$ has a period of $k$ we say that $x$ is a \emph{period-$k$} point. The set of all period-$k$ points of $f$ is denoted by $\text{Per}_k(f)$ and the set of all periodic points of $f$ is denoted $\text{Per}(f)$. Moreover, $f^n(x) = x \iff n = lk$, for some $l \in \mathbb{N}$. The orbit $\mathcal{O}_f(x) = \lbrace x, f(x), \dots, f^{k-1}(x) \rbrace$ of a periodic point is a finite set of unique points, called a \emph{periodic orbit} of period $k$ or simply a \emph{k-cycle}.
\end{defn}

In most topological dynamical systems only a small subset of points are periodic. Most often a larger set of points either enter a periodic orbit after a certain number of iterations of $f$ or converge asymptotically to a periodic orbit, leading us directly into the following definitions.

\begin{defn}[Eventually Periodic, Asymptotically Periodic] \label{defn:eventually-asymptotically-periodic}
    Let $(X, f)$ be a topological dynamical system. A point $x \in X$ is \emph{eventually periodic} of period $k$ if the point $x$ is not periodic and there exists a $n > 0$ such that $f^{k+i}(x) = f^i(x)$, for $i \geq n$. The point $x \in X$ is \emph{asymptotically periodic} to a periodic point $p \in X$ if $\lim_{n \to \infty} d(f^n(x), f^n(p)) = 0$.
\end{defn}

\begin{prop} \label{prop:eventually-periodic-implies-periodic}
    Let $(X, f)$ be a topological dynamical system. If $f$ is an invertible map, then every eventually periodic point is periodic.
    \begin{proof}
        Suppose $x \in X$ is eventually periodic of period $k$ in $f$. Then $f^{k + i}(x) = f^i(x)$ for some $n > 0$. By applying $i$ iterations of $f^{-1}$ we obtain, $f^{-i} \circ f^{k + i}(x) = f^{-i} \circ f^{i}(x) \implies f^k(x) = x$. Hence $x$ is periodic with period $k$.
    \end{proof}
\end{prop}

When studying chaos in topological dynamical systems it can be useful to understand how the system behaves for an increasing number of iterations. The $\omega$-limit set, defined below as the set of limit points of a particular orbit, allows us to understand the behaviour of the system asymptotically.

\begin{defn}[Omega-limit Set] \label{defn:omega-limit-set}
    Let $(X, f)$ be a topological dynamical system. The $\omega$\emph{-limit} set of $x \in X$, denoted $\omega(x, f)$ is the set of all limit points of the orbit $\mathcal{O}_f(x)$ given by \[\omega(x, f) := \bigcap_{n=0}^\infty\overline{\left\lbrace f^k(x) : k \geq n \right\rbrace}\] and the $\omega$\emph{-limit} set of the entire map $f$ is defined as \[\omega(f) := \bigcup_{x \in X} \omega(x, f)\]
\end{defn}

By the definition, we can immediately see that the $\omega$-limit set of a period-$k$ point or eventually periodic point of period-$k$ is simply the $k$-cycle. Furthermore, if a point is asymptotically periodic then the $\omega$-limit set is clearly a cycle and hence finite. Now that elementary definitions of topological dynamics have been defined, we shall, analyse the dynamics of some popular topological dynamical systems.

\section{Examples of Topological Dynamical Systems}

In this section we shall provide some examples of topological dynamical systems. This first example is probably the most studied system in literature due to its simplicity and elegance.

\begin{exmp}[Logistic Map] \label{exmp:logitic-map}
    Define $F_{\mu}: [0, 1] \to [0, 1]$ to be the \emph{logistic map}, where $F_{\mu}(x)=\mu x(1-x)$ and $0 < \mu \leq 4$. Since $[0, 1]$ is a closed interval it is compact. Hence $([0, 1], F_{\mu})$ describes a topological dynamical system. Note that the logistic map only becomes chaotic when $\mu = 4$, whereby orbits of $F_4$ are cantor sets. Hence we shall only be studying the map for this value of $\mu$; however interesting results can be proved as $\mu \to 4$. Figure \ref{fig:logistic_3} gives an example of this map.

    \begin{figure}[h]
        \centering
        \includegraphics[width=4cm]{logistic_3}
        \caption{Logistic map $F_\mu$ with $\mu = 3$.}
        \label{fig:logistic_3}
    \end{figure}
\end{exmp}

\begin{exmp}[Tent Map] \label{exmp:tent-map}
    Define $T_s: [0, 1] \to [0,1]$ to be the \emph{tent map}, where $T_s(x) = sx$ for $x \in \left[0, \frac{1}{2}\right]$, $T_s(x) = s(1-x)$ for $x \in \left[\frac{1}{2}, 1\right]$ with $s \in (1, 2]$. Since $[0, 1]$ is a closed interval it is compact. Hence $([0, 1], T_s)$ describes a topological dynamical system. The dynamics of this system becomes difficult to understand as the number of iterations of $T_s$ increases. This is due to the function's piecewise definition with the total possibility of different applications of the tent map doubling for every extra iteration. In Chapter \ref{chap:conjugacy-symbol-dynamics} we shall develop the technique of using symbolic dynamics to better investigate the complex topological dynamics of this system. Figure \ref{fig:tent_1.75} gives an example of this map.

    \begin{figure}[H]
        \centering
        \includegraphics[width=4cm]{tent_1.75}
        \caption{Tent map $T_s$ with $s = \frac{7}{4}$.}
        \label{fig:tent_1.75}
    \end{figure}
\end{exmp}

\begin{exmp}[Doubling Map] \label{exmp:doubling-map}
    Define $D: [0,1] \to [0,1]$ to be the \emph{doubling map on} $[0, 1]$, where $D(x) = 2x$ for $x \in \left[0, \frac{1}{2}\right]$ and $D(x) = 2x - 1$ for $x \in \left[\frac{1}{2}, 1\right]$. Since $[0, 1]$ is a closed interval it is compact. Interestingly this system can be expressed over $S^1 = \left\lbrace z \in \mathbb{C}: |z| = 1 \right\rbrace = \left\lbrace e^{i\theta} : 0 \leq \theta \leq 2\pi \right\rbrace$, the unit circle in the complex plane. This space uses the arc length metric, where if $w = e^{i\theta}, z = e^{i\phi} \in S^1$ then $d(w, z) = d(e^{i\theta}, e^{i\phi}) = |\theta - \phi|$ or $1 - |\theta - \phi|$ depending on if $|\theta - \phi| \leq \frac{1}{2}$ or $|\theta - \phi| > \frac{1}{2}$. This re-expression over $S^1$ is possible due to the fact that $[0, 1]$ is homeomorphic to $S^1$ via the homeomorphism $\varphi : [0, 1] \to S^1$ where $\varphi(x) = e^{2\pi i x}$. Note that $S^1$ is a closed subset of $\mathbb{C}$, and so is compact. In this space we define the doubling map as $D: S^1 \to S^1$ where $D(z) = z^2$ or equivalently $D(e^{i\theta}) = e^{2i\theta}$ for some $\theta \in \mathbb{R}$. Figure \ref{fig:doubling} gives an example of the first iterations of this mapping. This alternative definition of $D$ removes the discontinuity at $x = \frac{1}{2}$ present in the former definition. Just like the tent map, the dynamics of this system become difficult to understand as the number of iterations of $D$ increases. Symbolic dynamics can also be heavily applied to this mapping to better understand the topological dynamics of this system, as we shall see in Chapter \ref{chap:conjugacy-symbol-dynamics}.

    \begin{figure}[H]
        \centering
        \includegraphics[width=4cm]{doubling}
        \hspace{1cm}
        \includegraphics[width=4.8cm]{doubling_circle}
        \caption{\emph{Left}: Doubling map $D$ defined over $[0, 1]$. \emph{Right}: Doubling map $D$ defined over $S^1$ with first ten iterations.}
        \label{fig:doubling}
    \end{figure}
\end{exmp}

\begin{exmp}[Rigid Rotations] \label{exmp:rigid-rotations}
    Define $R_\alpha: S^1 \to S^1$ to be the \emph{rigid rotations of the unit circle}, where $\alpha \in [0, 2\pi)$ and the function $R_{\alpha}(z) = ze^{i\alpha}$, or equivalently $R_\alpha(e^{i\theta}) = e^{i(\theta + \alpha)}$. Again $S^1$ is compact as it is a closed subset of $\mathbb{C}$. Hence $(S^1, R_{\alpha})$ describes a topological dynamical system. Figure \ref{fig:rigid-circle} gives an example of this map.

    \begin{figure}[h]
        \centering
        \includegraphics[width=4.8cm]{rigid_circle}
        \caption{First ten iterations of the rigid rotations $R_\alpha$.}
        \label{fig:rigid-circle}
    \end{figure}
\end{exmp}

An interesting property of the rigid rotations is how the behaviour of the dynamical system changes depending on the rationality or irrationality of $\alpha$. This brings us to the following proposition.

\begin{prop} \label{prop:rigid-rotations-irrational}
    If $\alpha$ is an irrational number, then for all $z \in S^1$, the orbit $\mathcal{O}_{R_\alpha}(z)$ is infinite and dense on $S^1$.
    \begin{proof}
        Let $z \in S^1$ be arbitrary. If $R_\alpha^m(z) = R_\alpha^n(z)$ for some $m, n \in \mathbb{Z}$ then $ze^{(m-n)i\alpha} = z \implies (m - n)i\alpha = 0$. Since $\alpha \notin \mathbb{Q}$ and $(m - n)\alpha \neq 2\pi n$ for $n \in \mathbb{N}$ we have $m = n$. Hence all the points in the orbit are distinct and so $\mathcal{O}_{R_\alpha}(z)$ is infinite. Now let $w \in S^1$ be arbitrary and let $\varepsilon > 0$. Choose $N$ such that $\frac{2\pi}{N} < \varepsilon$. Now there exists $0 \leq l, k \leq N$ such that $d\left( R_\alpha^k, R_\alpha^l \right) \leq \frac{2\pi}{N}$. As $R_\alpha$ is an isometry i.e. $d(R_\alpha(x), R_\alpha(y)) = d(x, y)$ we obtain $d(R_\alpha^{(k - l)}(z), z) \leq \varepsilon$. Now the set $X = \lbrace R_\alpha^{n(k - l)}(z) : n \in \mathbb{N} \rbrace$ partitions $S^1$ into arcs of of length less than $\varepsilon$. Hence there must exist a $R_\alpha^{i(k - l)}(z) \in X$ such that $d(R_\alpha^{i(k - l)}(z), w) \leq \varepsilon$ and so $\mathcal{O}_{R_{\alpha}}(z)$ is dense in $S^1$.
    \end{proof}
\end{prop}

Depending on the parameters $\mu$, $s$ and $\alpha$ for $F_\mu$, $T_s$ and $R_\alpha$ the dynamical behaviour of these systems can range from predictable and periodic to chaotic. In Chapter \ref{chap:conjugacy-symbol-dynamics} we shall subsequently prove that $T_2$, $F_4$ and $D$ are equivalent topologically speaking and share various topological properties. We shall prove in Chapter \ref{chap:defining-chaos} that all these systems are chaotic and exhibit highly complex and irregular dynamics.