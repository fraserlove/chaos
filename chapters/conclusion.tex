In Chapter \ref{chap:introduction}, we defined the notion of a topological dynamical system $(X, f)$ as being a non-empty compact metric space with a corresponding continuous map. We then introduced some examples of topological dynamical systems, namely the logistic map $([0, 1], F_\mu)$, the tent map $([0, 1], T_s)$, the rigid rotations $(S^1, R_\alpha)$, and the doubling map $([0, 1], D)$.

In Chapter \ref{chap:conjugacy-symbol-dynamics}, we covered symbolic dynamics and introduced the topological dynamical system of the shift map $(\Sigma_2, \sigma)$. Then we defined topological conjugation and proved that the shift map $(\Sigma_2, \sigma)$, the doubling map $([0, 1], D)$, the tent map $([0, 1], T_2)$, and the logistic map $([0, 1], F_\mu)$ are all topologically semi-conjugate or conjugate to each other, as described in the following diagram. Furthermore, we proved that various topological properties are conjugate invariant. Most significantly we proved the density of orbits and the density of periodic points are conjugate invariant. Hence, we deduced that by topological conjugacy, the doubling map, the tent map and the logistic map all have a dense orbit and dense periodic points.
\begin{center}
    \begin{tikzpicture}
        \node(t1) {$\Sigma_2$};
        \node(d1) [below=of t1] {$\Sigma_2$};
        \node(t2) [right=of t1] {$[0, 1]$};
        \node(d2) [right=of d1] {$[0, 1]$};
        \node(t3) [right=of t2] {$[0, 1]$};
        \node(d3) [right=of d2] {$[0, 1]$};
        \node(t4) [right=of t3] {$[0, 1]$};
        \node(d4) [right=of d3] {$[0, 1]$};
    
        \draw[->] (t1.south) -- node[left] {$\sigma$} (d1.north);
        \draw[->] (t2.south) -- node[left] {$D$} (d2.north);
        \draw[->] (t3.south) -- node[left] {$T_2$} (d3.north);
        \draw[->] (t4.south) -- node[left] {$F_4$} (d4.north);

        \draw[->] (t1.east) -- node[above] {$\varphi_1$} (t2.west);
        \draw[->] (d1.east) -- node[above] {$\varphi_1$} (d2.west);
        \draw[->] (t2.east) -- node[above] {$\varphi_2$} (t3.west);
        \draw[->] (d2.east) -- node[above] {$\varphi_2$} (d3.west);
        \draw[<->] (t3.east) -- node[above] {$\varphi_3$} (t4.west);
        \draw[<->] (d3.east) -- node[above] {$\varphi_3$} (d4.west);
    \end{tikzpicture}
\end{center}
Next, we proved a simplified version of Sharkovsky's forcing theorem by Li and Yorke \cite{li-yorke} which stated that if a system $(I, f)$ defined over a closed interval, has a period three point, then it has periodic points of every other period. Finally, in this chapter we proved that Sharkovsky's realisation theorem can be used to construct systems $(I, f)$ with a set of periods in the tail of the Sharkovsky order.

In Chapter \ref{chap:defining-chaos} we first introduced topological transitivity, and proved that it is equivalent to the existence of a dense orbit for systems with no isolated points, i.e.\ systems defined over closed intervals. Topological transitivity is a property that introduces indecomposability to the system. That is, points in arbitrary small neighbourhoods can be mapped to any other arbitrarily small neighbourhood under a repeated number of iterations of the map. Furthermore, the property of sensitive dependence on initial conditions, ensures that points which are arbitrarily close together get mapped arbitrarily far apart under repeated iterations of the map.

Then we introduced chaos in the sense of Devaney. This definition included unpredictability via sensitive dependence on initial conditions, repetitive behaviour through dense periodic points and indecomposability through topological transitivity. However, we proved sensitive dependence on initial conditions to be redundant, being implied by the other two conditions. Furthermore if the topological dynamical system is defined over a closed interval, then dense periodic points is also superfluous. Furthermore, since topological transitivity and dense periodic points are purely topological properties, we proved that Devaney chaos is a conjugacy invariant property. Using this fact we then proved that the shift map $(\Sigma_2, \sigma)$, the doubling map $([0, 1], D)$, the tent map $([0, 1], T_2)$, and the logistic map $([0, 1], F_\mu)$ are all chaotic in the sense of Devaney. 

The second definition of chaos we explored was Li-Yorke chaos. This definition originally only applied to continuous maps defined over closed intervals. This notion was later generalised by Blanchard et al.\cite{blanchard} using Li-Yorke pairs and scrambled sets. We defined a Li-Yorke pair to be two points which can be mapped arbitrarily far apart and mapped together under multiple iterations of the map, ensuring the iterates of these points are essentially scrambled throughout the entire set $X$. 

We then introduced topological chaos in topological dynamical systems, the definition of which is simply a system which has positive topological entropy. We gave two equivalent definitions for topological entropy itself. One defined topological entropy using open covers and the other using the language of metric spaces. Topological entropy is a non-negative real number describing the complexity of a topological dynamics system by the asymptotic mean growth in the number of distinguishable collections of orbits at an arbitrarily fine resolution. Furthermore we proved that topological entropy and topological chaos are conjugacy invariant properties. We then proved that the the doubling map $(S^1, \mathcal{D})$ and shift map $(\Sigma_2, \sigma)$ are topologically chaotic with a positive topological entropy equal to $\log 2$. Finally, to conclude the chapter we noted that topological chaos implies Li-Yorke chaos, and if $(I, f)$ is a topological dynamical system defined over a closed interval $I$ then Devaney chaos is equivalent to topological chaos.

This concludes our study of chaos in topological dynamics systems. Throughout this text we have introduced various different definitions of chaos for these systems. One shared characteristic of all these definitions however, is that chaos has irregular and complex behaviour. Specifically, chaos, is a combination of indecomposability, unpredictability and periodicity, at arbitrarily small scales.
