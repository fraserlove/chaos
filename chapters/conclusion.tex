In Chapter \ref{chap:introduction}, we defined the notion of a topological dynamical system $(X, f)$ as being a non-empty compact metric space $X$ with a corresponding continuous map $f: X \to X$. We then introduced some examples of topological dynamical systems, namely the logistic map $([0, 1], F_\mu)$, the tent map $([0, 1], T_s)$, the rigid rotations $(S^1, R_\alpha)$, the doubling maps defined over the unit interval $([0, 1], D)$ and unit circle the complex plane $(S^1, \mathcal{D})$. We also introduced important concepts from discrete dynamics, such as orbits, periodic points, cycles, eventually and asymptotically periodic points and $\omega$-limit sets.

Subsequently in Chapter \ref{chap:conjugacy-symbol-dynamics}, we introduced symbolic dynamics and topological conjugacy. We first defined topological conjugation, a term used to describe when two maps exhibit the same topological behaviour. We prove that various topological properties such as the density of orbits and periodic points, are conjugate invariant properties. Next we introduced symbolic dynamics, a field which studies how the shift map effects infinite sequences of symbols, called itineraries, that describe the complex dynamics of some topological dynamical systems. Specifically it was the assignment of orbits of a topological dynamical system to a sequence of discrete symbols in sequence space, denoted $\Sigma_2$. We defined the shift map $(\Sigma_2, \sigma)$ on sequence space and proved that it was in fact a topological dynamical system. Then we went on to prove the periodic points of the shift map are dense and that the shift map has a dense orbit. Furthermore, using topological conjugacy we proved that through various topological conjugacy's and semi-conjugacy's, shown in the diagram below (where an uni-directional arrow represents a semi-conjugacy and a bi-directional a full conjugacy), that the periodic points of logistic map $([0, 1], F_\mu)$, the tent map $([0, 1], T_2)$ and the doubling map $([0, 1], D)$ are dense in $[0, 1]$.
\begin{center}
    \begin{tikzpicture}
        \node(l1) {$\Sigma_2$};
        \node(r1) [right=of l1] {$\Sigma_2$};
        \node(l2) [below=of l1] {$[0, 1]$};
        \node(r2) [below=of r1] {$[0, 1]$};
        \node(l3) [below=of l2] {$[0, 1]$};
        \node(r3) [below=of r2] {$[0, 1]$};
        \node(l4) [below=of l3] {$[0, 1]$};
        \node(r4) [below=of r3] {$[0, 1]$};
    
        \draw[->] (l1.east) -- node[above] {$\sigma$} (r1.west);
        \draw[->] (l2.east) -- node[above] {$D$} (r2.west);
        \draw[->] (l1.south) -- node[left] {$\varphi_1$} (l2.north);
        \draw[->] (r1.south) -- node[right] {$\varphi_1$} (r2.north);

        \draw[->] (l3.east) -- node[above] {$T_2$} (r3.west);
        \draw[->] (l2.south) -- node[left] {$\varphi_2$} (l3.north);
        \draw[->] (r2.south) -- node[right] {$\varphi_2$} (r3.north);

        \draw[->] (l4.east) -- node[above] {$F_4$} (r4.west);
        \draw[<->] (l3.south) -- node[left] {$\varphi_3$} (l4.north);
        \draw[<->] (r3.south) -- node[right] {$\varphi_3$} (r4.north);
    \end{tikzpicture}
\end{center}
We next proved a simplified version of Sharkovsky's forcing theorem by Li and Yorke \cite{li-yorke}. This theorem stated that if a topological dynamical system $(I, f)$ where $I$ is a closed interval, has a period three point then it has periodic points of every other period. The full version of this theorem states that if a topological dynamical system has a period-$k$ point then there must exist a period-$l$ point for every $l \lhd k$ in the Sharkovsky ordering. Finally, in this chapter we proved Sharkovsky's realisation theorem, which states that every tail of Sharkovsky order is the set of periods for some topological dynamical system $(I, f)$. 

In Chapter \ref{chap:defining-chaos}, we introduced three important definitions of chaos and the topological properties they required. The first topological property we came across was that of topological transitivity, a property that ensures that points in arbitrary small neighbourhoods are mapped outside their initial neighbourhood under a repeated number of iterations of the map. Furthermore we also proved that if the topological dynamical system has no isolated points, such as in a closed interval, then transitivity and the existence of a dense orbit are equivalent. The next topological property we came across was that of sensitive dependence on initial conditions. This property ensures that points that are arbitrarily close together are eventually mapped arbitrary far apart under repeated applications of the map. We also touched on another property called expansiveness, where all points arbitrarily close together are mapped arbitrarily far apart. Our first notion of chaos came from Devaney \cite{devaney} and is widely accepted as a leading definition of chaos. Devaney's interpretation of chaos included unpredictability via sensitive dependence on initial conditions, repetitive behaviour through dense periodic points and indecomposability through topological transitivity. However, we proved that sensitive dependence on initial conditions is redundant. Furthermore if the topological dynamical system is defined over a closed interval the density of periodic points is also redundant. Hence on a topological dynamical system $(I, f)$ where $I$ is a closed interval, topological transitivity implies that the system is chaotic in the sense of Devaney. Furthermore, due to topological transitivity and the density of periodic points being purely topological properties, we proved that Devaney chaos is a conjugacy invariant property. Finally, we proved that $([0, 1], F_4)$, $([0, 1], T_2)$ and $([0, 1], D)$ all exhibit Devaney chaos as they are all topologically transitive. The second definition of chaos we uncovered was Li-Yorke chaos \cite{li-yorke}. This definition originally only applied to topological dynamical systems defined over closed intervals and required that the topological dynamical system contains an uncountable set with no asymptotically stable points. This notion was later generalised by Blanchard et at.\cite{blanchard} using Li-Yorke pairs and scrambled sets. We defined a Li-Yorke pair to be two points which can be mapped arbitrarily far apart and mapped together under multiple iterations of the map, ensuring the iterates of these points are essentially scrambled throughout the entire set $X$. The final definition we covered was topological chaos, the definition of which being a topological dynamical system with positive topological entropy. We gave two equivalent definitions for topological entropy itself. The first was by Adler et al.\cite{adler} and defined topological entropy using open covers. The second was by Bowen and Dinaburg \cite{bowen} \cite{dinaburg} and uses the language of metric spaces. From these definitions we concluded that topological entropy is a non-negative real number describing the complexity of a topological dynamics system by the asymptotic mean growth in the number of distinguishable collections of orbits at an arbitrarily fine resolution. We then proved that if $(X, f)$ is a topological dynamical system where $f$ is an isometry we have that the topological entropy of the system is zero, and hence is not topologically chaotic. Furthermore we proved that topological entropy and hence topological chaos it a conjugacy invariant property of topological dynamical systems. We then proved that the the doubling map $(S^1, \mathcal{D})$ and shift map $(\Sigma_2, \sigma)$ are topologically chaotic with a positive topological entropy equal to $\log 2$. Finally, to conclude the chapter we noted that topological chaos implies Li-Yorke chaos, and if $(I, f)$ is a topological dynamical system defined over a closed interval $I$ then Devaney chaos is equivalent to topological chaos.

This concludes our study of different types of chaos in topological dynamics systems. Note that the types of chaos we have studied is not exhaustive. Other definitions of chaos in literature include distributional chaos, $\omega$ chaos, $P$ chaos, Block chaos, Wiggins chaos, etc. As we have seen throughout this text, there is no one,  universally accepted definition of chaos. However, all definitions we have covered in this text believe that chaotic topological dynamical systems are irregular and complex on even the most arbitrarily small scales. From our analysis, topological transitivity appears to be the most important property of topological dynamical systems which are Devaney chaotic. Hence it is important that chaotic topological dynamical systems can map points in any arbitrarily small neighbourhood to any other arbitrary neighbourhood under repeated iterations. Furthermore, the existence of an uncountable scrambled set is the important property in Li-Yorke chaos. This means that chaotic topological dynamical systems must map an uncountable set of pairs of points arbitrarily far apart and arbitrarily close together under repeated iterations. Furthermore, topological chaos has the important property of positive topological entropy. Hence, the asymptotic mean growth in the number of collections of orbits at arbitrarily fine resolutions must be positive in a chaotic topological dynamical system. All of these properties we have outlined ensure that chaos has the property of arbitrarily fine complexity. The difference between Devaney chaos, Li-Yorke chaos and topological chaos is essentially how they characterise this arbitrarily fine complexity in their definition.
